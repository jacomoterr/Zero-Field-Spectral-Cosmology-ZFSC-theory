\documentclass[12pt,a4paper]{article}
\usepackage[utf8]{inputenc}
\usepackage{amsmath}
\usepackage{geometry}
\geometry{margin=2.5cm}

\title{Note on an Entropy Marker for DUNE Neutrino Oscillations}
\author{Independent researcher}
\date{\today}

\begin{document}
\maketitle

\noindent
In a two-flavor reduction of $\nu_\mu \to \nu_e$ oscillations, the linear entropy of the reduced density matrix is:

\[
S_L = 1 - \mathrm{Tr}\,\rho^2.
\]

A possible empirical marker is:

\[
E_{\min S_L} \;\approx\; E_{\max P(\nu_\mu\to\nu_e)}.
\]

\textbf{How it can be tested:}  
DUNE’s broadband beam covers a wide energy range. By reconstructing both appearance probabilities and effective entropies, one can test whether the energy of minimal entropy aligns with the energy of maximal appearance probability.

\textbf{Why it may be important:}  
If valid, this relation could provide a new handle on analyzing CP violation effects and coherence in long-baseline oscillations.

\end{document}
