\documentclass[a4paper,12pt]{article}
\usepackage{geometry}
\geometry{margin=2.5cm}

% Языки
\usepackage[russian]{babel}

% --- Шрифты ---
\usepackage{fontspec}
\usepackage{unicode-math}
\setmainfont{CMU Serif}
\setmathfont{Latin Modern Math}

% --- Математика ---
\usepackage{amsmath}
\everymath{\displaystyle}
\emergencystretch=2em

% --- Графика ---
\usepackage{tikz}
\usepackage{pgfplots}
\pgfplotsset{compat=1.18}

\title{Квантово-геометрический ключ к двум Вселенным: \\
популярное изложение ZFSC v3.1}
\author{Евгений Монахов \\ LCC «VOSCOM ONLINE» Research Initiative https://zfsc-theory.org\\ evgeny.monakhov@voscom.online ORCID: 0009-0003-1773-5476}
\date{\today}

\begin{document}
\maketitle

\begin{abstract}
Мы предлагаем компактное популярное изложение модели Zero-Field Spectral Cosmology (ZFSC) v3.1.
Это полностью квантово-геометрический подход, в котором \emph{вся физика} описывается спектром самосопряжённых матриц конечного размера.
В модели есть две сопряжённые ветви (правая и левая), общий ``переключатель'' динамики и \emph{обратная связь}, которая быстро гасит приток хаотической энергии.
Мы показываем смысл каждого члена главного уравнения и формулируем проверяемые следствия --- без перегруженной технической детали.
\end{abstract}

\section*{1. Идея в одном абзаце}
ZFSC v3.1 рассматривает Вселенную как \emph{квантовый спектр геометрических операторов}.
Две зеркальные ветви (правая/левая) стартуют симметрично; затем включается гладкий переключатель $S(t)$, и в систему поступает энергия.
Однако специальная \emph{омега-обратная связь} быстро ``закрывает краник'': чем хаотичнее спектр и чем ярче его дискретные циклы, тем слабее подпитка.
Так формируются устойчивые \emph{плато} спектра (три ``оболочки''), а из локальных унитарных деформаций возникают характерные матрицы смешивания.

\section*{2. Главное уравнение}
Сердце модели --- эффективный гамильтониан
\begin{equation}\label{eq:main}
H_{\mathrm{eff}}(t)=S_\Omega(t)\Big[H_{\mathrm{core}}+\Theta\,\Xi\Big],\qquad
H_{\mathrm{core}}:=H_0+\sum_{i=1}^{8}\alpha_i O_i.
\end{equation}
\vspace{-0.7em}
\begin{itemize}
  \item $H_0$ --- базовая геометрия (\emph{самосопряжённая} матрица).
  \item $O_i$ --- восемь \emph{геометрических} операторов-путей (стабилизатор спектра, радиальный конфайнмент в три оболочки, локальные унитарные деформации и др.).
  \item $\Xi$ --- структурированный потенциал (отдельный геометрический канал).
  \item $S_\Omega(t)\in[0,1]$ --- гладкий переключатель динамики (включает эволюцию и ``насыщение'').
\end{itemize}
Коэффициенты $\alpha_i$ фиксируются конструктивно \emph{геометрией}, а не подбором под данные.

\section*{3. Две ветви реальности}
Модель содержит зеркальную пару базовых состояний
\begin{equation}\label{eq:branches}
H_\pm = H_0 \pm \varepsilon K_\chi,\qquad \varepsilon>0.
\end{equation}
Здесь $K_\chi$ --- малое хиральное (CP-нечётное) возмущение.
Две ветви эволюционируют \emph{параллельно} и собираются в блок-диагональную систему
\begin{equation}
H_{\mathrm{dual}}(t)=\mathrm{diag}\!\big(H_{\mathrm{eff}}^{(+)}(t),H_{\mathrm{eff}}^{(-)}(t)\big).
\end{equation}
Согласованность ветвей оценивается \emph{мерой связи}
\begin{equation}\label{eq:coupling}
I(H_+,H_-)=\int_0^\infty w(t)\,\big\|e^{-tH_+}-e^{-tH_-}\big\|_{\mathrm{HS}}^2\,dt\ge 0,
\end{equation}
которая обращается в ноль только при изоспектральности.

\section*{4. Переключатель и омега-запирание}
Подпитка геометрического канала $\Xi$ управляется параметром
\begin{equation}\label{eq:theta}
\Theta=\kappa\,\Phi^\ast\,\Omega,
\end{equation}
где $\kappa>0$ --- пропускная способность, $\Phi^\ast>0$ --- \emph{стабилизатор}, извлекаемый из спектра, а $\Omega\in(0,1]$ --- \emph{омега-фактор обратной связи}.
Интуитивно $\Omega$ убывает при росте хаотичности и цикличности спектра, что \emph{быстро} гасит подпитку канала.
Мы используем компактную формулу
\begin{equation}\label{eq:omega}
\Omega=\exp\!\Big(-\frac{h+C}{2}\Big),
\end{equation}
где $h$ --- отпечаток хаоса (насколько статистика уровней похожа на случайную матрицу), а $C$ --- сила дискретных циклов спектра (кратности $8,16,32,64$).
Чем больше $h$ и $C$, тем \emph{меньше} $\Omega$ и слабее вклад $\Theta\,\Xi$ в \eqref{eq:main}.

\section*{5. Что делают операторы пути}
В сумме $\sum \alpha_i O_i$ зашита \emph{геометрия динамики}:
\begin{itemize}
  \item стабилизация спектра (подавление ``пилообразности'', поддержка \emph{плато});
  \item радиальный конфайнмент, из-за которого выделяются \emph{три} устойчивые оболочки (аналог трёх поколений);
  \item локальные унитарные деформации, порождающие \emph{малые} углы смешивания в одном секторе и \emph{б\'ольшие} --- в другом.
\end{itemize}
Идея проста: если деформации действуют \emph{локально по блокам}, собственные подпространства сохраняются почти ортогональными, и матрица смешивания близка к единичной; если деформация ``скрещивает'' блоки --- углы крупнее.

\section*{6. Что модель обещает проверить}
Модель в текущей версии \emph{не вводит абсолютную шкалу}, поэтому предсказывает \emph{соотношения} и \emph{углы}.
Проверки (численные и феноменологические):
\begin{itemize}
  \item \textbf{Три устойчивые оболочки} в спектре $H_{\mathrm{eff}}(t)$ при разумных режимах переключателя $S_\Omega$.
  \item \textbf{Малые углы CKM-типа} для блочно-локальных деформаций; \textbf{большие PMNS-углы} при несоосности базисов.
  \item \textbf{Омега-запирание} по формуле \eqref{eq:omega}: при росте хаоса/цикличности вклад канала $\Theta\,\Xi$ заметно убывает во времени.
  \item \textbf{Дуальность L/R}: при малом $\varepsilon$ и общей стабилизации $\Phi^\ast$ спектры ветвей согласуются (мала мера \eqref{eq:coupling}).
\end{itemize}

\section*{7. Почему это не ``подгонка''}
Коэффициенты и операторы определяются \emph{геометрически}, а не подбором к данным; численные эксперименты служат только для проверки следствий и оценки устойчивости плато.
Мы отделяем \emph{чистую геометрию} (определения, операторы, формулы \eqref{eq:main}--\eqref{eq:omega}) от феноменологии (сравнение соотношений и углов).

\section*{8. Вывод}
ZFSC v3.1 --- это \emph{квантово-геометрический ключ} к двум зеркальным ветвям реальности.
Главное уравнение \eqref{eq:main} задаёт динамику через переключатель и омега-обратную связь: система быстро выходит из режима ``потопа'' и стабилизируется на спектральных плато.
Из локальной геометрии следуют характерные матрицы смешивания.
Следующий шаг --- систематическая численная верификация безразмерных предсказаний на стандартизованных наборах геометрий.

\bigskip
\noindent\textbf{Основные формулы (собрание):}
\begin{align*}
& H_{\mathrm{eff}}(t)=S_\Omega(t)\Big[H_0+\sum_{i=1}^{8}\alpha_i O_i+\Theta\,\Xi\Big], \\[0.3em]
& H_\pm = H_0 \pm \varepsilon K_\chi,\qquad
H_{\mathrm{dual}}(t)=\mathrm{diag}\!\big(H_{\mathrm{eff}}^{(+)},H_{\mathrm{eff}}^{(-)}\big), \\[0.3em]
& \Theta=\kappa\,\Phi^\ast\,\Omega,\qquad
\Omega=\exp\!\Big(-\frac{h+C}{2}\Big), \\[0.3em]
& I(H_+,H_-)=\int_0^\infty w(t)\,\big\|e^{-tH_+}-e^{-tH_-}\big\|_{\mathrm{HS}}^2\,dt\;.
\end{align*}

\end{document}

