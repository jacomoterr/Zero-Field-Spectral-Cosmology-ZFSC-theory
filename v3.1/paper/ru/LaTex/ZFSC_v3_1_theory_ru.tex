% =========================================================
% ZFSC v3.1 — Двуветвевой спектральный каркас (квантово-геометрическая теория)
% VERSION: 3.1.0
% DATE: 2025-09-15
% DOC_ID: ZFSC-v3.1-dual-chiral
% =========================================================

\documentclass[a4paper,12pt]{article}

% --- Поля страницы ---
\usepackage{geometry}
\geometry{margin=2.5cm}

% --- Язык ---
\usepackage[russian]{babel}

% --- Шрифты ---
\usepackage{fontspec}
\usepackage{unicode-math}
\setmainfont{CMU Serif}
\setmathfont{Latin Modern Math}

% --- Математика ---
% --- \usepackage{amsmath} ---
\everymath{\displaystyle}
\emergencystretch=2em

% --- Графика (на будущее, не используется в тексте) ---
\usepackage{tikz}
\usepackage{pgfplots}
\pgfplotsset{compat=1.18}

% --- Ссылки ---
\usepackage{hyperref}
\pdfstringdefDisableCommands{%
  \def\Omega{Ω}%
  \def\Xi{Ξ}%
  \def\Theta{Θ}%
  \def\Phi{Φ}%
  \def\Psi{Ψ}%
  \def\Sigma{Σ}%
  \def\Delta{Δ}%
  \def\Lambda{Λ}%
  \def\Gamma{Γ}%
  \def\Pi{Π}%
  \def\_{}%
}

% --- Макросы ---
\newcommand{\Tr}{\mathrm{Tr}}
\newcommand{\Id}{\mathbb{I}}

\begin{document}

\begin{center}
{\Large \textbf{Zero-Field Spectral Cosmology (ZFSC) v3.1}}\\[4pt]
{\large Двуветвевой квантово-геометрический каркас со спектральной стабилизацией}\\[8pt]
Версия документа: \texttt{3.1.0} \quad (15 сентября 2025)
\end{center}

\begin{abstract}
Представлена версия ZFSC v3.1 как двуветвевой (лево-/правохиральный) квантово-геометрический каркас, в котором наблюдаемые массы и смешивания выводятся из спектра эффективного гамильтониана на дискретной геометрии. 
Ключевые элементы: общий спектральный стабилизатор $\Phi^*$, квантовый переключатель $S_{\Omega}(t)$, резонансный ОМ-фактор $\Omega$ и пропускная способность резервуара $\kappa$, собранные в сквозной масштаб $\Theta=\kappa/(\Phi^*\,\Omega)$. 
Дана строгая операторная запись, процедуры извлечения $\Phi^*$ и $\Omega$ из спектральных данных, а также протокол численной валидации с метриками устойчивости плато.
\end{abstract}

\tableofcontents

\section{Спектральный принцип и постановка задачи}
Фундаментальная гипотеза ZFSC: наблюдаемый спектр масс и смешиваний определяется собственными значениями и собственными векторами эффективного гамильтониана, построенного на дискретной квантовой геометрии.
\begin{equation}
m_n = \lambda_n\!\big(H_{\mathrm{eff}}(t)\big),\qquad 
\mathrm{Mix} = U_A^\dagger U_B,
\end{equation}
где $U_{A,B}$ --- матрицы собственных векторов в секторах $A,B$ (например, $u/d$, $\ell/\nu$). В версии v3.1 гамильтониан имеет двуветвевую структуру и явную временную модуляцию включения.

\section{Слоистое гильбертово пространство и геометрия}
Пусть $\mathcal H=\bigoplus_{\ell}\mathcal H_\ell$ --- слоистое гильбертово пространство. 
Дискретная геометрия слоя $\ell$ задаётся связностью $C_\ell$ и лапласианом $L_\ell=D_\ell-C_\ell$.
Базовый геометрический гамильтониан $H_0$ строится на $\{L_\ell\}$ и может включать квазипериодические/квазикристаллические компоненты.

\section{Двуветвевой (хиральный) каркас}\label{sec:dual}
ZFSC v3.1 описывает две сопряжённые ветви --- левую ($L$) и правую ($R$). Эффективный гамильтониан блочно-диагонален:
\begin{equation}
H_{\mathrm{eff}}(t)=S_{\Omega}(t)\,
\mathrm{diag}\!\Big(H_L,\,H_R\Big),
\qquad
H_B = H_0 + \sum_i \alpha_i\,O_i \;+\; \Theta\,\Xi,\quad B\in\{L,R\}.
\label{eq:Heff31}
\end{equation}
Здесь $O_i$ --- фиксированный набор операторов (см. \S\ref{sec:operators}); $\Xi$ --- оператор резервуара (см. \S\ref{sec:reservoir}); $\Theta$ --- сквозной масштаб подпитки (см. \S\ref{sec:theta}); $S_{\Omega}(t)$ --- переключатель включения (\S\ref{sec:switch}). Важное \emph{требование симметрии} v3.1: \textbf{одни и те же} стабилизатор $\Phi^*$ и резонансный фактор $\Omega$ применяются одновременно к обеим ветвям $L/R$ (общая ``крышка'').

\section[Набор операторов Oi (содержит v3.0 как подмножество)]{Набор операторов \(O_i\) (содержит v3.0 как подмножество)}\label{sec:operators}
\begin{enumerate}
\item \textbf{Структурированная энтропия:} $O_{\mathrm{ent}}=\Phi(L)$, где $L=\bigoplus_\ell L_\ell$ и $\Phi$ --- монотонная эрмитова функция (типично $X$ или $X^2$ при нормировке).
\item \textbf{Каналы подпитки полей:} $O_{\mathrm{ch}}=\sum_{\alpha\in\{\mathrm{gr,em,wk,st}\}}\Pi_\alpha$ (проекторы на канальные подсекторы).
\item \textbf{Радиальный конфайнмент (поколения):} $O_{\mathrm{rad}}=\sum_\ell f(B_\ell)$, где $B_\ell=g_r L^{(\ell)}_{\mathrm{rad}}+V^{(\ell)}_{\mathrm{cap}}(r)$; три нижних собственных уровня $\epsilon_{\ell,g}$ ($g=1,2,3$) реализуют поколения.
\item \textbf{Топология смешиваний:} $O_{\mathrm{mix}}=(W_g\!\otimes\!\Id_s)H_0(W_g^\dagger\!\otimes\!\Id_s)-H_0$, где $W_g(\kappa)=e^{-i\kappa K_g}$ с локальными генераторами $K_g$.
\item \textbf{Фиксация устойчивых плато:} $O_{\mathrm{stab}}=-\sum_n \Xi_n P_n$ (см. \S\ref{sec:stability}); Линдблад-реализация даёт полную положительность.
\item \textbf{Циклическая геометрия:} $O_{\mathrm{cyc}}=f(L_{C_P})$, $L_{C_P}=2\Id-(X+X^\dagger)$ на цикле $C_P$.
\end{enumerate}
Коэффициенты $\alpha_i$ --- малые, иерархия норм сохраняет симметрии базовой геометрии.

\section[Спектральный стабилизатор Фи*: определение и извлечение]{Спектральный стабилизатор $\Phi^*$: определение и извлечение}\label{sec:phi}
Стабилизатор $\Phi^*$ --- безразмерная величина, задающая норму стабилизирующего вклада и масштаб резервуара. Он \emph{извлекается из спектра} по одной из процедур:
\begin{description}
\item[(A) Near-degeneracy:] анализ отношений ближайших собственных значений $\lambda_{n\pm 1}/\lambda_n$ и их сходимости к плато.
\item[(B) Gap-sealing:] использование относительных зазоров $g_n=(\lambda_{n+1}-\lambda_n)/\overline{\lambda}$ и гармонического среднего по стабильным модам.
\item[(C) Chaos/modes ratio:] сравнение энерговклада хаотического фона и мод-плато; симметризация между $L/R$ через геометрическое среднее.
\end{description}
В хаос-нормировке обычно $\Phi^*\approx 1\pm \varepsilon$ (малое отклонение фиксируется по метрикам стабильности).

\section[ОМ-фактор Ω и переключатель SΩ]{ОМ-фактор \(\Omega\) и переключатель $S_{\Omega}(t)$}\label{sec:switch}
ОМ-фактор $\Omega$ --- безразмерный резонансный множитель, характеризующий \emph{эффективную ширину} ``горлышка'' включения динамики. Он влияет как на амплитуду, так и на фазовый режим переключателя. В v3.1:
\begin{equation}
S_{\Omega}(t)=S\big(t;\Omega\big),\qquad
\Omega \in \mathbb{R}_+,\ \ \Omega\approx 1\pm \varepsilon_{\Omega}.
\end{equation}
Практические формы $S(t;\Omega)$: сглаженная ступенька (сигмоида), ступенчатый режим, либо узкополосная модуляция для сканов по резонансам. $\Omega$ извлекается из частотно-временных характеристик устойчивости плато (см. \S\ref{sec:metrics}).

\section[Резервуар и сквозной масштаб Θ]{Резервуар и сквозной масштаб \(\Theta\)}\label{sec:theta}
Резервуар управляет дозированной подпиткой спектра. В v3.1 его вклад собран в \emph{сквозной масштаб}
\begin{equation}
\Theta \;=\; \frac{\kappa}{\Phi^*\;\Omega},
\label{eq:theta}
\end{equation}
где $\kappa$ --- пропускная способность. Квантование $\kappa$ допускает дискретные наборы (например, $\{3,5,8,13\}$) либо масштабирование $\kappa\sim\sqrt{N}$, что удобно при изменении размерности матрицы.

\section[Оператор резервуара Ξ]{Оператор резервуара \(\Xi\)}\label{sec:reservoir}
Оператор $\Xi$ действует в канальных подсекторах и реализует дозированную подачу:
\begin{equation}
\Xi \;=\; \sum_{\alpha\in\{\mathrm{gr,em,wk,st}\}} \xi_\alpha\,\Pi_\alpha,
\qquad \xi_\alpha\ge 0,\quad \|\Xi\|\ll \|H_0\|.
\end{equation}
Замечание о нотации: символ $\Xi$ для резервуара отличается от индикаторов устойчивости $\Xi_n$ в \S\ref{sec:stability} (контекст определяет значение).

\section{Полная форма эффективного гамильтониана v3.1}
Собрав определения \eqref{eq:Heff31}--\eqref{eq:theta}, получаем для каждой ветви $B\in\{L,R\}$:
\begin{align}
H_B &= H_0 
+ \alpha_{\mathrm{ent}}\,\Phi(L)
+ \alpha_{\mathrm{ch}} \sum_{\alpha}\Pi_\alpha
+ \alpha_{\mathrm{rad}}\sum_\ell f(B_\ell)
+ \alpha_{\mathrm{mix}}\Big[(W_g\!\otimes\!\Id_s)H_0(W_g^\dagger\!\otimes\!\Id_s)-H_0\Big] \nonumber\\
&\quad
+ \alpha_{\mathrm{cyc}}\,f(L_{C_P})
- \alpha_{\mathrm{stab}} \sum_n \Xi_n P_n
\;+\;
\Theta\,\Xi
\;-\;
\mu\,\Id,
\label{eq:Heff_full}
\end{align}
и
\begin{equation}
H_{\mathrm{eff}}(t)=S_{\Omega}(t)\,
\mathrm{diag}\!\big(H_L,\,H_R\big).
\end{equation}
Иерархия норм сохраняет базовые симметрии: $\| \alpha_{\mathrm{ent}}\Phi(L)\| \gtrsim \| \alpha_{\mathrm{ch}}\sum \Pi_\alpha \| \gtrsim \|\alpha_{\mathrm{rad}}\sum f(B_\ell)\| \gtrsim \|\alpha_{\mathrm{mix}}(\cdot)\| \gtrsim \|\alpha_{\mathrm{stab}}\sum \Xi_n P_n\| \gtrsim \|\alpha_{\mathrm{cyc}}f(L_{C_P})\| \gg \|\Theta\,\Xi\|$.

\section{Поколения и смешивания}
Три поколения появляются как три нижние связанные моды $B_\ell$ ниже порога континуума. Отсутствие четвёртой моды при типичных $V^{(\ell)}_{\mathrm{cap}}(r)$ реализует естественный запрет. 
Топология смешиваний через локальный $W_g$ обеспечивает слабые дальние переходы: малые углы в кварковом секторе и большие --- в лептонном возникают без подгонки.

\section{Фиксация устойчивости плато и Линдблад-реализация}\label{sec:stability}
Индикатор устойчивости для моды $n$ в скане параметра $t$:
\begin{equation}
\Xi_n = \exp\!\Big(- a\frac{|\dot{\lambda}_n|}{\varepsilon_1}
- b\frac{|\ddot{\lambda}_n|}{\varepsilon_2}
- c\,\delta_{\mathrm{ring}}(n)\Big)\in(0,1],\qquad P_n=|\psi_n\rangle\langle\psi_n|.
\end{equation}
Энергетическая фиксация $-\alpha_{\mathrm{stab}}\sum_n \Xi_n P_n$ повышает робастность плато. 
Динамическая (полностью положительная) форма на уровне плотности $\rho$:
\begin{equation}
\dot\rho=-i[H_B,\rho]+\sum_j\big(D_j\rho D_j^\dagger-\tfrac12\{D_j^\dagger D_j,\rho\}\big),\qquad D_j=\sqrt{\beta_j}\,P^{(j)}_{\mathrm{stable}}.
\end{equation}

\section{Циклическая геометрия и угловая дискретизация}
На цикле $C_P$ имеем $L_{C_P}=2\Id-(X+X^\dagger)$, $X|m\rangle=|m+1\rangle$, $Z|m\rangle=\omega^m|m\rangle$, $\omega=e^{2\pi i/P}$. 
Выбор $P\in\{8,16,32,64\}$ (и их композиции) формирует устойчивые угловые щели, согласующиеся с плато.

\section{Нормировка и перевод в физические массы}
Общий сдвиг снимается $\mu$ (например, через условие $\Tr H_B=0$). Абсолютный масштаб задаётся функционалом резервуара:
\begin{equation}
\Lambda_* \propto \Omega(\rho)=\Omega_0+\xi\,\Tr\!\big(\rho\,\Phi(L)\big),\qquad 
m_n^{\mathrm{phys}}=\Lambda_*\,g\!\left(\frac{\lambda_n}{\Lambda_*}\right),
\end{equation}
где $g$ --- фиксированная монотонная функция (на первой итерации $g(x)=x$).

\section{Метрики устойчивости и извлечение параметров}\label{sec:metrics}
Для оценки плато используются:
\begin{itemize}
\item \textbf{plateau\_persistence:} доля $t$-окна, где $\Xi_n \ge \tau$ (порог).
\item \textbf{shell\_purity:} чистота выбранной спектральной оболочки (отсутствие скрещиваний/пересадок мод).
\item \textbf{gap\_ratio:} нормированное отношение локальных зазоров $g_n$.
\end{itemize}
Извлечение $\Phi^*$ и $\Omega$ опирается на стабилизированные наборы мод с максимальной $plateau\_persistence$ и согласованными $gap\_ratio$. Симметризация $L/R$ делается через геометрическое среднее соответствующих оценок.

\section{Предсказания и проверяемые следствия v3.1}
\label{sec:predictions}
\begin{enumerate}
\item \textbf{Три поколения} из радиальной квантизации; отсутствие четвёртой моды ниже порога.
\item \textbf{CKM/PMNS-структура} из локальной топологии смешиваний; подавление дальних углов в кварках и усиление в лептонах.
\item \textbf{Угловые щели} по выборным $P$ и их композициям.
\item \textbf{Малая $\Lambda$} как следствие слабой дозированной подпитки (малость $\Theta\,\Xi$).
\item \textbf{Робастность к шумам} геометрии из-за присутствия $\Phi^*$ и энергетической/линдбладовской фиксации.
\item \textbf{Двуветвевой симметрический стабилизатор:} совпадение $\Phi^*$ и $\Omega$ для $L/R$ является условием согласованной динамики; отклонение диагностируется по рассогласованию метрик.
\end{enumerate}

\section{Численная валидация: протокол v3.1}
\begin{enumerate}
\item Синтезировать $\{L_\ell\}$, построить $H_0$, задать набор $O_i$ и $L_{C_P}$; выбрать $P$.
\item Выбрать класс $S_{\Omega}(t)$ и сетку параметров; задать кванты для $\kappa$; инициализировать оценки $\Phi^*,\Omega$.
\item Собрать $H_B$ по \eqref{eq:Heff_full} для $B=L,R$; сформировать $H_{\mathrm{eff}}(t)$.
\item Диагонализовать, вычислить метрики \S\ref{sec:metrics}; извлечь $\Phi^*$ (варианты A/B/C) и $\Omega$; симметризовать $L/R$.
\item Выполнить сканы по $\alpha_i$, $P$, классам $S_{\Omega}$, дискретам $\kappa$; сравнить долю/качество плато с/без каждого $O_i$.
\item Отчёт: средние/медианные метрики, распределения щелей, карты устойчивости; проверка предсказаний \S~\ref{sec:predictions}.
\end{enumerate}

\section{Симметрии и совместимость членов}
При малых $\alpha_i$ и $\Theta$ базовые симметрии $H_0$ сохраняются. Коммутационные свойства между $O_i$ выбираются так, чтобы не индуцировать избыточное спонтанное нарушение симметрий. Разумная иерархия норм приведена после \eqref{eq:Heff_full}.

\section*{Заключение}
ZFSC v3.1 формализует двуветвевой (хиральный) спектральный каркас с общим стабилизатором $\Phi^*$ и резонансным фактором $\Omega$. Эффективный гамильтониан \eqref{eq:Heff_full} включает необходимые геометрические и топологические вклады и допускает строгую численную валидацию без подгонки параметров. Каркас воспроизводит ключевые структурные свойства спектра частиц и задаёт программу последующих проверок.

\end{document}