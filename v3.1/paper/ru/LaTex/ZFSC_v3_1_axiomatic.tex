
\documentclass[a4paper,12pt]{article}
\usepackage{geometry}
\geometry{margin=2.5cm}

% Языки
\usepackage[russian]{babel}

% --- Шрифты ---
\usepackage{fontspec}
\usepackage{unicode-math}
\setmainfont{CMU Serif}
\setmathfont{Latin Modern Math}

% --- Математика ---
\usepackage{amsmath}
\everymath{\displaystyle}
\emergencystretch=2em

% --- Графика ---
\usepackage{tikz}
\usepackage{pgfplots}
\pgfplotsset{compat=1.18}

\title{Zero-Field Spectral Cosmology (ZFSC) v3.1: \\
Аксиоматическая формулировка квантово-геометрической двухветвевой модели}
\author{Евгений Монахов \\ LCC «VOSCOM ONLINE» Research Initiative https://zfsc-theory.org\\ evgeny.monakhov@voscom.online ORCID: 0009-0003-1773-5476}
\date{\today}

\begin{document}
\maketitle

\begin{abstract}
Представлена строгая аксиоматическая версия квантово-геометрической модели ZFSC v3.1.
Модель описывает две сопряжённые ветви (правохиральную и левохиральную) на единой дискретной геометрии, задаваемой самосопряжёнными операторами конечного размера.
Динамика включается гладкой функцией переключателя $S(t)\in[0,1]$ и содержит унитарную обратную связь ``резервуар--омега'', модифицирующую вклад структурированного потенциала $\Xi$ через параметр $\Theta=\kappa\,\Phi^\ast\,\Omega$.
Все члены уравнения определены как чисто геометрические и чисто квантовые функционалы от базовой матрицы.
В работе фиксируются определения, аксиомы, леммы об унитарности эволюции и существовании фикс-точки для $\Phi^\ast$, а также формализуются спектральные метрики (хаотичность $h$, цикличность $C$), мера связи ветвей $I(H_+,H_-)$, понятие плато спектра и механизм смешиваний.
Формализм предсказывает только безразмерные отношения (массовые и угловые), абсолютный масштаб в этой версии не вводится.
\end{abstract}

\section{Формальная постановка}
\subsection*{Пространство и базовые объекты}
\textbf{A1 (Конечномерная квантовая система).}
Фиксируется гильбертово пространство $\mathcal{H}=\mathbb{C}^N$, $N\in\mathbb{N}$, со стандартным скалярным произведением.
Все операторы заданы матрицами размера $N\times N$ над $\mathbb{C}$ и являются самосопряжёнными, если не указано иное.

\textbf{A2 (Базовая геометрия).}
Задана базовая самосопряжённая матрица $H_0=H_0^\ast$, индуцированная дискретной геометрией (например, через ориентированный/неориентированный граф с матрицей смежности $A$ и/или лапласианом $L$).
Спектр $H_0$ действителен и конечен.

\textbf{A3 (Операторы пути).}
Определяется набор самосопряжённых операторов $\{O_i\}_{i=1}^8$, являющихся \emph{геометрическими функционалами} от $A$, $L$ и/или $H_0$.
В частности, фиксируется следующая конструкция (с указанием требований на регулярность):
\begin{align}
& O_1 := H_0, \label{eq:O1}\\
& O_2 := \Phi(L), \quad \Phi:\mathbb{R}_{\ge 0}\to\mathbb{R} \text{ --- борелев функционал, монотонный и липшицев}, \label{eq:O2}\\
& O_3 := 0 \quad \text{(см. ниже: вклад резервуара реализован через } \Theta\cdot\Xi\text{)}, \label{eq:O3}\\
& O_4 := \sum_{\ell} \zeta_\ell f(B_\ell), \quad B_\ell \text{ --- проекторы ``оболочек'', } f \text{ --- ограниченный борелев функционал}, \label{eq:O4}\\
& O_5 := K_g(H_0):=(W_g\otimes I)H_0(W_g^\dagger\otimes I)-H_0, \label{eq:O5}\\
& O_6 := g(L_{C,P}), \quad P\in\{8,16,32,64\}, \quad g \text{ --- ограниченный функционал}, \label{eq:O6}\\
& O_7 := P_{\text{shell}} \quad \text{(проекторы/операторы упорядочивания спектральных оболочек)}, \label{eq:O7}\\
& O_8 := 0 \quad \text{(нормировочный сдвиг не используется в v3.x).} \label{eq:O8}
\end{align}
Все $O_i$ предполагаются ограниченными: $\|O_i\|<\infty$.

\textbf{A4 (Структурированный потенциал).}
Фиксируется самосопряжённый ограниченный оператор $\Xi=\Xi^\ast$ (\emph{геометрический потенциал}), не сводимый к линейной комбинации $\{O_i\}$.
Он будет входить в гамильтониан с коэффициентом $\Theta$ как отдельный канал.

\textbf{A5 (Две ветви).}
Вводятся две сопряжённые ветви $H_\pm$ посредством малой CP-нечётной деформации базовой геометрии:
\begin{equation}
H_\pm := H_0 \pm \varepsilon K_\chi, \qquad \varepsilon>0,\quad K_\chi=K_\chi^\ast,
\end{equation}
где $K_\chi$ --- самосопряжённый ``хиральный'' оператор малой нормы $\|K_\chi\|\ll\|H_0\|$.
Хиральный скаляр $\chi$ задаётся как спектрально-геометрический инвариант базового ориентированного графа (например, $\chi=\frac{1}{N^3}\,\mathrm{Tr}\,J^3$ для антисимметризированной части $J$).

\textbf{A6 (Переключатель).}
Функция $S(t)\in C^1(\mathbb{R}\to[0,1])$ (\emph{включатель}) --- гладкая сигмоидальная огибающая, $S(t)=0$ для $t\ll 0$, $S(t)\to1$ при $t\to+\infty$.

\textbf{A7 (Обратная связь).}
Определяется параметр
\begin{equation}
\Theta := \kappa\,\Phi^\ast\,\Omega, \qquad \kappa>0,
\end{equation}
где $\Phi^\ast>0$ --- \emph{стабилизатор}, извлекаемый из спектра, а $\Omega\in(0,1]$ --- \emph{омега-фактор обратной связи}, определяемый исключительно спектрально-геометрическими метриками.

\subsection*{Эффективный гамильтониан}
\textbf{A8 (Эффективная динамика).}
Эффективный гамильтониан записывается как
\begin{equation}\label{eq:Heff}
H_{\mathrm{eff}}(t) = S_\Omega(t)\Big[ H_{\mathrm{core}} + \Theta\,\Xi \Big], \qquad
H_{\mathrm{core}} := H_0 + \sum_{i=1}^8 \alpha_i O_i,
\end{equation}
где $S_\Omega\in C^1$, $0\le S_\Omega\le 1$ --- допустимо взять $S_\Omega\equiv S$ на базовом уровне.
Все коэффициенты $\alpha_i\in\mathbb{R}$ фиксированы конструктивно (геометрией), а не путём подгонки.

\textbf{A9 (Дуальная система).}
Полная двухветвная система представляется блок-диагонально:
\begin{equation}
H_{\mathrm{dual}}(t) := \mathrm{diag}\big(H_{\mathrm{eff}}^{(+)}(t),\, H_{\mathrm{eff}}^{(-)}(t)\big),
\end{equation}
где $H_{\mathrm{eff}}^{(\pm)}$ строится по \eqref{eq:Heff} на ветвях $H_\pm$ при разделяемом стабилизаторе $\Phi^\ast$ и омега-факторе $\Omega$ (см. ниже симметризации L/R).

\section{Спектральные метрики и омега-фактор}
\subsection*{Хаотичность $h$}
Пусть $P(s)$ --- эмпирическое распределение ближайших расстояний между соседними собственными значениями (после стандартной развёрстки).
Обозначим $P_{\mathrm{GOE}}(s)$ и $P_{\mathrm{Pois}}(s)$ --- эталонные распределения (Гауссова ортогональная ансамблевая статистика и пуассоновская соответственно).
Определим метрику хаотичности как
\begin{equation}
h := \min\Big\{ D_{\mathrm{KS}}\big(P, P_{\mathrm{GOE}}\big),\; D_{\mathrm{KS}}\big(P, P_{\mathrm{Pois}}\big) \Big\},
\end{equation}
где $D_{\mathrm{KS}}$ --- статистика Колмогорова--Смирнова (или другая фиксированная метрика распределений).

\subsection*{Цикличность $C$}
Зададим спектральную функцию $S(\omega)$ (например, спектр автокорреляции отсортированных $\{\lambda_k\}$).
Для фиксированного набора гармоник $\mathcal{P}=\{8,16,32,64\}$ определим
\begin{equation}
C := \sum_{P\in\mathcal{P}} w_P\, \frac{\int_{\mathcal{B}_P} S(\omega)\,d\omega}{\int_{\mathbb{R}_+} S(\omega)\,d\omega},
\end{equation}
где $\mathcal{B}_P$ --- узкие частотные окна вокруг кратностей $P$ (фиксируются заранее), $w_P>0$, $\sum w_P=1$.

\subsection*{Омега-фактор}
\begin{equation}
\Omega := \exp\Big(-\frac{h+C}{2}\Big)\in (0,1].
\end{equation}
Заметим, что $h\ge0$, $C\ge0$, и при исчезающих сигнатурах хаоса/цикличности $\Omega\to1$.

\section{Стабилизатор \texorpdfstring{$\Phi^\ast$}{Phi*} и фикс-точка}
\subsection*{Два режима извлечения}
\textbf{Режим (Core).}
Определяется из спектра $H_{\mathrm{core}}$ \eqref{eq:Heff}: $\Phi^\ast := F\big(\mathrm{Spec}(H_{\mathrm{core}})\big)$, где $F$ --- фиксированный положительный функционал (например, через почти-вырождение уровней, усредняющие щели, и т.\,п.). В этом режиме круговая ссылка отсутствует.

\textbf{Режим (Fixed-Point).}
Задаётся уравнение на фикс-точку
\begin{equation}
\Phi^\ast = F\big(\mathrm{Spec}(H_{\mathrm{core}}+\kappa\,\Phi^\ast\,\Xi)\big),
\end{equation}
где $F$ --- непрерывный положительный функционал. При компактности образа и непрерывности существует по теореме Брауэра/Шаудера хотя бы одна фикс-точка.

\subsection*{Симметризация L/R}
Стабилизатор $\Phi^\ast$ применяется \emph{одинаково} к обеим ветвям $(+)$ и $(-)$, т.\,е. используется одна и та же величина, извлечённая из общих спектральных инвариантов (\emph{геометрическое среднее} вариантов L/R при необходимости).

\section{Мера связи ветвей}
\subsection*{Определение}
Для двух самосопряжённых операторов $H_\pm$ определим меру
\begin{equation}
I(H_+,H_-):=\int_0^\infty w(t)\,\big\| e^{-tH_+}-e^{-tH_-}\big\|^2_{\mathrm{HS}}\,dt,
\end{equation}
где $w(t)\ge0$, $\int_0^\infty w(t)\,dt<\infty$, а $\|\cdot\|_{\mathrm{HS}}$ --- гильберт--шмидтовская норма.
Тогда $I\ge0$. Если $I=0$, то $e^{-tH_+}\equiv e^{-tH_-}$ для почти всех $t$, откуда следует изоспектральность.

\section{Эволюция и унитарность}
\subsection*{Условие унитарности}
Пусть $S_\Omega\in C^1$, а $H_{\mathrm{core}}$ и $\Xi$ --- ограниченные самосопряжённые.
Тогда $H_{\mathrm{eff}}(t)$ \eqref{eq:Heff} --- кусочно-гладкая самосопряжённая матрица.
Соответствующий эволюционный оператор $U(t)$ существует и унитарен (стандартная конструкция для конечномерных $H(t)$).

\section{Спектральные плато и смешивания}
\subsection*{Плато}
Множество собственных значений $\{\lambda_k\}$ разбивается на кластеры $\{\mathcal{C}_m\}$, если
\begin{equation}
\max_{\lambda_i,\lambda_j\in\mathcal{C}_m} |\lambda_i-\lambda_j| \le \delta_{\mathrm{in}}, \qquad
\min_{\lambda_i\in\mathcal{C}_m,\,\lambda_j\in\mathcal{C}_n,\,m\ne n} |\lambda_i-\lambda_j| \ge \delta_{\mathrm{out}},
\end{equation}
где $\delta_{\mathrm{in}}\ll\delta_{\mathrm{out}}$ --- фиксированные пороги.
Определим метрики: \emph{устойчивость плато} (persistency) как долю $t$ в окне, где кластеризация сохраняется, и \emph{чистоту оболочек} как долю массы проекторов $B_\ell$, попадающей в свои кластеры.

\subsection*{Смешивания}
Пусть $P_u,P_d,P_\ell,P_\nu$ --- проекторы на сектора (u/d/л/ν), а локальные унитарные деформации $W_g$ в \eqref{eq:O5} действуют блочно-локально.
Тогда матрицы смешивания
\begin{equation}
\mathrm{CKM}:=U_u^\dagger U_d, \qquad \mathrm{PMNS}:=U_\ell^\dagger U_\nu,
\end{equation}
где $U_s$ состоят из собственных векторов соответствующих ограничений $P_s H_{\mathrm{eff}} P_s$, демонстрируют малость внеблочных элементов при малых блочно-локальных возмущениях (оценки возмущений собственных подпространств стандартны).

\section{Отсутствие абсолютной шкалы}
В версии v3.x член $-\mu I$ \emph{не используется}, следовательно, абсолютный энергетический масштаб не фиксируется.
Предсказываются безразмерные отношения спектральных величин и углы смешивания.

\section{Фальсифицируемые заявления}
\begin{itemize}
\item \textbf{Три устойчивые спектральные оболочки} (три ``поколения'') при разумном выборе $B_\ell$ и $f(B_\ell)$.
\item \textbf{Малость углов CKM-типа} при блочно-локальных унитарных деформациях $K_g$; \textbf{б\'ольшие углы PMNS-типа} при несоосности соответствующих базисов.
\item \textbf{Омега-запирание:} при увеличении хаотичности $h$ и/или цикличности $C$ омега-фактор $\Omega$ убывает $\to$ вклад канала $\Theta\,\Xi$ эффективно затухает.
\item \textbf{Дуальность L/R:} симметричный стабилизатор $\Phi^\ast$ обеспечивает согласованность спектральных инвариантов обеих ветвей при малом $\varepsilon$ и ограниченной мере $I(H_+,H_-)$.
\end{itemize}

\section{Численный протокол и воспроизводимость}
\subsection*{Версионирование}
Каждая программа обязана логировать константы \texttt{VERSION} и \texttt{PROGRAM\_HASH}; конфигурации содержат \texttt{CONFIG\_HASH}; при запуске сверять и писать в лог полный набор (версия/хэши) и результат проверки.

\subsection*{Структура прогонов}
Запуски разделяются по стадиям (\texttt{runs/T0}, \texttt{runs/T1}, \dots) и по временным меткам; результаты сохраняются в CSV/JSON, метаданные --- отдельно; визуализации не генерируются, если не запрошены явно.

\subsection*{Извлечение \texorpdfstring{$\Phi^\ast$}{Phi*}}
Допускаются режимы \textbf{A/B/C}: (A) почти-вырождения; (B) ``gap-sealing'' (гармоническое усреднение щелей); (C) энергия хаос/моды.
Для дуальности L/R стабилизатор симметризуется геометрическим средним.

\section{Заключение}
Сформулирована аксиоматическая версия ZFSC v3.1 как полностью квантово-геометрической двухветвевой модели на конечномерном уровне.
Даны строгие определения всех членов гамильтониана, метрик обратной связи и меры связи ветвей, а также заявлены проверяемые свойства спектра и смешиваний.
Дальнейшая работа включает численную верификацию безразмерных предсказаний и анализ предельных переходов $N\to\infty$.

\appendix

\section{Рецепты для \texorpdfstring{$\Phi^\ast$}{Phi*}}
\paragraph{A (Почти-вырождение).} Определить $\Phi^\ast$ через средние отношения соседних щелей в выбранных диапазонах спектра; обеспечить непрерывность по данным и ограниченность снизу/сверху.
\paragraph{B (Gap-sealing).} Рассчитать гармоническое среднее относительных щелей между кластерами; \ $\Phi^\ast$ --- монотонный функционал этого среднего.
\paragraph{C (Chaos/modes).} Ввести показатель соотношения хаотической и модовой энергии (по спектральным проекциям) и положить $\Phi^\ast$ равным гладкому функционалу от этого показателя.

\section{Определения метрик плато}
\paragraph{Устойчивость плато.} Доля временных точек в окне, где кластеризация по порогам $(\delta_{\mathrm{in}},\delta_{\mathrm{out}})$ инвариантна с точностью до перестановок внутри кластеров.
\paragraph{Чистота оболочек.} Для проекторов $B_\ell$ --- отношение суммы диагональных масс в собственном кластере к суммарной массе проектора.

\section{Замечания по дуальности}
\paragraph{Фиксация $\Phi^\ast$.} При различиях спектров ветвей возможны отдельные оценки $\Phi^\ast_\pm$; использовать симметризацию $\Phi^\ast=\sqrt{\Phi^\ast_+\,\Phi^\ast_-}$.
\paragraph{Мера сцепки.} Для выбора $w(t)$ удобно брать убывающую экспоненту с масштабом, большим обратных разрывов главных кластеров.

\end{document}
