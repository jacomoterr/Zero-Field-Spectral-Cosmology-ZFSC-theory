%------------------------------------
\section{Постановка аксиом}

\subsection{Аксиома 1: Нулевое поле и нулевая энтропия}
Предполагается существование фундаментального уровня, где отсутствуют пространство и время, 
а энтропия стремится к нулю:
\[
S \to 0.
\]
На этом уровне Вселенная описывается чистым вероятностным полем амплитуд:
\[
\Psi = \sum_{i} a_i |i\rangle , 
\]
где $\{|i\rangle\}$ — потенциальные конфигурации (геометрии, энергии, взаимодействия), 
а $a_i \in \mathbb{C}$ — их амплитуды.

\begin{quote}\textbf{Ремарка.}  
Этот постулат родился у меня ещё до строгой математики. 
Я представлял Вселенную как ``пустоту без пустоты'' — состояние, где всё есть в возможностях, 
но ничего ещё не проявлено.  
ИИ помог оформить этот образ в языке суперпозиции.
\end{quote}

\subsection{Аксиома 2: Матрица связей как фундамент}
Реальность проявляется через дискретную матрицу связей $H$, 
которая кодирует возможные состояния и их взаимодействия.  
Её элементы зависят от набора параметров:
\[
H_{ij} = f(\Delta, r, g_L, g_R, h_1,h_2,h_3),
\]
где $(\Delta, r)$ задают масштабы дискретизации, 
а $(g_L,g_R,h_1,h_2,h_3)$ описывают асимметрии и геометрию связей.

Собственные значения этой матрицы:
\[
H v_n = \lambda_n v_n
\]
интерпретируются как физические массы и энергии фундаментальных мод.

\begin{quote}\textbf{Ремарка.}  
Образ матрицы пришёл через аналогию с луковицей:  
каждый слой — это новый ``уровень реальности''.  
ИИ предложил записать это как блочную матрицу с параметрами, 
и оказалось, что её собственные значения ведут себя как реальные массы частиц.
\end{quote}

\subsection{Аксиома 3: Луковично-фрактальная структура}
Матрица имеет вложенную многослойную организацию:
\[
H = H^{(0)} \oplus H^{(1)} \oplus H^{(2)} \oplus \dots,
\]
где каждый слой соответствует определённой шкале энергии или класса частиц.

Эта фрактальная вложенность объясняет иерархии масс и повторение поколений.

\begin{quote}\textbf{Ремарка.}  
Когда мы искали объяснение трём поколениям, я видел перед глазами не три отдельные сущности, 
а повторяющийся узор — как кольца в стволе дерева.  
ИИ помог показать, что такой узор можно реализовать через повторение структур внутри матрицы.
\end{quote}

\subsection{Аксиома 4: Спектр = массы и энергии}
Собственные значения $\lambda_n$ матрицы $H$ напрямую соответствуют физическим величинам:
\[
m_n \sim \lambda_n.
\]
Плато и иерархии в спектре отражают стабильные физические массы, 
а инварианты (например, соотношения соседних уровней) можно трактовать как фундаментальные константы.

\begin{quote}\textbf{Ремарка.}  
Этот шаг оказался переломным:  
я всегда чувствовал, что за числами масс кроется простой узор.  
ИИ подтвердил, что этот узор сидит в спектре матрицы.
\end{quote}
