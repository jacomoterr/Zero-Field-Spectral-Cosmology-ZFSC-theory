\documentclass[12pt,a4paper]{article}
\usepackage[utf8]{inputenc}
\usepackage[T2A]{fontenc}
\usepackage[russian]{babel}
\usepackage{amsmath,amssymb}
\usepackage{hyperref}
\usepackage{geometry}
\usepackage{verbatim} % для блока цитирования
\geometry{margin=2.5cm}
\usepackage[T2A]{fontenc}
\usepackage[utf8]{inputenc}
\usepackage[russian]{babel}
\usepackage{braket}

\title{Zero-Field Spectral Cosmology (ZFSC): \\
Хроника исследовательского диалога человека и искусственного интеллекта}
\author{Евгений Монахов и ИИ-партнёр \\ VOSCOM Research Initiative}
\date{Сентябрь 2025}

\begin{document}
\maketitle

\begin{abstract}
Данный текст представляет собой хронику исследовательского диалога между человеком и искусственным интеллектом.
В ходе этого диалога была сформулирована и численно проверена нулевополевая спектральная космология (ZFSC).
Человек внёс аксиомы и образы, а ИИ помог перевести их в строгую математическую форму и провести численные проверки.
Текст фиксирует путь открытия и может рассматриваться как логическая база ZFSC.
\end{abstract}

%------------------------------------
\section{Введение}
ZFSC отвечает на фундаментальные вопросы физики: откуда берутся поколения фермионов, как устроено смешивание частиц, что такое тёмная энергия и какова судьба Вселенной.
Текст построен как двойной слой: строгие формулы и комментарии-ремарки, отражающие ход поиска.

%------------------------------------
\section{Постановка аксиом}
(Аксиома 1: нулевое поле и нулевая энтропия.  
Аксиома 2: матрица связей как фундамент.  
Аксиома 3: самоподобная фрактальная структура.  
Аксиома 4: спектр = массы и энергии.  
С ремарками.)

%------------------------------------
\section{Первые проверки: три поколения и коэффициенты $c$}
(Определение $c_f$, инвариантность к аффинным преобразованиям, численные совпадения модель/эксперимент, выводы.)

%------------------------------------
\section{CKM и PMNS как следствие фрактальных деформаций}
(Формализм $H_f = T_f(H)$, диагонализации $U_f$, CKM $\approx I$, PMNS с большими углами, примеры матриц.)

%------------------------------------
\section{Гравитон и тахион как нулевые и отрицательные моды}
(Нулевая мода как гравитон, отрицательные моды как тахионы, гипотеза подматриц частиц–античастиц.)

%------------------------------------
\section{Энергодоли Вселенной как спектральные доли}
(Интегралы по спектральной плотности: $\Omega_{\rm vis}, \Omega_{\rm dark}, \Omega_{\rm DE}$.  
Подматрицы частиц–античастиц как скрытые сектора.)

%------------------------------------
\section{Инфляция как раскалывание спектра}
(Экспоненциальное расхождение уровней, $N(t)\sim e^{\alpha t}$, $a(t)\sim e^{\beta \alpha t}$, число e-folds, каскады Фибоначчи, роль античастиц.)

%------------------------------------
\section{Хиггс как диагональный элемент бозонного блока}
(Диагональный элемент $\Delta_H$ в $H_{\rm boson}$ как источник $m_H^2$, фиксированная точка самоподобия, баланс частиц–античастиц.)

%------------------------------------
\section{Чёрные дыры и отсутствие сингулярности}
(Сингулярность заменяется спектральным ядром, горизонт событий как $\min(\lambda_+)$, решение информационного парадокса, роль подматриц.)

%------------------------------------
\section{Будущее Вселенной: насыщение спектра или фрактальные циклы}
(Сценарий 1: плато $N_{\max}$.  
Сценарий 2: фрактальные циклы.  
Отрицательные моды как триггеры, подматрицы античастиц как источник асимметрии, связь с $\Omega_{\rm DE}$.)

%------------------------------------
\section{Заключение}
ZFSC показывает, что структура масс, смешиваний, инфляции и космологических долей может быть объяснена через спектр самоподобной матрицы $H$.  
Эта теория возникла в диалоге между человеком и ИИ и фиксируется здесь как логическая база для дальнейшей формализации и исследований.

%------------------------------------
\begin{thebibliography}{99}

% --- Космология, инфляция, чёрные дыры ---
\bibitem{Guth1981}
A. H. Guth, ``Inflationary universe: A possible solution to the horizon and flatness problems,'' Phys. Rev. D \textbf{23}, 347 (1981).

\bibitem{Linde1983}
A. D. Linde, ``Chaotic inflation,'' Phys. Lett. B \textbf{129}, 177 (1983).

\bibitem{Planck2018}
Planck Collaboration, ``Planck 2018 results. VI. Cosmological parameters,'' Astron. Astrophys. \textbf{641}, A6 (2020).

\bibitem{PlanckInflation}
Planck Collaboration, ``Planck 2018 results. X. Constraints on inflation,'' Astron. Astrophys. \textbf{641}, A10 (2020).

\bibitem{Hawking1975}
S. W. Hawking, ``Particle creation by black holes,'' Commun. Math. Phys. \textbf{43}, 199–220 (1975).

\bibitem{Hawking2005}
S. W. Hawking, ``Information loss in black holes,'' Phys. Rev. D \textbf{72}, 084013 (2005).

\bibitem{Penrose2010}
R. Penrose, \emph{Cycles of Time: An Extraordinary New View of the Universe} (Bodley Head, 2010).

% --- Экспериментальные обзоры ---
\bibitem{PDG2024}
Particle Data Group, ``Review of Particle Physics,'' Prog. Theor. Exp. Phys. \textbf{2024}, 083C01 (2024).

\bibitem{NuFIT}
NuFIT Collaboration, ``Global fits to neutrino oscillation data,'' \url{https://www.nu-fit.org/}.

% --- Стандартная модель и Хиггс ---
\bibitem{HiggsDiscovery}
ATLAS and CMS Collaborations, ``Observation of a new particle in the search for the Standard Model Higgs boson,'' Phys. Lett. B \textbf{716}, 1–29 (2012).

\bibitem{WeinbergSM}
S. Weinberg, ``A model of leptons,'' Phys. Rev. Lett. \textbf{19}, 1264 (1967).

\bibitem{Glashow1961}
S. L. Glashow, ``Partial-symmetries of weak interactions,'' Nucl. Phys. \textbf{22}, 579–588 (1961).

\bibitem{Salam1968}
A. Salam, ``Weak and electromagnetic interactions,'' in \emph{Elementary Particle Theory} (Almqvist and Wiksell, 1968), pp. 367–377.

% --- Теория струн и квантовая гравитация ---
\bibitem{GreenSchwarzWitten}
M. B. Green, J. H. Schwarz, E. Witten, \emph{Superstring Theory}, Vol. 1–2 (Cambridge University Press, 1987).

\bibitem{Polchinski}
J. Polchinski, \emph{String Theory}, Vol. 1–2 (Cambridge University Press, 1998).

\bibitem{Zwiebach}
B. Zwiebach, \emph{A First Course in String Theory}, 2nd ed. (Cambridge University Press, 2009).

\bibitem{Rovelli}
C. Rovelli, \emph{Quantum Gravity} (Cambridge University Press, 2004).

% --- Фрактальные матрицы, спектры, квазипериодичность ---
\bibitem{AubryAndre}
S. Aubry and G. André, ``Analyticity breaking and Anderson localization in incommensurate lattices,'' Ann. Israel Phys. Soc. \textbf{3}, 133–164 (1980).

\bibitem{KohmotoFibonacci}
M. Kohmoto, L. P. Kadanoff, and C. Tang, ``Localization problem in one dimension: Mapping and escape,'' Phys. Rev. Lett. \textbf{50}, 1870 (1983).

\bibitem{MaciaFibonacci}
E. Macià, ``The role of aperiodic order in science and technology,'' Rep. Prog. Phys. \textbf{69}, 397 (2006).

\bibitem{SenechalQuasicrystals}
M. Senechal, \emph{Quasicrystals and Geometry} (Cambridge University Press, 1996).

% --- Графы и спектральная теория ---
\bibitem{ChungSpectral}
F. R. K. Chung, \emph{Spectral Graph Theory} (AMS, 1997).

\bibitem{Biggs}
N. Biggs, \emph{Algebraic Graph Theory}, 2nd ed. (Cambridge University Press, 1993).

\bibitem{Godsil}
C. Godsil, G. Royle, \emph{Algebraic Graph Theory} (Springer, 2001).

\bibitem{Mohar}
B. Mohar, ``The Laplacian spectrum of graphs,'' in \emph{Graph Theory, Combinatorics, and Applications}, Vol. 2, Wiley (1991).

% --- Математика и спектральные методы ---
\bibitem{Tao}
T. Tao, \emph{Topics in Random Matrix Theory} (AMS, 2012).

\bibitem{Mehta}
M. L. Mehta, \emph{Random Matrices}, 3rd ed. (Elsevier, 2004).

\bibitem{AndersonLocalization}
P. W. Anderson, ``Absence of diffusion in certain random lattices,'' Phys. Rev. \textbf{109}, 1492 (1958).

\end{thebibliography}

\end{document}
