\documentclass[12pt,a4paper]{article}
\usepackage[utf8]{inputenc}
\usepackage[T2A]{fontenc}
\usepackage[russian]{babel}
\usepackage{amsmath,amssymb}
\usepackage{hyperref}
\usepackage{geometry}
\usepackage{verbatim} % для блока цитирования
\geometry{margin=2.5cm}
\usepackage[T2A]{fontenc}
\usepackage[utf8]{inputenc}
\usepackage[russian]{babel}
\usepackage{braket}

\title{Сравнение: Стандартная модель, ZFSC и наблюдаемые следствия}
\author{Евгений Монахов \\ LCC ``VOSCOM ONLINE'' Research Initiative}
\date{Сентябрь 2025}

\begin{document}
\maketitle

\section*{Сравнительная таблица}

\begin{tabular}{p{4.5cm}p{5.5cm}p{5.5cm}}
\toprule
\textbf{Стандартная модель (SM)} & \textbf{Zero-Field Spectral Cosmology (ZFSC)} & \textbf{Наблюдаемые следствия} \\
\midrule
Пространство-время Минковского --- фундаментально & Пространство и время --- \emph{эффективные координаты} в матричной геометрии & Возможность фазовых переходов без геометрической «основы» (следы в космологии) \\
\midrule
Скорость света $c$ --- постулат теории относительности & $c$ --- \emph{спектральный предел} когерентного переноса фаз & Универсальное ограничение скорости для всех частиц; возможные микродеформации в ГВ/нейтринных задержках \\
\midrule
Массы через Хиггсовский механизм & Массы = \emph{собственные значения матрицы $H$} & Иерархии масс → \emph{устойчивые плато} в спектре; предсказуемые соотношения \\
\midrule
Частицы и античастицы связаны CPT-симметрией & Удвоенное пространство (частица/античастица) с возможным слабым C/CP-нарушением & Малые расщепления спектра; вклад в вакуумную плотность \\
\midrule
Тахионы запрещены как нестабильные & Тахионные моды $\lambda<0$ допустимы, группируются в \emph{тахионные поколения} & Сверхзапутанные моды; космологический фон; инфляционно-подобные фазы \\
\midrule
Аксионы --- гипотеза за пределами SM & Аксионы = \emph{фрактальные моды} «золотой лестницы» спектра & Массы $\mu\mathrm{eV}$--$\mathrm{meV}$; сигналы в haloscope, LSW, NMR-поисках \\
\midrule
CKM и PMNS матрицы вводятся феноменологически & CKM/PMNS = \emph{разные геометрии связей} одного $H$ & CKM почти единичная (малые углы), PMNS с большими углами → нейтринные осцилляции \\
\midrule
Тёмная материя/энергия как новые сущности & Тёмная материя/энергия = \emph{вклад высоких мод, тахионов и аксионных расщеплений} & Объяснение космологического фона без новых «внешних» полей \\
\bottomrule
\end{tabular}

\section*{Вывод}
Стандартная модель опирается на постулаты, ZFSC же \emph{выводит} те же ограничения и спектры как следствия единой фрактальной матрицы.  
Наблюдаемые эффекты --- устойчивые плато масс, CKM/PMNS структуры, возможные аксионные и тахионные вклады --- открывают путь к проверке модели в эксперименте.



\end{document}
