\documentclass[a4paper,12pt]{article}
\usepackage{geometry}
\geometry{margin=2.5cm}

% --- Стандартный стиль для наших статей ---
\usepackage{fontspec}
\usepackage{unicode-math}
\setmainfont{CMU Serif}          % текст
\setmathfont{Latin Modern Math}  % формулы

\usepackage{mathtools}
\usepackage{physics}
\everymath{\displaystyle}

\title{ZFSC и измерения постоянной тонкой структуры $\alpha$: \\
сопоставление экспериментов и интерпретация}
\author{Jacomo T \and Напарник-ИИ}
\date{}

\begin{document}
\maketitle

\section*{Задача}
Разные эксперименты дают слегка разные значения $\alpha$.
В рамках стандартной модели это объясняют «бегущестью» константы
из-за ренормгруппы. В ZFSC различия трактуются естественнее:
они отражают разные аспекты спектральной геометрии $H$ и слабую эволюцию $G_\text{eff}$.

---

\section{Общий принцип}
Каждое измерение $\alpha$ фактически соответствует
разному «окну наблюдения» матрицы $H$:
\begin{itemize}
  \item масштаб $\mu$ (энергия/длина волны),  
  \item топология слоя (особенно $U(1)$),  
  \item значение эффективной константы $G_\text{eff}(E)$.  
\end{itemize}
Поэтому значения слегка отличаются \emph{систематически}, а не случайно.

---

\section{Таблица соответствий}

\begin{center}
\begin{tabular}{|l|l|l|l|}
\hline
\textbf{Метод} & \textbf{Что меряют} & \textbf{ZFSC-интерпретация} & \textbf{Ожидание} \\
\hline
Атомная физика (рекойл, Ридберг) &
Энергии переходов при $E\to 0$ &
$\mu\approx 0$, $G_\text{eff}\approx G_0$ &
Базовое $\alpha(0)\approx 137.036$ \\
\hline
$g-2$ электрона &
Аномальный магнитный момент &
$\mu$ выше, чувствительность к новым модам &
Незначительная бегущесть \\
\hline
Квантовый эффект Холла &
Квантование проводимости &
Топология $U(1)$-слоя, связность &
Смещение из-за $\chi$ (геометрия) \\
\hline
Высокоэнергетические рассеяния ($M_Z$) &
Сечения процессов при $E\sim M_Z$ &
$\mu\sim M_Z$, $G_\text{eff}$ чуть изменён &
$\alpha(M_Z)\approx 127.95$ \\
\hline
\end{tabular}
\end{center}

---

\section{Интерпретация простыми словами}
Разные эксперименты смотрят на \emph{одну и ту же геометрию}, но под разными углами:
\begin{itemize}
  \item Атомная физика видит «низкоуровневый слой» — $\alpha(0)$.  
  \item $g-2$ фиксирует первые признаки бегущести.  
  \item QHE раскрывает топологическую структуру $U(1)$-слоя.  
  \item Коллайдеры проверяют $\alpha$ на высоких энергетических этажах.  
\end{itemize}

---

\section{Вывод}
ZFSC утверждает: \textbf{различия в измеренных значениях $\alpha$ не ошибка и не артефакт,
а закономерное следствие спектральной геометрии}.
Если удастся показать стабильную формулу для $\alpha$ и её естественную «бегущесть»,
то мы получим ключевое подтверждение теории.

\end{document}
