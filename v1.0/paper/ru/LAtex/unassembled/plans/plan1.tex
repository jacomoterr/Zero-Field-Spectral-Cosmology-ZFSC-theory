\documentclass[a4paper,12pt]{article}
\usepackage{geometry}
\geometry{margin=2cm}

% --- Шрифты CERN-style (стандарт LaTeX) ---
\usepackage{fontspec}
\usepackage{unicode-math}
\setmainfont{CMU Serif}        % полный набор: кириллица+латиница
\setmathfont{Latin Modern Math}  % формулы

\usepackage{mathtools}
\usepackage{physics}
\everymath{\displaystyle}

\title{ZFSC: План работ по подтверждени теории.\\ Часть 1. Вычислительные подтверждения \\ констант микромира.}
\author{Евгений Монахов}
\date{}


\begin{document}
\maketitle

\section*{Стратегическая цель}
Доказать Zero-Field Spectral Cosmology (ZFSC) через микромир:
\begin{itemize}
  \item массы поколений частиц,
  \item матрицы смешивания CKM/PMNS,
  \item силы взаимодействий,
  \item строгое геометрическое определение постоянной тонкой структуры $\alpha$.
\end{itemize}
Главный ориентир №1: вывести $\alpha$ как геометрический инвариант без подгонки.

---

\section*{Этап A. Базовая матрица $H$ и сектора}
\begin{enumerate}
  \item Построить $H$ (размер $N\times N$, параметры $\Delta,r,g_L,g_R,h_1,h_2,h_3$).
  \item Сектора ($u,d,\ell,\nu$) как геометрические трансформации $H$.
  \item Собственные значения $\lambda_n(H_s)\to$ массы: $m_n^{(s)}=c_s\cdot \lambda_n(H_s)$.
  \item Критерий: устойчивые топ-3 собственных значений для поколений.
\end{enumerate}

\textit{По-человечески:}  
Мы проверяем, что базовая матрица $H$ порождает три устойчивых значения,
которые можно трактовать как массы поколений частиц.

---

\section*{Этап B. Смешивание CKM и PMNS}
\[
  \mathrm{CKM} = U_u^\dagger U_d, 
  \qquad
  \mathrm{PMNS} = U_\ell^\dagger U_\nu .
\]

\textit{По-человечески:}  
Берём собственные векторы из разных секторов.  
Если они чуть расходятся, рождаются матрицы смешивания.  
Мы ждём малые углы для CKM и большие углы для PMNS.

---

\section*{Этап C. Силы взаимодействий}
\begin{itemize}
  \item Связность слоёв $\to$ группы $SU(3),SU(2),U(1)$.
  \item Эффективные константы: $g_i \propto f(\text{связность})$.
  \item Критерий: правильные порядки отношений $(g_3:g_2:g_1)$ на шкале $\mu_{\mathrm{geo}}$.
\end{itemize}

---

\section*{Этап D. Геометрическое $\alpha$}
Ищем инвариант $\mathcal{I}$ такой, что
\[
\alpha^{-1} = \mathcal{F}(\text{геометрия матрицы $H$ и слоёв}).
\]

Главная цель — получить $1/\alpha \approx 137$ без подгонки.

---

\section*{Этап E. Верификация}
\begin{itemize}
  \item Сравнить с $\alpha^{-1}(0)\approx 137.036$ и $\alpha^{-1}(M_Z)\approx 127.95$.
  \item Проверить устойчивость: $N$, шум $\pm 1\%$, разные граничные условия.
  \item Критерий: $\alpha^{-1}$ остаётся в $137.0 \pm 0.3$.
\end{itemize}

---

\section*{Этап F. Контрольные тесты}
\begin{itemize}
  \item Универсальность по секторам.
  \item Сходимость при росте $N$.
  \item Робастность к шумам.
  \item Независимость от нормировок.
\end{itemize}

---

\section*{Этап G. Мини-таймлайн}
\begin{enumerate}
  \item \textbf{Итерация 1 (3--5 дней):} спектры, черновые CKM/PMNS, первые $\alpha$.
  \item \textbf{Итерация 2 (5--7 дней):} устойчивость, выбор лучших кандидатов $\alpha$.
  \item \textbf{Итерация 3 (5--7 дней):} финальный отчёт.
\end{enumerate}

---

\section*{Выходные артефакты}
\begin{itemize}
  \item CSV со спектрами по секторам.
  \item CSV с углами CKM/PMNS.
  \item CSV с кандидатами $\alpha$.
  \item PDF-отчёт: формула, стабильность, шкала.
\end{itemize}

\end{document}
