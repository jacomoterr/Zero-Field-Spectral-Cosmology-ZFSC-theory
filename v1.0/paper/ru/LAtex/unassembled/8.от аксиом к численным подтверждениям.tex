\documentclass[12pt,a4paper]{article}
\usepackage[utf8]{inputenc}
\usepackage[T2A]{fontenc}
\usepackage[russian]{babel}
\usepackage{amsmath,amssymb}
\usepackage{hyperref}
\usepackage{geometry}
\usepackage{verbatim} % для блока цитирования
\geometry{margin=2.5cm}
\usepackage[T2A]{fontenc}
\usepackage[utf8]{inputenc}
\usepackage[russian]{babel}
\usepackage{braket}

\title{Zero-Field Spectral Cosmology (ZFSC): \\ от аксиом к численным подтверждениям}
\author{Евгений Монахов \\ VOSCOM Research Initiative}
\date{Сентябрь 2025}

\begin{document}
\maketitle

\begin{abstract}
Предлагается структура исследований нулевополевой спектральной космологии (ZFSC) --- от аксиоматического уровня до численных проверок. 
Определены 7 ключевых прорывных направлений, каждое из которых может быть подтверждено численным моделированием и связано с экспериментальными данными.
\end{abstract}

\section{Аксиомы}
\begin{enumerate}
  \item \textbf{Нулевой уровень энтропии:} фундаментальное состояние без времени и пространства, $\Psi = \sum_i a_i |i\rangle$.
  \item \textbf{Матричная структура:} реальность задаётся дискретной симметричной матрицей связей.
  \item \textbf{Собственные значения:} массы и энергии частиц соответствуют собственным значениям матрицы.
  \item \textbf{Самоподобие:} матрица строится фрактально (луковичные/φ-структуры).
  \item \textbf{Инварианты:} фундаментальные константы соответствуют спектральным инвариантам матрицы.
\end{enumerate}

\section{Цель}
Перейти от аксиом к численным проверкам, демонстрирующим совпадения с наблюдаемыми свойствами элементарных частиц и космологии.

\section{Прорывные направления исследований}

\subsection{Поколения фермионов из геометрии}
Показать, что три наинизших положительных собственных значения в секторах $\nu,\ell,u,d$ соответствуют трём поколениям частиц. 
Вычислить коэффициенты
\[
c_{\rm sec} = \frac{m_3^2 - m_2^2}{m_2^2 - m_1^2},
\]
и сравнить с экспериментальными $c_\nu, c_\ell, c_u, c_d$.

\subsection{CKM и PMNS из геометрии}
Показать, что различие геометрических трансформаций секторов даёт
\[
\text{CKM} = U_u^\dagger U_d \approx \mathbb{1}, \quad
\text{PMNS} = U_\ell^\dagger U_\nu \ \text{с большими углами}.
\]

\subsection{Нулевой и отрицательный моды}
Идентифицировать:
\[
\lambda_0 \approx 0 \quad \Rightarrow \quad \text{гравитон}, \qquad
\lambda < 0 \quad \Rightarrow \quad \text{тахион/нестабильность}.
\]

\subsection{Энергетические доли Вселенной}
Сравнить спектральные энергодоли
\[
\Omega_{\rm sec} = \frac{E_{\rm sec}}{\sum_i \lambda_i^2}
\]
с космологическими измерениями $\Omega_{\rm baryon},\Omega_{\rm dark},\Omega_\Lambda$.

\subsection{Инфляция как раскалывание спектра}
Показать, что при переходе от нулевой матрицы к первой фрактальной структуре:
\[
N(t) \sim \varphi^{t/\tau}, \quad
a(t) \propto N(t)^\alpha,
\]
и это даёт $N_e \approx 60$ e-folds без инфлатонного поля.

\subsection{Хиггс как центральная диагональ}
Выделить центральный диагональный элемент бозонного блока
\[
\Delta_H = \frac{\delta_{\rm bos}}{r},
\]
и показать, что он соответствует массе $\sim 125$ ГэВ.

\subsection{Скорость света как спектральный инвариант}
Показать, что
\[
c \approx \sqrt{\kappa_{U(1)}}, \qquad 
\kappa_{U(1)} = \frac{2}{N_{U(1)}} \sum_{e\in U(1)} w_e,
\]
и что $c$ стабилен при разных масштабах матрицы.

\section{Вывод}
Каждый из этих пунктов может быть проверен численно. 
Совпадение хотя бы части из них с экспериментом станет серьёзным подтверждением модели ZFSC. 
Полное совпадение приведёт к фундаментальному сдвигу в физике: массы, поколения, константы и космология будут объяснены как спектральные свойства дискретной матрицы связей.

\end{document}
