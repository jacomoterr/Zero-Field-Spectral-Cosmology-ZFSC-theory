\documentclass[12pt,a4paper]{article}
\usepackage[utf8]{inputenc}
\usepackage[T2A]{fontenc}
\usepackage[russian]{babel}
\usepackage{amsmath,amssymb}
\usepackage{hyperref}
\usepackage{geometry}
\usepackage{verbatim} % для блока цитирования
\geometry{margin=2.5cm}
\usepackage[T2A]{fontenc}
\usepackage[utf8]{inputenc}
\usepackage[russian]{babel}
\usepackage{braket}

\title{Чёрные дыры и судьба Вселенной в нулевополевой спектральной космологии (ZFSC)}
\author{Евгений Монахов \\ VOSCOM Research Initiative}
\date{Сентябрь 2025}

\begin{document}
\maketitle

\begin{abstract}
В работе анализируется поведение чёрных дыр и глобальная эволюция Вселенной в рамках нулевополевой спектральной космологии (ZFSC). 
Показано, что чёрные дыры в данной модели представляют собой локальные вложенные подматрицы, что устраняет проблему сингулярности и сохраняет информацию. 
Будущее Вселенной предсказывается как либо асимптотическое насыщение спектра, либо циклическое фрактальное расширение.
\end{abstract}

\section{Введение}
Классическая общая теория относительности предсказывает образование сингулярностей в центре чёрных дыр и в начале космологической эволюции. 
Наша теория (ZFSC) основана на том, что фундаментальная реальность задаётся матрицей связей, чьи собственные значения соответствуют массам частиц и энергиям взаимодействий:
\[
E = \sum_i \lambda_i^2, \qquad \Psi = \sum_i a_i |i\rangle.
\]
Это позволяет интерпретировать чёрные дыры и судьбу Вселенной в спектральных терминах.

\section{Чёрные дыры как вложенные подматрицы}
Пусть глобальная матрица $M$ имеет спектр $\{\lambda_i\}$. 
Формирование чёрной дыры соответствует выделению блока $M_{\text{BH}}$, встроенного в $M$, такого что
\[
M = \begin{pmatrix}
M_{\text{BH}} & 0 \\ 0 & M_{\text{ext}}
\end{pmatrix}.
\]

\subsection{Отрицательные моды и гравитация}
Гравитация в ZFSC соответствует низшим собственным значениям. 
При коллапсе масса локализуется и вызывает появление новых отрицательных мод:
\[
\lambda_{\text{BH}} < 0.
\]

Однако, так как размер блока конечен, то
\[
\min(\lambda_{\text{BH}}) > -\infty,
\]
и сингулярность не возникает.

\subsection{Информация}
Горизонт событий соответствует границе, за которой собственные векторы ортогональны внешнему пространству состояний:
\[
\langle v_{\text{BH}} | v_{\text{ext}} \rangle = 0.
\]
Информация не исчезает, а сохраняется внутри $M_{\text{BH}}$ и может частично утекать в виде коррелированных мод (аналог излучения Хокинга).

\section{Отсутствие сингулярности}
В отличие от классической ОТО, где плотность в центре дыры обращается в бесконечность, в ZFSC спектр всегда дискретен и ограничен:
\[
\sum_i \lambda_i^2 < \infty.
\]
Сингулярность заменяется фрактальным вложением подматриц, где энергия перераспределяется по глубоким модам, но не уходит в бесконечность.

\section{Эволюция Вселенной}
Пусть число доступных мод $N(t)$ растёт по закону:
\[
N(t) \sim \varphi^{t/\tau}, \qquad \varphi = \frac{1+\sqrt{5}}{2}.
\]
Тогда масштабный фактор:
\[
a(t) \propto N(t)^{\alpha}, \qquad H(t) = \frac{d}{dt}\ln a = \alpha \frac{\ln \varphi}{\tau}.
\]

\subsection{Окончание инфляции}
Когда фрактальное самоподобие исчерпывается или переходит в степенной рост $N(t) \sim t^p$, имеем
\[
H(t) \sim \frac{\alpha p}{t},
\]
и Вселенная выходит из инфляционной стадии в обычный FRW-режим.

\subsection{Современная стадия}
Текущее ускорение описывается как слабый остаточный механизм:
\[
H_{\text{now}} \approx \alpha_{\text{res}} \Gamma \ln \varphi,
\]
где $\alpha_{\text{res}} \ll 1$, $\Gamma \ll 1/\tau$. 
Это соответствует «эхо-следу» первичного самоподобия.

\section{Прогнозы}
ZFSC предсказывает два сценария будущего:
\begin{enumerate}
  \item \textbf{Насыщение:} спектр заполняется полностью, новые моды не рождаются, и Вселенная выходит на стадию квазистатического расширения.
  \item \textbf{Фрактальные циклы:} каждый слой матрицы раскрывает новые подуровни, что приводит к сериям «мини-инфляций» на всё больших масштабах.
\end{enumerate}

В обоих сценариях отсутствуют сингулярности и тепловая смерть: энергия всегда конечна и перераспределяется в спектре.

\section{Выводы}
\begin{enumerate}
  \item Чёрные дыры в ZFSC описываются как вложенные подматрицы с конечным спектром, что устраняет проблему сингулярности и обеспечивает сохранение информации.
  \item Инфляция и её окончание объясняются динамикой фрактального роста числа мод.
  \item Современное ускорение связано с остаточной самоподобностью спектра.
  \item Будущее Вселенной предсказывается как либо насыщение спектра, либо циклическое фрактальное расширение.
\end{enumerate}

\end{document}
