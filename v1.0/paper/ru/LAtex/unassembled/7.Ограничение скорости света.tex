%------------------------------------
\section{Ограничение скорости света}

В рамках спектральной космологии нулевого поля (ZFSC) ограничение скорости света $c$ выводится не из геометрии пространства-времени Минковского, а как фундаментальное свойство спектра базовой матрицы $H$.

%-----------------------------
\subsection*{Аксиома 1. Базовая матрица и когерентность}
Фундамент Вселенной описывается эрмитовой матрицей $H$ размерности $N \times N$:
\[
H \, | \psi_i \rangle = \lambda_i \, | \psi_i \rangle ,
\]
где $\lambda_i \in \mathbb{R}$ — собственные значения (спектр энергий), а $|\psi_i\rangle$ — собственные векторы (состояния частиц).

%-----------------------------
\subsection*{Аксиома 2. Когерентный перенос}
Эволюция квантового состояния задаётся фазой амплитуд:
\[
|\Psi(t)\rangle = \sum_i a_i e^{-i\lambda_i t/\hbar} |\psi_i\rangle .
\]
Скорость переноса когерентности определяется градиентом фазы:
\[
v \;=\; \left| \frac{d\varphi}{dt} \cdot \frac{dx}{d\varphi} \right|,
\]
где $\varphi$ — фаза собственных векторов, $x$ — эффективная пространственная координата.

%-----------------------------
\subsection*{Аксиома 3. Спектральный предел скорости}
Существует глобальный предел фазового градиента:
\[
v_{\max} \equiv c \;=\; \max \left| \frac{d\varphi}{dt} \cdot \frac{dx}{d\varphi} \right| ,
\]
который универсален для всех секторов (u, d, $\ell$, $\nu$), так как все они происходят из одной базовой матрицы $H$.

%-----------------------------
\subsection*{Следствие 1. Масса как спектральный сдвиг}
Масса частицы определяется сдвигом собственного значения $\lambda_i$:
\[
m_i \;=\; \frac{\lambda_i}{c^2}.
\]
\begin{itemize}
  \item Для безмассовых частиц ($\lambda_i \approx 0$): $v = c$.
  \item Для массивных частиц ($\lambda_i > 0$): $v < c$.
\end{itemize}

%-----------------------------
\subsection*{Следствие 2. Тахион как разрушение когерентности}
Если $\lambda_i < 0$, состояние соответствует тахиону. Формально оно обладает $v > c$, но разрушает когерентность матрицы и потому нестабильно:
\[
\lambda_i < 0 \;\;\Rightarrow\;\; \text{нестабильность спектра}.
\]

%-----------------------------
\subsection*{Следствие 3. Универсальность $c$}
Так как $c$ задаётся не свойством конкретной частицы, а глобальной когерентностью матрицы $H$, этот предел един для всех безмассовых возбуждений:
\[
\text{фотоны}, \quad \text{гравитоны (нулевая мода)}, \quad \text{другие безмассовые поля}.
\]

%-----------------------------
\subsection*{Итог}
Таким образом, в ZFSC ограничение скорости света $c$ представляет собой \textit{верхний предел скорости когерентного переноса фазовых амплитуд} в матрице $H$. Масса возникает как сдвиг собственных значений, а тахионы — как нестабильные отрицательные моды.
