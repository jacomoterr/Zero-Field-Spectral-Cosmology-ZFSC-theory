\documentclass[12pt,a4paper]{article}
\usepackage[utf8]{inputenc}
\usepackage[T2A]{fontenc}
\usepackage[russian]{babel}
\usepackage{amsmath,amssymb}
\usepackage{hyperref}
\usepackage{geometry}
\usepackage{verbatim} % для блока цитирования
\geometry{margin=2.5cm}
\usepackage[T2A]{fontenc}
\usepackage[utf8]{inputenc}
\usepackage[russian]{babel}
\usepackage{braket}

\title{Эволюция Вселенной в спектральной космологии нулевого поля (ZFSC): \\
от ранних чёрных дыр к галактикам}
\author{Евгений Монахов \\ ООО ``VOSCOM ONLINE'' Research Initiative \\ 
\href{https://orcid.org/0009-0003-1773-5476}{ORCID: 0009-0003-1773-5476}}
\date{Сентябрь 2025}

\begin{document}
\maketitle

\begin{abstract}
В рамках спектральной космологии нулевого поля (ZFSC) описана последовательность фаз развития Вселенной, объясняющая аномально зрелые галактики и сверхмассивные чёрные дыры (SMBH) на больших красных смещениях. Показано, что растянутое время и временная эволюция эффективной гравитационной постоянной $G_{\rm eff}(t)$ естественным образом приводят к раннему образованию SMBH, их последующему разнесению и слияниям, а также к фазе захвата барионного вещества и формированию галактик. 
\end{abstract}

\section{Введение}
Наблюдения телескопа Джеймс Вебб (JWST) показывают существование массивных и упорядоченных галактик на красных смещениях $z\sim 10-15$, что ставит под сомнение достаточность времени для их формирования в стандартной модели $\Lambda$CDM.

Теория ZFSC предлагает альтернативный сценарий:
\begin{enumerate}
  \item время появляется сразу (нулевая мода),
  \item пространство разворачивается слоями матрицы связности,
  \item силы взаимодействий эволюционируют во времени через $G_{\rm eff}(t)$,
  \item ``растянутое'' время обеспечивает больший возраст Вселенной при фиксированном $z$.
\end{enumerate}

\section{Фаза I: Сверхранние коллапсы (тысячи лет)}
В первые $\sim 10^3$ лет после развёртывания:
\begin{itemize}
  \item усиленная гравитация ($G_{\rm eff}/G_0 \approx 1.05-1.10$) за счёт плотной связности матрицы;
  \item крайне короткие времена коллапса неоднородностей;
  \item формирование множества проточёрных дыр и их быстрый рост в SMBH ($10^6-10^9 M_\odot$).
\end{itemize}

\section{Фаза II: Растяжение матрицы (до сотен млн лет)}
По мере снижения связности:
\begin{itemize}
  \item пространство ``растягивается'', разносит уже образованные SMBH;
  \item происходят многочисленные слияния SMBH;
  \item закладывается ``скелет'' крупномасштабной структуры.
\end{itemize}

\section{Фаза III: Захват вещества и активные ядра}
После охлаждения барионного газа:
\begin{itemize}
  \item SMBH становятся центрами притяжения вещества;
  \item возникают сверхяркие активные ядра галактик (AGN, квазары);
  \item наблюдается быстрая звездообразовательная активность.
\end{itemize}

\section{Фаза IV: Переход к обычным галактикам}
Когда $G_{\rm eff}$ снижается до $\lesssim 1\%$ усиления (возраст $\sim 6-7$ Гир, $z\sim 1$):
\begin{itemize}
  \item рост SMBH замедляется;
  \item активность ядер убывает;
  \item галактики приобретают привычные морфологии (спиральные, эллиптические).
\end{itemize}

\section{Математическая основа}
\subsection{Эффективная гравитационная постоянная}
\[
G_{\rm eff}(t)=G_0\left[1+\varepsilon_G e^{-t/\tau}\right], \quad 
\varepsilon_G \sim 0.05-0.10,\ \tau \sim 1.5\,\text{Гир}.
\]

\subsection{Растянутое время}
\[
t(z)=t_{\Lambda \text{CDM}}(z)+\Delta t_0 \left(1-\frac{1}{(1+z)^\alpha}\right),
\]
где $\Delta t_0 \sim 1-3$ Гир, $\alpha \sim 1-2$.

\subsection{Рост SMBH}
Время эддингтоновского роста:
\[
t_S \propto \frac{1}{G_{\rm eff}}.
\]
При $\Delta G/G \sim 0.05-0.10$ рост в ранние эпохи ускоряется, что позволяет достичь $10^9 M_\odot$ к $z\sim 10$.

\section{Заключение}
В ZFSC последовательность фаз такова:
\begin{enumerate}
  \item \textbf{Ранний урожай SMBH} (первые $\sim 10^3$ лет),
  \item \textbf{Растяжение и слияния} (до сотен млн лет),
  \item \textbf{Захват вещества и яркие квазары} ($z\sim 6-15$),
  \item \textbf{Переход к спокойным галактикам} ($z\lesssim 1$).
\end{enumerate}

Это объясняет зрелые галактики JWST и быстрый рост SMBH без необходимости постулировать экзотические механизмы аккреции или отдельную ``тёмную энергию''.
\end{document}
