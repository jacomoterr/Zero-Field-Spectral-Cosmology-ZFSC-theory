\documentclass[12pt,a4paper]{article}
\usepackage[utf8]{inputenc}
\usepackage[russian]{babel}
\usepackage{amsmath,amssymb,amsfonts}
\usepackage{bm}
\usepackage{physics}
\usepackage{siunitx}
\usepackage{booktabs}
\usepackage{hyperref}
\usepackage{geometry}
\usepackage{tikz}
\geometry{margin=2.3cm}
\hypersetup{colorlinks=true,linkcolor=blue,citecolor=blue,urlcolor=blue}
\usepackage[utf8]{inputenc}
\usepackage[russian]{babel}
\usetikzlibrary{arrows.meta}
\usetikzlibrary{positioning}
% В преамбулу:
\usetikzlibrary{arrows.meta, positioning}

\title{Спектральное происхождение массы: \\ 
от Zero-Field Spectral Cosmology к формуле Эйнштейна}
\author{Евгений Монахов \\ VOSCOM ONLINE Research Initiative}
\date{\today}

\begin{document}
\maketitle

\section*{Введение}
Формула Альберта Эйнштейна 
\[
E = mc^2
\]
стала символом физики XX века, выразив глубокую связь массы и энергии. 
Однако в ней масса $m$ принимается как данное свойство. 
В рамках Zero-Field Spectral Cosmology (ZFSC) мы выдвигаем 
более фундаментальный принцип: масса возникает из спектра фундаментальной матрицы. 
Эта идея записывается краткой формулой:
\[
m = \lambda(H),
\]
где $\lambda(H)$ — собственные значения матрицы $H$, определяющей структуру связей в «нулевом поле».

\section*{Постулаты ZFSC}
\begin{enumerate}
  \item \textbf{Нулевой уровень энтропии:}
  \[
  S \to 0, \qquad 
  \Psi = \sum_i a_i |i\rangle .
  \]
  Вселенная на фундаментальном уровне описывается вероятностным полем амплитуд.
  
  \item \textbf{Массы как спектр:}
  \[
  m = \lambda(H).
  \]

  \item \textbf{Поколения частиц:}
  \[
  m^{(n)}_f = \lambda_n(H), \quad n=1,2,3.
  \]

  \item \textbf{Смешивания:}
  \[
  \mathrm{Mix} = U_A^\dagger U_B.
  \]
\end{enumerate}

\section*{Главный закон: $m = \lambda(H)$}
Эта формула утверждает: масса частицы — это не фундаментальная данность, 
а результат спектральных свойств матрицы $H$. 
Так же как звуки возникают из колокола, а ноты — из пианино, 
массы возникают из внутренней структуры фундаментальной матрицы.

\subsection*{Примеры-аналогии}
\begin{itemize}
  \item  \textbf{Пианино:} клавиши и струны = $H$, ноты = $\lambda(H)$, массы = звуки.
  \item  \textbf{Колокол:} форма = $H$, собственные частоты звона = $\lambda(H)$, массы = тоны.
  \item  \textbf{Призма:} структура = $H$, спектр = $\lambda(H)$, массы = цвета.
\end{itemize}

\section*{Связь с Эйнштейном}
Формула Эйнштейна:
\[
E = mc^2
\]
становится частным случаем, если подставить $m = \lambda(H)$:
\[
E = \lambda(H) \, c^2.
\]
Таким образом, энергия частицы определяется не абстрактной массой, 
а конкретным спектральным числом фундаментальной матрицы.

\section*{Лесенка выводов}
\begin{enumerate}
  \item Фундамент: $m = \lambda(H)$.
  \item Подстановка: $E = \lambda(H)c^2$.
  \item Связь с квантовой механикой: $E = \hbar \omega$.
  \item Геометрия и симметрии: структура $H$ порождает группы $SU(3)\times SU(2)\times U(1)$.
  \item Классическая физика: предел больших масштабов.
\end{enumerate}

\section*{Гипотезы и расширения}
\begin{itemize}
  \item Нулевая мода: $\lambda_0 \approx 0 \Rightarrow$ гравитон.
  \item Отрицательные моды: $\lambda<0 \Rightarrow$ тахионы.
  \item Энергетические поправки через квантовую запутанность.
  \item Временная эволюция фундаментальных констант: $G_\mathrm{eff}(t)$, $\alpha(t)$.
\end{itemize}

\section*{Заключение}
Формула $m=\lambda(H)$ выступает более фундаментальным законом, 
чем $E=mc^2$, поскольку она объясняет происхождение самой массы. 
Эйнштейн связал массу и энергию; ZFSC показывает, откуда берётся масса.

\section*{План исследований}
\begin{enumerate}
  \item Численные спектральные сканы больших матриц $H$.
  \item Сравнение с экспериментальными массами поколений.
  \item Анализ тахионных мод и их возможных эффектов.
  \item Космологические применения: эволюция $G_\mathrm{eff}(t)$.
\end{enumerate}

\end{document}
