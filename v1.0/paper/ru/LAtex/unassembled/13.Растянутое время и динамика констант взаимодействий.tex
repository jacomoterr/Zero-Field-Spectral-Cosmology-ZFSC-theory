\documentclass[12pt,a4paper]{article}
\usepackage[utf8]{inputenc}
\usepackage[T2A]{fontenc}
\usepackage[russian]{babel}
\usepackage{amsmath,amssymb}
\usepackage{hyperref}
\usepackage{geometry}
\usepackage{verbatim} % для блока цитирования
\geometry{margin=2.5cm}
\usepackage[T2A]{fontenc}
\usepackage[utf8]{inputenc}
\usepackage[russian]{babel}
\usepackage{braket}

\title{Растянутое время и динамика констант взаимодействий \\
в спектральной космологии нулевого поля (ZFSC)}
\author{Евгений Монахов \\ ООО ``VOSCOM ONLINE'' Research Initiative \\ 
\href{https://orcid.org/0009-0003-1773-5476}{ORCID: 0009-0003-1773-5476}}
\date{Сентябрь 2025}

\begin{document}
\maketitle

\begin{abstract}
В рамках теории спектральной космологии нулевого поля (ZFSC) показано, что стандартное соответствие между красным смещением $z$ и возрастом Вселенной $t$ должно быть модифицировано. Вводится поправка, отражающая фазовое разворачивание пространственной связности, что приводит к \emph{растянутому возрасту} Вселенной по сравнению с моделью $\Lambda$CDM. Дополнительно рассматривается слабая временная эволюция эффективных констант взаимодействий. Обсуждается влияние на быстрое формирование структур и ранние сверхмассивные чёрные дыры (SMBHs).
\end{abstract}

\section{Введение}
Современная космология (модель $\Lambda$CDM) оценивает возраст Вселенной как $13.8$~Гир, исходя из температуры реликтового излучения ($T_{\text{CMB}}=2.725$~K), параметров Хаббла и состава материи. Однако наблюдения космического телескопа Джеймс Вебб (JWST) обнаружили массивные и упорядоченные галактики уже на $z\sim 10-15$, что ставит под сомнение достаточность времени для их формирования.

В ZFSC постулируется: 
\begin{enumerate}
  \item время разворачивается с нулевой модой (гравитон),
  \item пространство формируется поэтапно, через матричные слои связности,
  \item силы взаимодействий выводятся из спектра матрицы и могут эволюционировать во времени.
\end{enumerate}

Это приводит к модифицированному соответствию $t(z)$ и возможности объяснить аномально быстрый рост структур.

\section{Модификация связи $t(z)$}
В стандартной космологии:
\[
t(z) = \int_z^\infty \frac{dz'}{(1+z')H(z')},
\]
где $H(z)$ — параметр Хаббла. 

В ZFSC предлагается поправка:
\[
t(z) = t_{\Lambda\text{CDM}}(z) + \gamma \ln(1+z),
\]
где $\gamma \approx 1.0-2.0$~Гир. 

При $z=10$:
\[
t_{\Lambda\text{CDM}} \approx 0.47~\text{Гир}, \quad 
t_{\text{ZFSC}}(\gamma=1.5) \approx 4.1~\text{Гир}.
\]

Аналогично, при $z=15$:
\[
t_{\Lambda\text{CDM}} \approx 0.27~\text{Гир}, \quad 
t_{\text{ZFSC}}(\gamma=1.5) \approx 4.4~\text{Гир}.
\]

Таким образом, Вселенная оказывается на несколько миллиардов лет старше в терминах доступного времени формирования структур.

\section{Эволюция констант взаимодействий}
Так как все взаимодействия в ZFSC выводятся из спектра матрицы связности, слабая временная эволюция естественна:
\[
G_{\text{eff}}(t) = G_0 \left[1 + \varepsilon_G f(t)\right], \quad 
g_i(t) = g_{i,0} \left[1 + \varepsilon_i f(t)\right],
\]
где $0<\varepsilon\lesssim 0.1$, а $f(t)$ — убывающая функция (например, $f(t)=e^{-t/\tau}$). 

В ранней Вселенной ($t \lesssim 10^6$ лет) это означает более сильное гравитационное связывание и ускоренный рост неоднородностей.

\section{Формирование сверхмассивных чёрных дыр}
В стандартной картине SMBH с массами $\sim 10^9 M_\odot$ на $z\sim 10$ трудно объяснить без экстремальной аккреции. 

В ZFSC:
\begin{itemize}
  \item добавка ко времени $t(z)$ даёт в распоряжении $\sim 3-4$~Гир вместо $\sim 0.3-0.5$~Гир,
  \item усиленное $G_{\text{eff}}$ сокращает времена коллапса,
  \item затравки ($10^3-10^6 M_\odot$) естественно вырастают до наблюдаемых SMBH.
\end{itemize}

\section{Наблюдательные следы}
\begin{enumerate}
  \item \textbf{CMB:} фазовые сдвиги акустических пиков.
  \item \textbf{BBN:} возможные коррекции соотношений D/H и $^7$Li.
  \item \textbf{JWST:} наличие зрелых галактик и SMBH на $z\sim 10-15$.
\end{enumerate}

\section{Заключение}
ZFSC даёт естественное объяснение ``слишком взрослым'' структурам на больших красных смещениях. Модифицированная зависимость $t(z)$ и временная эволюция констант взаимодействий ведут к более старой и динамичной Вселенной, что открывает новые горизонты для интерпретации наблюдений.

\end{document}
