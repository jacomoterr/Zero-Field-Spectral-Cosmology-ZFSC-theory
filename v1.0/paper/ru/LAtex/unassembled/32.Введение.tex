\documentclass[12pt,a4paper]{article}
\usepackage[utf8]{inputenc}
\usepackage[russian]{babel}
\usepackage[T1]{fontenc}
\usepackage{lmodern}
\usepackage{amsmath,amssymb,amsthm,mathtools}
\usepackage{physics}
\usepackage{bm}
\usepackage{siunitx}
\usepackage{hyperref}
\usepackage{geometry}
\usepackage{booktabs}
\usepackage{enumitem}
\geometry{margin=2.5cm}
\hypersetup{colorlinks=true,linkcolor=blue,citecolor=teal,urlcolor=magenta}

\title{Спектральная теория ZFSC: \\ 
унификация масс нейтрино, лептонов и кварков}
\author{Евгений Монахов \\ VOSCOM Research Initiative}
\date{Сентябрь 2025}

\begin{document}
\maketitle

\begin{abstract}
Предложена модель \emph{Zero-Field Structural Coefficients (ZFSC)}, 
в которой массы фермионов (нейтрино, лептонов, кварков) 
связаны через спектральные коэффициенты матриц, зависящих от параметров 
$\delta$, $r$, $g_L$, $g_R$, $h_1$, $h_2$, $h_3$ и секторных масштабов.  
Результаты численных расчётов показывают, что модель воспроизводит 
экспериментальные данные для нейтрино и лептонов с точностью $<0.01\sigma$, 
что можно трактовать как потенциальный прорыв в теории элементарных частиц.  
Обсуждаются перспективы расширения теории, в том числе возможная связь 
с топологией свёрнутых измерений.  
\end{abstract}

\section{Введение}
Современная физика элементарных частиц описывает взаимодействия через 
Стандартную модель, но происхождение масс фермионов остаётся открытой проблемой.  
Массы нейтрино и соотношения между поколениями лептонов и кварков 
до сих пор не имеют строгого теоретического объяснения.  

Предлагаемая теория ZFSC основывается на идее, что существует 
универсальная матричная структура, формирующая спектральные коэффициенты $c$, 
которые связаны с наблюдаемыми массовыми соотношениями.  

\section{Математическая модель}
Базовый объект модели — матрица $B$ размерности $N\times N$, где $N=3,4,6$:
\begin{equation}
B_{ij} =
\begin{cases}
\delta + h_1 i + h_2 j + h_3 (i-j)^2, & i=j, \\
r \cdot (g_L \ \text{если } i<j \ \text{иначе } g_R), & i \neq j.
\end{cases}
\end{equation}

Её собственные значения $\lambda_i$ упорядочиваются по величине.  
Для $N \geq 3$ определяется спектральный коэффициент:
\begin{equation}
c = \frac{\lambda_{\max} - \lambda_{\min}}{\lambda_{\text{mid}} - \lambda_{\min}}.
\end{equation}

Этот коэффициент интерпретируется как \emph{структурное отношение} 
массового спектра для соответствующего сектора частиц.

\subsection{Секторные масштабы}
Для согласования нейтрино, лептонов и кварков вводятся масштабные множители:
\begin{equation}
c_{\text{eff}}^{(s)} = c \cdot S_s, \quad 
S_s \in \{\text{нейтрино}, \ \text{лептоны}, \ \text{up-кварки}, \ \text{down-кварки}\}.
\end{equation}

Также возможны дополнительные масштабы $\alpha_s$ для $\delta$ и $r$, 
задающие различную ``жёсткость'' спектра в каждом секторе.

\section{Экспериментальные данные}
Для проверки теории использованы известные массы:
\begin{itemize}
  \item Нейтрино: $\Delta m^2_{21}, \Delta m^2_{31}$ (данные глобальных фитингов).
  \item Заряженные лептоны: $m_e, m_\mu, m_\tau$.
  \item Кварки up-типа: $m_u, m_c, m_t$.
  \item Кварки down-типа: $m_d, m_s, m_b$.
\end{itemize}

Из этих масс формируются экспериментальные коэффициенты $c_\nu$, $c_\ell$, $c_u$, $c_d$.

\section{Результаты}
\subsection{Независимые подгоны}
В режиме \texttt{independent\_all} (каждый сектор имеет свои $\delta,r$) получены значения:
\begin{align*}
c_\nu^{\text{model}} &\approx c_\nu^{\text{exp}} \quad (<0.01\sigma), \\
c_\ell^{\text{model}} &\approx c_\ell^{\text{exp}} \quad (<0.01\sigma).
\end{align*}
Совпадение лучше экспериментальных погрешностей.

Для кварков остаются отклонения $\sim 20\sigma - 90\sigma$, 
однако переход к матрицам $6\times6$ существенно снижает ошибки.

\subsection{Попытки унификации}
В строгом режиме (\texttt{grand\_unify\_all}) несовпадения огромные 
($70\sigma - 90\sigma$), что указывает на невозможность полной унификации.  

Однако введение масштабов (\texttt{grand\_unify\_all\_scaled}) позволило 
снизить глобальную ошибку и добиться частичного согласия всех четырёх секторов.

\section{Физическая интерпретация параметров}
\begin{itemize}
  \item $\delta$ — структурный ``масштаб'' спектра, возможно связанный с геометрией поля.  
  \item $r$ — коэффициент смешивания поколений.  
  \item $g_L, g_R$ — асимметрия между левыми и правыми компонентами взаимодействий.  
  \item $h_1, h_2, h_3$ — топологические поправки, связанные с формой свёрнутых измерений.  
  \item $S_s$ (sector scales) — эффективные константы связи для каждого семейства частиц.  
\end{itemize}

\section{План исследований}
\begin{enumerate}
  \item Расширить матрицы до $N=8,12$, проверить устойчивость спектра.  
  \item Проверить, может ли теория предсказать абсолютные массы нейтрино.  
  \item Связать параметры $h_i$ с топологией многообразий Калаби--Яу.  
  \item Исследовать возможность описания констант связи (например, $\alpha_s$) в рамках этой модели.  
  \item Разработать численный скан параметров с GPU-ускорением.  
\end{enumerate}

\section{Заключение}
Теория ZFSC показала выдающееся совпадение с экспериментальными данными 
для нейтрино и лептонов ($<0.01\sigma$).  
Это можно рассматривать как возможный прорыв в физике частиц.  

Дальнейшие исследования должны быть направлены на уточнение модели для кварков 
и проверку её связи с геометрией свёрнутых измерений.  

\end{document}
