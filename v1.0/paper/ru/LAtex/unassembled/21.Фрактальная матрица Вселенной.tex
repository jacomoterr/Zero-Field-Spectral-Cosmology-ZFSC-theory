\documentclass[12pt,a4paper]{article}
\usepackage[utf8]{inputenc}
\usepackage[T2A]{fontenc}
\usepackage[russian]{babel}
\usepackage{amsmath,amssymb}
\usepackage{hyperref}
\usepackage{geometry}
\usepackage{verbatim} % для блока цитирования
\geometry{margin=2.5cm}
\usepackage[T2A]{fontenc}
\usepackage[utf8]{inputenc}
\usepackage[russian]{babel}
\usepackage{braket}

\title{Zero-Field Spectral Cosmology (ZFSC): \\
Фрактальная матрица Вселенной, предел скорости, \\
запутанность, тахионные и аксионные слои, \\
и их вклад в наблюдаемый спектр}
\author{Евгений Монахов \\ LCC ``VOSCOM ONLINE'' Research Initiative \\ \href{https://orcid.org/0009-0003-1773-5476}{ORCID: 0009-0003-1773-5476}}
\date{Сентябрь 2025}

\begin{document}
\maketitle

\begin{abstract}
Представлена аксиоматическая формулировка ZFSC: фундаментальная эрмитова матрица $H$ фрактальной природы порождает спектр частиц и полей в виде ``луковичных'' слоёв. Предел скорости света $c$ выводится как максимальная скорость когерентного переноса фаз амплитуд в матричной геометрии. Запутанность между слоями и внутри слоёв даёт \emph{энергетические поправки} к наблюдаемому спектру. Вводится формализм \emph{тахионных поколений} (отрицательные собственные моды) и \emph{аксионного слоя} (лёгкие бозонные моды) с возможным слабым C/CP-нарушением на верхних уровнях, что ведёт к расщеплениям и вкладу в вакуумную плотность. Предложена фрактальная ``золотая лестница'' шкал, численный пайплайн и экспериментальные сигнатуры (коллайдерные, астрофизические, гравитационно-волновые).
\end{abstract}

%------------------------------------
\section{Постулаты и фрактальная матрица}
\textbf{Постулат 1 (Нулевой уровень).} Существует уровень с $S\to 0$, где нет пространства и времени. Вселенная описывается чистым вероятностным полем амплитуд
\[
\Psi = \sum_i a_i \,\ket{i}, \qquad a_i\in\mathbb{C}.
\]

\textbf{Постулат 2 (Матричная луковица).} Проявленная реальность --- это иерархия ``слоёв'', задаваемая базовой эрмитовой матрицей $H$ размерности $N\times N$:
\[
H\,\ket{\psi_k}=\lambda_k \ket{\psi_k},\quad \lambda_k\in\mathbb{R},
\]
где различные сектора (u, d, $\ell$, $\nu$; бозонный, тахионный, аксионный) порождаются геометрическими трансформациями $T_s$ и малыми деформациями $\Delta_s$:
\[
H_s = T_s\,H\,T_s^\dagger + \Delta_s .
\]

\textbf{Постулат 3 (Фрактальность).} Структура $H$ фрактальна: спектральные и топологические закономерности (включая масштабирование с золотым сечением $\varphi=\frac{1+\sqrt{5}}{2}$) повторяются на разных уровнях.

%------------------------------------
\section{Спектральная кинематика и предел скорости $c$}
Эволюция состояния:
\[
\ket{\Psi(t)}=\sum_k a_k e^{-i\lambda_k t/\hbar}\ket{\psi_k}.
\]
Определим эффективную координату $x$ и фазу $\varphi$ на многообразии собственных векторов. Тогда \emph{скорость переноса когерентности}
\[
v \;=\; \left|\frac{d\varphi}{dt}\cdot \frac{dx}{d\varphi}\right|.
\]
\textbf{Аксиома (предел скорости).}
\[
v_{\max}\equiv c \;=\; \max \left|\frac{d\varphi}{dt}\cdot \frac{dx}{d\varphi}\right|,
\]
общий для всех секторов, поскольку они происходят из одного $H$.

\textbf{Следствие (масса как сдвиг).} Для моды $i$ эффективная масса определяется сдвигом собственного значения:
\[
m_i=\frac{\lambda_i}{c^2}.
\]
Безмассовые ($\lambda\approx 0$) распространяются с $v=c$, массивные ($\lambda>0$) --- с $v<c$. Отрицательные моды ($\lambda<0$) тахионного типа нестабильны (разрушают когерентность слоя).

%------------------------------------
\section{Энергетические поправки от запутанности}
Пусть $I_{AB}$ --- мера межслойной взаимной информации/перекрытия собственных векторов между слоями $A$ и $B$, а $I_{\text{intra}}$ --- внутрислойная. Введём \emph{универсальную} поправку энергии для сектора $s$:
\begin{equation}
\Delta E^{(s)}_{\text{ent}} \;=\; \alpha\, I_{AB}^{(s)} \;+\; \beta\, I_{\text{intra}}^{(s)} ,
\label{eq:ent_energy}
\end{equation}
где $\alpha,\beta$ --- калибруемые коэффициенты. На уровне мод:
\[
I_{AB}^{(s)} \;\approx\; \sum_{i\in \mathcal{I}_s}\sum_{j\in \mathcal{J}_{\bar{s}}}
\big|\braket{\psi^{(s)}_i|\psi^{(\bar{s})}_j}\big|^2,
\quad
I_{\text{intra}}^{(s)} \;\approx\; -\sum_{i\in \mathcal{I}_s} p_i\ln(p_i+\epsilon),
\]
где $p_i$ --- нормированные веса/перекрытия внутри слоя, $\epsilon$ --- регулятор.

%------------------------------------
\section{Тахионные поколения: определение и вклад}
Определим тахионное подпространство (уровень/слой $\tau$):
\[
\mathcal{H}\ket{\tau_g}=\lambda^{(\tau)}_g \ket{\tau_g}, \qquad \lambda^{(\tau)}_g<0,\quad g=1,\dots,G.
\]
Поколение $g$ классифицируем по $|\lambda^{(\tau)}_g|$ (иерархия), глубине слоя и связности с видимыми секторами.

\subsection*{Смешивание с видимыми секторами}
Для сектора $s$ и его видимых мод $i\in \mathcal{I}_s$:
\[
M^{(s,g)}_i=\big|\braket{\psi^{(s)}_i|\tau_g}\big|^2,\qquad
\chi_{s,g}=\sum_{i\in \mathcal{I}_s} M^{(s,g)}_i .
\]
Эффективная константа смешивания:
\[
\kappa_{s,g}=\kappa_0\,\chi_{s,g}^{\,\gamma},\quad \gamma\in[0.5,1].
\]

\subsection*{Самоэнергии и итоговый спектр}
Стабилизированная самоэнергетическая поправка моды $i$ сектора $s$:
\begin{equation}
\Sigma^{(s)}_{i}\;=\;\sum_{g=1}^{G}
\frac{\kappa_{s,g}^2\, M^{(s,g)}_{i}\,|\lambda^{(\tau)}_g|}{\Lambda^2+|\lambda^{(\tau)}_g|}
\;-\;\eta_s \sum_{g=1}^G \kappa_{s,g}^2 M^{(s,g)}_{i},
\label{eq:selfenergy}
\end{equation}
где $\Lambda$ --- масштаб отсечки, $\eta_s\ge 0$ --- секторный контр-терм.
Итоговые уровни:
\[
\tilde{\lambda}^{(s)}_i \;=\; \lambda^{(s)}_i \;+\; \Sigma^{(s)}_{i} \;+\; \Delta E^{(s)}_{\text{ent}} .
\]

\subsection*{Фазовый потенциал тахионного слоя}
Для каждой $\tau_g$ введём феноменологический потенциал
\[
V_{\text{tach}}^{(g)}(\phi_g)= -\frac{1}{2}\mu_g^2\phi_g^2 + \frac{\lambda_g}{4}\phi_g^4,\qquad \mu_g^2\propto |\lambda^{(\tau)}_g|,
\]
что моделирует ``залечивание'' нестабильности ($\lambda^{(\tau)}_g\uparrow 0$) и космологические фазы.
Вакумный вклад:
\[
\rho_{\text{vac}}^{(\tau)} \approx \sum_g \Big(V_{\text{tach}}^{(g)}(\langle\phi_g\rangle)-V_{\text{tach}}^{(g)}(0)\Big).
\]

%------------------------------------
\section{C/CP-асимметрия в верхних слоях}
Для тахионного ($\tau$) и аксионного ($a$) слоёв вводим удвоенное пространство состояний (частица $\oplus$ античастица) и блочный гамильтониан
\[
H_{\text{ext}}^{(X)}=
\begin{pmatrix}
H_X & \Delta_X \\
\Delta_X^\dagger & H_{X,C}
\end{pmatrix},\quad X\in\{\tau,a\},
\]
где $H_{X,C}=C\,H_X^* C^{-1}$, а $\Delta_X=\varepsilon_X e^{i\phi_X} P_X$ с проектором $P_X$ на соответствующее подпространство. Малые параметры $\varepsilon_X,\phi_X$ описывают слабое C/CP-нарушение.

\subsection*{Расщепление спектра и фон}
Для моды $g$ слоя $X$:
\[
\lambda^{(X)}_{g,\pm}=\lambda^{(X)}_{g}\pm \frac{\delta^{(X)}_g}{2},\qquad
\delta^{(X)}_g \approx 2\varepsilon_X \big|\bra{\psi^{(X)}_g}P_X\ket{\psi^{(X)}_g}\big|\cos\phi_X .
\]
Вклад в вакуумную плотность (квадартичное приближение при малых $\delta$):
\[
\Delta\rho_{\text{vac}}^{(X)} \;\approx\; \frac{1}{4\mathcal{V}} \sum_g [\delta^{(X)}_g]^2\, F''(\lambda^{(X)}_g),
\]
где $F$ --- выбранная спектральная плотность с отсечкой; $\mathcal{V}$ --- нормировочный объём.
Усиление асимметрии запутанностью (перекрытием) реализуем заменой
\[
\delta^{(X)}_{g}\;\to\; \delta^{(X)}_{g}\,(1+\beta_X\,\chi_{s,g}).
\]

%------------------------------------
\section{Аксионы и ``золотая'' фрактальная лестница}
Фрактальная иерархия естественно задаёт семейство низкомассовых бозонных (аксионных/ALP) мод. Удобная параметризация переходов между уровнями:
\[
\Delta\lambda_k \;\approx\; \Lambda_0^2\,\varphi^{2k},\qquad k\in\mathbb{Z},
\]
где $\Lambda_0$ калибруется по наименьшему положительному межплатовому зазору в данных сектора. Тогда при проекции на фотонный/нуклонный подпространства получаем прикидку масс:
\[
m_a \;\sim\; \xi \frac{\Delta\lambda_k}{c^2},\qquad
\xi\in(0,1] \text{ --- фактор проекции (купплинга)}.
\]
Практически релевантные окна $k$ подбираются по максимумам перекрытий с фотонным и нуклонным секторами.

%------------------------------------
\section{Численный пайплайн (сканы/фиты)}
Для каждого узла параметров $(g_L,g_R,h_1,h_2,h_3,\dots)$:
\begin{enumerate}
  \item Построить $H_{\text{all}}$ (слойная сборка с межслойными связями).
  \item Найти $K$ нижних собственных значений/векторов ($\mathrm{eigsh}$, which=SA); выделить $\lambda^{(\tau)}_g<0$ и $\ket{\tau_g}$.
  \item Для каждого сектора $s$: решить $H_s\ket{\psi^{(s)}_i}=\lambda^{(s)}_i\ket{\psi^{(s)}_i}$; вычислить $M^{(s,g)}_i$, $\chi_{s,g}$, затем $\kappa_{s,g}$.
  \item Посчитать $\Sigma^{(s)}_i$ по (\ref{eq:selfenergy}) и $\Delta E^{(s)}_{\text{ent}}$ по (\ref{eq:ent_energy}); получить $\tilde{\lambda}^{(s)}_i$.
  \item Для верхних слоёв ($X=\tau,a$): собрать $H_{\text{ext}}^{(X)}$, оценить $\delta^{(X)}_g$ и $\Delta\rho_{\text{vac}}^{(X)}$.
  \item Целевая функция фита: невязка плато масс + CKM/PMNS + регуляризация на (i) ``жёсткие'' отрицательные моды $\sum_g |\lambda^{(\tau)}_g|$, (ii) нежелательные большие $\Delta\rho_{\text{vac}}$ вне космологического окна, (iii) штраф на чрезмерную $\eta_s$.
\end{enumerate}

%------------------------------------
\section{Экспериментальные сигнатуры}
\textbf{Коллайдеры (БАК):} нет узких пиков; ожидаются (i) систематический избыток мягких $p_T$ и дальние корреляции ($\Delta\eta$) как следствие фоновой когерентности; (ii) слабые энерго-зависимые деформации в лептонных финалах (через петлевые сдвиги смешивания). Полезны форвард-детекторы.

\textbf{Астро/космо:} (i) энерго-зависимые задержки/фазовые корреляции в нейтрино/ГВ-сигналах; (ii) вклад ``залечивания'' отрицательных мод в НЧ спектр стохастического ГВ-фона; (iii) для аксионов --- окна $\mu\mathrm{eV}$--$\mathrm{meV}$ (haloscope), более лёгкие --- оптические LSW-поиски; спиновые NMR-поиски при нано-эВ.

%------------------------------------
\section*{Итог}
ZFSC сводит сложность физики частиц и космологии к единой фрактальной матрице $H$. Предел $c$ --- спектральный максимум скорости когерентного переноса. Запутанность вносит универсальные энергетические поправки, тахионные поколения и аксионная лестница дают тонкую структуру спектра и вклад в фон, а слабое C/CP-нарушение на верхних слоях приводит к расщеплениям и измеримым эффектам.

%------------------------------------
\appendix
\section{Обозначения и параметры}
\begin{tabular}{ll}
\toprule
Символ & Смысл \\
\midrule
$H$ & базовая эрмитова матрица \\
$H_s$ & секторная матрица (u,d,$\ell$,$\nu$, бозонный и т.д.) \\
$\lambda_i,\ket{\psi_i}$ & собственные пары \\
$\Delta_s$ & малая деформация сектора \\
$I_{AB},\,I_{\text{intra}}$ & меры межслойной/внутрислойной запутанности \\
$\Delta E^{(s)}_{\text{ent}}$ & энергетическая поправка от запутанности \\
$\ket{\tau_g},\,\lambda^{(\tau)}_g<0$ & тахионные моды и их уровни \\
$M^{(s,g)}_i,\chi_{s,g}$ & перекрытия и агрегированная связность \\
$\kappa_{s,g}$ & эффективная константа смешивания \\
$\Sigma^{(s)}_i$ & самоэнергетическая поправка \\
$\Lambda,\eta_s$ & отсечка, контр-терм \\
$H_{\text{ext}}^{(X)}$ & блочный гамильтониан (частица/античастица) \\
$\varepsilon_X,\phi_X$ & параметры C/CP-асимметрии слоя $X$ \\
$\delta^{(X)}_g$ & расщепление пары ($\pm$) \\
$\Delta\rho_{\text{vac}}^{(X)}$ & вклад слоя $X$ в вакуумную плотность \\
$\Lambda_0,\varphi$ & базовый масштаб, золотое сечение \\
\bottomrule
\end{tabular}

\end{document}
