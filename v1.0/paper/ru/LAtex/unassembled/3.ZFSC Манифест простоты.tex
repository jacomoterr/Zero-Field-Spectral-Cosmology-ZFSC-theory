\documentclass[12pt,a4paper]{article}
\usepackage[utf8]{inputenc}
\usepackage[T2A]{fontenc}
\usepackage[russian]{babel}
\usepackage{amsmath,amssymb}
\usepackage{hyperref}
\usepackage{geometry}
\usepackage{verbatim} % для блока цитирования
\geometry{margin=2.5cm}
\usepackage[T2A]{fontenc}
\usepackage[utf8]{inputenc}
\usepackage[russian]{babel}
\usepackage{braket}

\title{ZFSC: Манифест простоты \\
Фрактальная матрица и инженерная основа Вселенной}
\author{Евгений Монахов \\ LCC ``VOSCOM ONLINE'' Research Initiative}
\date{Сентябрь 2025}

\begin{document}
\maketitle

\section*{Введение}
Современная физика частиц кажется безнадёжно сложной. Но в основе может лежать удивительно простая конструкция: \emph{фрактальная эрмитова матрица $H$}, порождающая весь спектр частиц и полей.  
Этот подход называется \textbf{Zero-Field Spectral Cosmology (ZFSC)}.

\section*{Ключевые идеи}
\begin{enumerate}
  \item \textbf{Фрактал:} матрица $H$ самоподобна на всех уровнях. Она поддерживает саму ``ткань'' спектра.
  \item \textbf{Геометрия связей:} многомерная структура связей определяет массы, симметрии и смешивания.
  \item \textbf{Предел скорости $c$:} это не свойство Минковского пространства, а спектральный максимум скорости когерентного переноса фаз.
  \item \textbf{Запутанность:} корреляции между слоями дают энергетические поправки к массам и матрицам смешивания.
  \item \textbf{Тахионы:} отрицательные собственные моды, сгруппированные в поколения, отражают нестабильности и фазовые переходы. Они вносят вклад в космологический фон.
  \item \textbf{Аксионы:} лёгкие фрактальные моды, ``золотая лестница'' масс. Кандидаты на тёмную материю.
  \item \textbf{Асимметрия:} слабое C/CP-нарушение в тахионном и аксионном слоях ведёт к расщеплениям и добавочному вакуумному фону.
\end{enumerate}

\section*{Формулы-маркеры}
\begin{itemize}
  \item Энергия состояния:
  \[
  H\ket{\psi_i}=\lambda_i\ket{\psi_i},\qquad m_i=\frac{\lambda_i}{c^2}.
  \]
  \item Предел скорости:
  \[
  c=\max\left|\frac{d\varphi}{dt}\cdot\frac{dx}{d\varphi}\right|.
  \]
  \item Поправка от запутанности:
  \[
  \Delta E_s=\alpha\,I_{AB}+\beta\,I_{\text{intra}}.
  \]
  \item Тахионные поколения:
  \[
  \lambda^{(\tau)}_g<0,\qquad g=1,2,3.
  \]
  \item Аксионная лестница:
  \[
  \Delta\lambda_k\approx \Lambda_0^2\varphi^{2k}.
  \]
\end{itemize}

\section*{Итог}
ZFSC демонстрирует: Вселенная устроена как \textbf{гениальная инженерная конструкция}.  
Фрактал поддерживает матрицу на всех уровнях, а многомерная геометрия связей делает спектр живым и динамичным.  
То, что мы наблюдаем --- лишь внешняя проекция глубокой спектральной матрицы.

\end{document}
