\documentclass[12pt,a4paper]{article}
\usepackage[utf8]{inputenc}
\usepackage[russian]{babel}
\usepackage{amsmath,amssymb,amsfonts}
\usepackage{bm}
\usepackage{physics}
\usepackage{siunitx}
\usepackage{booktabs}
\usepackage{hyperref}
\usepackage{geometry}
\geometry{margin=2.3cm}
\hypersetup{colorlinks=true,linkcolor=blue,citecolor=blue,urlcolor=blue}
\usepackage[utf8]{inputenc}
\usepackage[russian]{babel}
\usepackage{tikz}
\usetikzlibrary{arrows.meta}
\usetikzlibrary{positioning}

\title{Zero-Field Spectral Cosmology (ZFSC) \\ 
Формулы и гипотезы}
\author{Евгений Монахов \\ VOSCOM ONLINE Research Initiative}
\date{\today}

\begin{document}
\maketitle

\section*{Строгие постулаты и выводы}

\subsection*{Постулат 1: Нулевой уровень энтропии}
\[
S \to 0, \qquad 
\Psi = \sum_{i} a_i |i\rangle .
\]

\subsection*{Постулат 2: Спектральная масса}
Физические массы поколений частиц соответствуют собственным значениям матрицы $H$:
\[
m_f^{(n)} \;\;\equiv\;\; \lambda_n(H), \qquad n=1,2,3.
\]

\subsection*{Постулат 3: Матрицы смешивания}
\[
\mathrm{CKM} = U_u^\dagger U_d, 
\qquad
\mathrm{PMNS} = U_\ell^\dagger U_\nu .
\]

\subsection*{Ключевой результат (аналог $E=mc^2$)}
\[
\boxed{ \; m \;\;=\;\; \lambda(H) \;}
\]
Масса частицы есть спектральное собственное значение фундаментальной матрицы.

\bigskip

\section*{Гипотезы и расширения}

\begin{itemize}
  \item \textbf{Нулевая мода:} $\lambda_0 \approx 0$ — кандидат на гравитон.  
  \item \textbf{Отрицательная мода:} $\lambda < 0$ — тахионные поколения.  
  \item \textbf{Связность узлов:} распределения степеней соответствуют калибровочным симметриям $SU(3)\times SU(2)\times U(1)$.  
  \item \textbf{Запутанность:} энергетические поправки через взаимную информацию между слоями.  
\end{itemize}

\section*{План исследований}

1. Численные сканы спектров матриц $H$ больших размерностей.  
2. Фиты масс и матриц смешивания к экспериментальным данным.  
3. Изучение тахионных мод и их вкладов в космологию.  
4. Моделирование временной эволюции эффективных констант $G_\mathrm{eff}(t)$, $\alpha(t)$.  

\end{document}
