\documentclass[12pt,a4paper]{article}
\usepackage[utf8]{inputenc}
\usepackage[russian]{babel}
\usepackage{amsmath,amssymb,amsfonts}
\usepackage{bm}
\usepackage{physics}
\usepackage{siunitx}
\usepackage{booktabs}
\usepackage{hyperref}
\usepackage{geometry}
\geometry{margin=2.3cm}
\hypersetup{colorlinks=true,linkcolor=blue,citecolor=blue,urlcolor=blue}

\title{Квантовое сознание и матричная космология (ZFSC):\\
биологические подсистемы мозга как спектральные резонаторы Вселенной}
\author{Евгений Монахов\thanks{ООО «VOSCOM ONLINE» Research Initiative; ORCID: 0009-0003-1773-5476}}
\date{Сентябрь 2025}

\begin{document}
\maketitle

\begin{abstract}
Мы формулируем строгую, проверяемую гипотезу \emph{квантового сознания} в рамке Zero-field Spectral Cosmology (ZFSC). Мозг трактуется как вложенная матричная подсистема со своим гамильтонианом $H_{\text{brain}}$, спектром устойчивых мод (мысле-паттернов) и поправками на межслойную запутанность, совместимыми с ранее введёнными энергетическими сдвигами $\Delta E_s=\alpha I_{AB}+\beta I_{\text{intra}}$. Проводится сопоставление с известными квантово-биологическими механизмами (вибронная когерентность, радикальные пары, гипотеза спиновых кубитов фосфора, микротрубочки), обсуждаются ограничения по декогеренции и экспериментальные протоколы верификации. Предложена феноменологическая связь с глобальной «луковичной» матрицей Вселенной $H_{\text{universe}}$ через слабую спектральную связь и условия резонанса. Указаны проверяемые предсказания и границы применимости подхода.
\end{abstract}

\section{Постановка задачи}
Классическая нейрофизиология успешно описывает потенциалы действия и синаптическую пластичность, но оставляет открытыми вопросы интеграции опыта и возникновения единых субъективных состояний (квалий). Квантово-биологические эффекты обнаружены в ряде тёплых биосистем (фото-синтетическая когерентность, радикальные пары, шумо-ассистированный транспорт), что мотивирует осторожное рассмотрение квантовых вкладов в мозге. Наша цель — задать \emph{минимально достаточную} математическую рамку в духе ZFSC, совместимую с экспериментом и не противоречащую известным оценкам декогеренции.

\section{ZFSC как надсистема и мозг как вложенная матрица}
В ZFSC реальность представлена вложенными матричными уровнями («луковица»), где устойчивые плато собственных значений соответствуют наблюдаемым спектрам частиц и полей. Введём прямую сумму
\begin{equation}
H_{\text{universe}} = H_{\text{cosmo}} \oplus H_{\text{matter}} \oplus H_{\text{brain}} + \varepsilon\, V_{\text{cpl}},
\end{equation}
где $H_{\text{brain}}$ — биологическая подсистема, $V_{\text{cpl}}$ — слабая спектральная связь (например, через гравитационно-геометрические флуктуации или электромагнитный фон), $\varepsilon\ll 1$.

\paragraph{Био-гамильтониан.}
Мозг моделируется композиционно:
\begin{equation}
H_{\text{brain}} = H_{\text{micro}} + H_{\text{neuro}} + H_{\text{syn}} + H_{\text{spin}} + H_{\text{chem}},
\end{equation}
где $H_{\text{micro}}$ — внутримикротрубочечные взаимодействия; $H_{\text{neuro}}$ — динамика мембран и аксонов; $H_{\text{syn}}$ — синаптическая сеть; $H_{\text{spin}}$ — потенциальные спиновые степени свободы (напр. \,$^{31}$P в фосфатных кластерах); $H_{\text{chem}}$ — радикальные пары и прочие химические пути.

\paragraph{Графовая структура.}
На уровне сети рассмотрим ориентированный взвешенный граф $G=(\mathcal{V},\mathcal{E})$ нейронных и микротрубочечных узлов. С учётом мультиуровневости используем блочное представление:
\begin{equation}
\bm{A}=
\begin{pmatrix}
A_{\text{micro}} & C_{\text{m}\leftrightarrow \text{n}}\\
C_{\text{n}\leftrightarrow \text{m}} & A_{\text{neuro}}
\end{pmatrix},\qquad
\bm{L}=\bm{D}-\bm{A},
\end{equation}
где $\bm{A}$ — матрица смежности, $\bm{L}$ — лапласиан, $\bm{D}$ — диагональная матрица степеней. Спектр $\{\lambda_i(\bm{L})\}$ задаёт естественные моды согласованных колебаний.

\section{Сознательные состояния как устойчивые спектральные моды}
Определим \emph{сознательное состояние} как квази-стационарный кластер собственных мод $H_{\text{brain}}$ с малыми дрейфами по времени:
\begin{equation}
\frac{d\lambda_i}{dt}\approx 0,\quad \forall \lambda_i \in \mathcal{C}_{\text{cons}},
\end{equation}
и ненулевой межуровневой интеграцией информации. Следуя ZFSC-введению поправок на запутанность, положим
\begin{equation}\label{eq:deltaEs}
\Delta E_s = \alpha\, I_{AB} + \beta\, I_{\text{intra}},
\end{equation}
где $I_{AB}$ — взаимная информация между крупными областями мозга (лобные, теменные, лимбическая система и т.п.), $I_{\text{intra}}$ — когерентность/взаимная информация внутри подуровней (например, внутри ансамбля микротрубочек).

Тогда \emph{энергия сознательного эпизода} в феноменологической аппроксимации:
\begin{equation}
E_{\text{cons}} = \sum_{\lambda_i \in \mathcal{C}_{\text{cons}}} \lambda_i + \Delta E_s,
\end{equation}
а интеграционная метрика может быть соотнесена с IIT-подобной величиной $\Phi$:
\begin{equation}
\Phi \sim \sum_{(A,B)\in \mathcal{P}} I(A\!:\!B) - \sum_{S\in \mathcal{S}} I_{\text{cut}}(S),
\end{equation}
сопоставляя $I(\cdot\!:\!\cdot)$ с измеримой взаимной информацией функциональной нейровизуализации и внутренними метриками графа $G$.

\section{Когерентность: биофизические кандидаты и ограничения}
\subsection{Вибронная когерентность и шумо-ассистированный транспорт}
Эксперименты в фотосинтетических комплексах демонстрируют когерентные осцилляции при комнатных температурах на фемто-пико-секундных шкалах, что интерпретируется как квантовая волновая передача возбуждений; в моделях важную роль играют вибрации и \emph{оптимум} при конечной декогеренции (ENAQT). Эти механизмы указывают, что \emph{биосистемы способны использовать квантовые эффекты на тёплых масштабах}. 

\subsection{Радикальные пары}
Механизм радикальных пар (магнеторецепция птиц) показывает чувствительность биохимических реакций к слабым магнитным полям через когерентную спиновую динамику. Это демонстрирует реальность \emph{функциональной} когерентности спинов в «мокрой» биосреде.

\subsection{Спиновые кубиты фосфора}
Гипотеза ядерных спинов $^{31}$P (Posner-молекулы) предлагает долгоживущую квантовую память и пути химической генерации/транспортировки запутанных пар. Это совместимо с $H_{\text{spin}}$ и предоставляет естественные T$_2$-кандидаты для мозговой среды.

\subsection{Микротрубочки и тера-/гигагерцовые моды}
Ряд работ указывает на резонансные электродинамические свойства микротрубочек (гига- и терагерцовые моды, эффекты гистерезиса/«многослойной памяти» на одиночной микротрубочке). Мы трактуем это как вклад в $H_{\text{micro}}$ и $I_{\text{intra}}$ (формула \ref{eq:deltaEs}). 

\subsection{Оценки декогеренции}
Тегмарк дал пессимистичные оценки характерных времён декогеренции $10^{-13}\!-\!10^{-20}$\,с для типичных мозговых степеней свободы. Современные работы указывают, что \emph{в специфицированных каналах} (защищённые спины, коррелированная шумовая среда, \emph{диссипативная стабилизация}) окна когерентности могут быть существенно длиннее. Мы принимаем консервативный рабочий диапазон
\[
\tau_\phi \sim 10^{-9}\text{ c} \dots 10^{-4}\text{ c}
\]
для \emph{кандидатных} подсистем, подчёркивая модельную зависимость и необходимость прямых измерений.

\section{Связь с матрицей Вселенной: спектральный резонанс}
Пусть $\{\lambda_i\}$ — собственные значения $H_{\text{brain}}$, а $\{\Lambda_j\}$ — спектр соответствующего слоя $H_{\text{cosmo}}$ (в рамке ZFSC). Введём \emph{спектральное перекрытие}
\begin{equation}
\mathcal{S}(\delta) = \sum_{i,j} \exp\!\left[-\frac{(\lambda_i-\Lambda_j)^2}{2\delta^2}\right] \,\abs{\langle u_i | U | v_j\rangle}^2,
\end{equation}
где $|u_i\rangle$ и $|v_j\rangle$ — собственные векторы локальной и космологической подсистем, $U$ — слабая связь/проекционный оператор. Когерентное «прилипание» сознательных кластеров к вселенским модам ожидаемо при больших значениях $\mathcal{S}(\delta)$.

\paragraph{Эффективная динамика.}
В первом порядке по $\varepsilon$:
\begin{equation}
H_{\text{eff}} = H_{\text{brain}} + \varepsilon^2 \sum_{j} \frac{V \, \outerproduct{v_j}{v_j} \, V^\dagger}{E-\Lambda_j+i0^+},
\end{equation}
что ведёт к \emph{сдвигам Ламба} и \emph{эффективным куплингам} между мозговыми модами через «фон» космологического слоя. В рамках ZFSC такие сдвиги можно интерпретировать как малые поправки к плато собственных значений.

\section{Энергетические прикидки}
Пусть тубулин выступает как двухуровневая система с эффективным разностью уровней $\Delta \varepsilon \sim 10^{-2}$\,eV ($\sim 10^{-21}$\,J). Ансамбль когерентных тубулинов $N_c$ даёт характерную энергию
\[
E_c \sim N_c \Delta\varepsilon.
\]
Для $N_c \sim 10^6$ получаем $E_c \sim 10^{-15}$\,J (сопоставимо с энергией $10^{4}\!-\!10^{5}$ фотонов видимого диапазона). Это ещё не «мозг целиком», но уже измеримый масштаб для резонансной тера-/гигагерцовой спектроскопии и для высокочувствительных магнитометров (OPM/SQUID).

\section{Экспериментально проверяемые предсказания}
\begin{enumerate}
\item \textbf{ТГц/ГГц-спектроскопия микротрубочек:} наличия стабильных полос сужается при \emph{снижении} шума до оптимума (ENAQT-окно); сверх- и недо-декогеренция уменьшают интеграционный индекс $\Phi$.
\item \textbf{Спиновые тесты $^{31}$P:} наблюдение аномально длинных T$_2$ в кальций-фосфатных кластерах in vitro/in vivo; генерация/детектирование энтангламента посредством ферментативных реакций (пирофосфат$\to$фосфаты).
\item \textbf{Нейро-магнитометрия:} узкополосные, воспроизводимые компоненты в МЭГ/OPM при когнитивных задачах, устойчивые к подавлению сосудистых и мышечных артефактов и не редуцируемые к классической осцилляторике.
\item \textbf{Леггетта–Гарга тесты:} проектирование макроскопических неинвазивных корреляций для отдельных подсистем (например, «мезоскопические» микротрубочечные домены) с контролем «clumsiness loophole».
\item \textbf{Радикальные пары:} повышенная чувствительность некоторых нейрохимических путей к слабым RF-помехам при частотах, совпадающих с гирамагнитными параметрами радикальных пар.
\end{enumerate}

\section{Границы применимости и фальсифицируемость}
\begin{itemize}
\item Подход \emph{не} утверждает «всеобщую квантовость» мозга: речь о \emph{локальных каналах} с потенциалом когерентности, строго ограниченных биофизикой.
\item Любая из следующих наблюдаемых фальсифицирует ключевые узлы гипотезы: (i) отсутствие узкополосных ТГц/ГГц-мод у микротрубочек при физиологических условиях; (ii) невозможность продемонстрировать T$_2$ выше классической нижней границы в фосфатных кластерах; (iii) отрицательные результаты Леггетта–Гарга при устранении всех лазеек.
\end{itemize}

\section{Интеграция с ZFSC и дорожная карта}
\subsection*{Сопоставление с формализмом ZFSC}
\begin{align}
&\text{(i) Закон устойчивости спектральных мод:}\quad d\lambda_n/dt \approx 0 \ \Rightarrow \ \text{устойчивое сознательное состояние};\\
&\text{(ii) Запутанность-энергетика:}\quad \Delta E_s=\alpha I_{AB}+\beta I_{\text{intra}} \ \text{вносит сдвиги плато;}\\
&\text{(iii) Расширенный гамильтониан:}\quad
H_{\text{ext}}=\begin{pmatrix}H & \Delta \\ \Delta^\dagger & H_C\end{pmatrix}
\ \text{(частично нарушенные C/CP-симметрии в скрытых слоях)}.
\end{align}

\subsection*{План работ}
\begin{enumerate}
\item \textbf{Спектральное картирование in vitro}: измерение $\{\lambda_i\}$ микротрубочек и фосфатных кластеров (ТГц/ГГц, Рамановская спектроскопия).
\item \textbf{Граф-моделирование $G$}: построение $A_{\text{micro}}$, $A_{\text{neuro}}$, оценка $\Phi$, $I_{AB}$, $I_{\text{intra}}$ на данных fMRI/МЭГ.
\item \textbf{Спин-эксперименты}: T$_1$/T$_2$ $^{31}$P, контроль химии Posner-молекул, RF-сдвиги.
\item \textbf{Мезоскопические LG-тесты}: протоколы слабых измерений, минимизация инвазивности.
\item \textbf{Космологическое перекрытие $\mathcal{S}(\delta)$}: численные сканы в ZFSC-ядре для оценки резонансных окон.
\end{enumerate}

\section*{Заключение}
Мы задали строго совместимую с ZFSC рамку квантового сознания: мозг рассматривается как матричная подсистема со спектрально устойчивыми модами и измеримыми каналами когерентности. Связь с «матрицей Вселенной» реализуется как слабосвязанная резонансная корректировка. Подход \emph{фальсифицируем} и предлагает конкретные экспериментальные проверки.

\begin{thebibliography}{99}
\bibitem{HameroffPenrose2014}
S.~Hameroff, R.~Penrose, Consciousness in the universe: A review of the Orch OR theory, \emph{Physics of Life Reviews} \textbf{11} (2014) 39–78. % Elsevier link
\bibitem{Tegmark2000}
M.~Tegmark, The importance of quantum decoherence in brain processes, \emph{Phys. Rev. E} \textbf{61} (2000) 4194–4206; arXiv:quant-ph/9907009.
\bibitem{Fisher2015}
M.~P.~A.~Fisher, Quantum cognition: The possibility of processing with nuclear spins in the brain, \emph{Annals of Physics} \textbf{362} (2015) 593–602.
\bibitem{Engel2007}
G.~S.~Engel et al., Evidence for wavelike energy transfer through quantum coherence in photosynthetic systems, \emph{Nature} \textbf{446} (2007) 782–786.
\bibitem{HuelgaPlenio2013}
S.~F.~Huelga, M.~B.~Plenio, Vibrations, quanta and biology, \emph{Contemporary Physics} \textbf{54} (2013) 181–207.
\bibitem{Lambert2013}
N.~Lambert et al., Quantum biology, \emph{Nature Physics} \textbf{9} (2013) 10–18.
\bibitem{HoreMouritsen2016}
P.~J.~Hore, H.~Mouritsen, The radical-pair mechanism of magnetoreception, \emph{Annual Review of Biophysics} \textbf{45} (2016) 299–344.
\bibitem{Tononi2004}
G.~Tononi, An information integration theory of consciousness, \emph{BMC Neuroscience} \textbf{5} (2004) 42.
\bibitem{Sahu2013}
S.~Sahu et al., Multi-level memory-switching properties of a single brain microtubule, \emph{Applied Physics Letters} \textbf{102} (2013) 123701.
\bibitem{Frohlich1968}
H.~Fr\"ohlich, Long-range coherence and energy storage in biological systems, \emph{Int. J. Quantum Chem.} \textbf{2} (1968) 641–649; см. также подборку трудов в сборнике \emph{Coherent Excitations in Biological Systems} (Springer, 1983).
\bibitem{LeggettGargReview}
C.~Emary, N.~Lambert, F.~Nori, Leggett–Garg inequalities, \emph{Rep. Prog. Phys.} \textbf{77} (2014) 016001.
\end{thebibliography}

\end{document}
