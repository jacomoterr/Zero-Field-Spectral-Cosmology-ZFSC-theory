\documentclass[12pt,a4paper]{article}
\usepackage[utf8]{inputenc}
\usepackage[T2A]{fontenc}
\usepackage[russian]{babel}
\usepackage{amsmath,amssymb}
\usepackage{hyperref}
\usepackage{geometry}
\usepackage{verbatim} % для блока цитирования
\geometry{margin=2.5cm}

\title{Гипотеза вероятностного поля:\\
От до-пространственных состояний к возникновению физики}
\author{Евгений Монахов \\ ООО "VOSCOM ONLINE" Research Initiative \\ https://orcid.org/0009-0003-1773-5476}
\date{Сентябрь 2025}

\begin{document}
\maketitle

\begin{abstract}
Предлагается гипотеза о том, что физическое пространство-время и взаимодействия возникают из более фундаментального вероятностного поля, существующего на нулевом уровне энтропии. В этом состоянии отсутствуют пространство и время, а присутствуют лишь амплитуды и вероятностные поля, представляющие потенциальные конфигурации всех возможных энергий и взаимодействий. Сформулированы базовые постулаты, приведены предварительные математические соотношения и набросан план исследований, направленных на сопоставление данной модели с известными физическими законами и константами.
\end{abstract}

%------------------------------------
\section{Постулат 1: Нулевой уровень энтропии}
Предполагается существование фундаментального уровня, на котором отсутствуют время и пространство, а энтропия стремится к нулю:
\[
S \to 0.
\]
На этом уровне Вселенная описывается чистым вероятностным полем амплитуд:
\[
\Psi = \sum_{i} a_i |i\rangle ,
\]
где $\{|i\rangle\}$ — потенциальные конфигурации (пространства, энергии, взаимодействия), а $a_i \in \mathbb{C}$ — их амплитуды.

%------------------------------------
\section{Постулат 2: Энергия как чистый потенциал}
Энергия на этом уровне существует в виде потенциальной возможности, а не реализованной динамики:
\[
E = \frac{1}{2}k |u|^2 ,
\]
где $u$ — ``смещение'' в вероятностном поле, $k$ — универсальный коэффициент жёсткости.

Квантованная форма:
\[
E_n = \hbar \omega \left(n + \tfrac{1}{2}\right),
\]
где $\omega$ определяется не геометрией, а структурой вероятностного поля.

%------------------------------------
\section{Постулат 3: Возникновение пространства-времени}
Появление времени и пространства моделируется как процесс декогеренции:
\[
\Psi \;\xrightarrow{\text{декогеренция}}\; \rho(x,t) = |\Psi(x,t)|^2 .
\]
Таким образом, координаты $(x,t)$ являются производными явлениями, возникающими из амплитудной структуры.

%------------------------------------
\section{Постулат 4: Свёрнутые и развёрнутые моды}
Не все моды разворачиваются в макропространство. Энергия распределяется по размерностям:
\[
E = \sum_{D=0}^{\infty} \sum_{n_D} \hbar \omega_D \left(n_D + \epsilon_D\right),
\]
где $\omega_D$ — собственные частоты $D$-мерных мод, а $\epsilon_D$ — нулевая энергия.  
Моды с большими $\omega_D$ остаются свёрнутыми (например, на многообразиях Калаби–Яу), а с малыми $\omega_D$ разворачиваются в наблюдаемое макропространство.

%------------------------------------
\section{Постулат 5: Энтропийная динамика}
Рост энтропии соответствует разворачиванию пространств:
\begin{itemize}
  \item При $S=0$ все измерения существуют лишь как потенциальные моды.
  \item При $S>0$ часть мод декогерирует и формирует развёрнутое пространство-время.
\end{itemize}

%------------------------------------
\section{Программа исследований}
\subsection*{Этап 1. Математическая формализация}
\begin{itemize}
  \item Построить модель $\Psi$ как вероятностного поля без координат.
  \item Ввести спектр $\{\omega_D\}$ как универсальные моды.
  \item Связать $\omega_D$ с фундаментальными константами $\hbar, c, G, k_B$.
\end{itemize}

\subsection*{Этап 2. Сведение к известным законам}
\begin{itemize}
  \item Проверить, что при проекции на $3+1$ размерности воспроизводятся $E = mc^2$, уравнение Шрёдингера и уравнения Эйнштейна.
  \item Перенормировать коэффициенты $k, \omega_D$ в планковских единицах.
\end{itemize}

\subsection*{Этап 3. Численное моделирование}
\begin{itemize}
  \item Смоделировать простейшие вероятностные поля (две–три суперпозиции мод).
  \item Отслеживать, какие моды разворачиваются в ``пространство'' при росте энтропии.
  \item Сравнить с известными схемами компактификации многообразий Калаби–Яу.
\end{itemize}

\subsection*{Этап 4. Связь со стандартной моделью}
\begin{itemize}
  \item Попробовать выразить константы взаимодействий (электрослабое, сильное, гравитация) через параметры $\omega_D$.
  \item Сравнить с наблюдаемыми величинами после перенормировки.
\end{itemize}

\subsection*{Этап 5. Эмпирические следы}
\begin{itemize}
  \item Искать отпечатки свёрнутых мод в спектре реликтового излучения и распределении тёмной энергии.
  \item Предположение: малые флуктуации из свёрнутых измерений могут проявляться как шум или аномалии в наблюдаемом спектре.
\end{itemize}

%------------------------------------
\section{Заключение}
Предложена модель, в которой Вселенная возникает из вероятностного поля на нулевом уровне энтропии, а пространство-время, энергия и взаимодействия появляются в результате декогеренции и разворачивания мод. Данный подход меняет направление: не от пространства к волне, а от волны к пространству. Дальнейшие исследования предполагают перенормировку коэффициентов для воспроизведения известных физических законов и поиск экспериментальных следов.

%------------------------------------
\section*{Цитирование (BibTeX)}
\begin{verbatim}
@misc{prob_field_hypothesis_ru_2025,
  author       = {Евгений Монахов and ООО "VOSCOM ONLINE" Research Initiative},
  title        = {Гипотеза вероятностного поля: От до-пространственных состояний к возникновению физики},
  year         = {2025},
  publisher    = {Zenodo},
  orcid        = {0009-0003-1773-5476},
  url_orcid    = {https://orcid.org/0009-0003-1773-5476},
  organization = {https://voscom.online/}
}
\end{verbatim}

\end{document}
