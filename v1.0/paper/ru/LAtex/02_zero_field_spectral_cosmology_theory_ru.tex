\documentclass[12pt,a4paper]{article}
\usepackage[utf8]{inputenc}
\usepackage[T2A]{fontenc}
\usepackage[russian]{babel}
\usepackage{amsmath}
\usepackage{amssymb}
\usepackage{amsthm}
\usepackage{mathtools}
\usepackage{hyperref}
\usepackage{geometry}
\usepackage{verbatim}
\usepackage{enumitem}
\usepackage{bm}

\geometry{margin=2.5cm}

\title{Спектральная космология нулевого поля. Теория.\\
(Zero-field spectral cosmology. Theory)}
\author{Евгений Монахов \\ ООО "VOSCOM ONLINE" Research Initiative \\ \href{https://orcid.org/0009-0003-1773-5476}{ORCID: 0009-0003-1773-5476}}
\date{}

\begin{document}
\maketitle

\begin{abstract}
Предлагается гипотеза о том, что физическое пространство-время и взаимодействия возникают из более фундаментального вероятностного поля, существующего на нулевом уровне энтропии. В этом состоянии отсутствуют пространство и время, а присутствуют лишь амплитуды и вероятностные поля, представляющие потенциальные конфигурации всех возможных энергий и взаимодействий. Сформулированы базовые постулаты, приведены предварительные математические соотношения и набросан план исследований, направленных на сопоставление данной модели с известными физическими законами и константами.
\end{abstract}


\section{Вводная интуиция и постулаты ZFSC}


\subsection{Постулат 1: Нулевой уровень энтропии}
Предполагается существование фундаментального пред-геометрического уровня, на котором отсутствуют классические расстояния, пространственне и временные измерения, а энтропия стремится к нулю:
\[
S \to 0.
\]
Формально начальное состояние представляется чистым квантовым состоянием \(\rho\) на вероятностно-амплитудной структуре \(\mathcal H\) с нулевой энтропией
\[
S(\rho)=-\mathrm{Tr}\,(\rho\ln\rho)=0.
\]
На этом уровне Вселенная описывается чистым вероятностным полем амплитуд:
\[
\Psi = \sum_{i} a_i |i\rangle ,
\]
где $\{|i\rangle\}$ — потенциальные конфигурации (пространства, энергии, взаимодействия), а $a_i \in \mathbb{C}$ — их амплитуды.

\begin{itemize}
    \item Обозначим гильбертово пространство «потенциальных состояний» \(\mathcal{H}\).
    \item На \(\mathcal{H}\) задан \textbf{самосопряжённый оператор} (наблюдаемый)  
    \[
    \boxed{\Lambda = L + M} \tag{1}
    \]
    где \(L\) — «внутрисекторная» часть (локальные связи), \(M\) — «межсекторные» связи (смешивания).
\end{itemize}

\textbf{Физический смысл.} Спектр \(\{\lambda_k^{\rm eff}\}\) оператора \(\Lambda\) кодирует потенциальные «частоты» \(\tilde{\omega}_k\) элементарных \textbf{мод}, из которых потом эмерджируют частицы, поля и геометрия.

\subsection{Собственные моды и базовые формулы}
\[
\Lambda \mathbf{v}_k = \lambda_k^{\rm eff} \mathbf{v}_k, \qquad
\tilde{\omega}_k \equiv \sqrt{\lambda_k^{\rm eff}} \ (\ge 0). \tag{2}
\]

\begin{itemize}
    \item \(\mathbf{v}_k\) — собственный вектор (форма «моды»).
    \item \(\lambda_k^{\rm eff} \ge 0\) — собственное значение (квадрат «частоты»).
    \item \(\tilde{\omega}_k\) — эффективная «частота» моды.
\end{itemize}

\textbf{Массы частиц.}
\[
\boxed{m_k = \frac{\hbar}{c^2} \tilde{\omega}_k = \frac{\hbar}{c^2} \sqrt{\lambda_k^{\rm eff}}} \tag{3}
\]
\begin{itemize}
    \item \(\hbar\) — редуцированная постоянная Планка.
    \item \(c\) — скорость света в вакууме.
\end{itemize}

\textbf{Смешивания (PMNS/CKM).}
\[
\boxed{U_{\alpha i} \sim \langle \alpha | \mathbf{v}_i \rangle} \tag{4}
\]
\begin{itemize}
    \item \(|\alpha\rangle\) — базис «ароматов/секторов» (электронный, мюонный, и т.д.).
    \item Перекрытия собственных векторов дают \textbf{углы смешивания} и \textbf{фазу CP}.
\end{itemize}

\section{Эмерджентная геометрия: как «рождаются» время и пространство}

\subsection{Спектральный переход (ZFST): «Великое развёртывание»}
\textbf{Гипотеза перехода.} «Большой взрыв» заменяем на \textbf{спектральный переход нулевого поля (ZFST)} — быстрый режим роста связности и появления не-нулевой энтропии \(S\).

Вводим «прото-время» \(\tau\) — параметр эволюции спектра под действием некоторого \textbf{градиентного потока} (минимизации «спектрального действия»):
\[
\frac{d\Lambda}{d\tau} = -\frac{\delta \mathcal{S}_{\rm spec}}{\delta \Lambda}, \qquad
\mathcal{S}_{\rm spec} = \mathrm{Tr}\, f\left( \frac{\Lambda}{\Lambda_*} \right). \tag{5}
\]
\begin{itemize}
    \item \(f\) — положительная затухающая функция (например, сглаженный срез спектра).
    \item \(\Lambda_*\) — масштаб отсечки (Планков порядок).
    \item \textbf{Смысл:} система «раскладывает» высокие и низкие моды в структуру с минимальным «спектральным действием».
\end{itemize}

\textbf{Эмерджентное физическое время.}
\[
\boxed{t(\tau) = \int^{\tau} \zeta \big( S(\tau') \big) \, d\tau', \quad \zeta' > 0} \tag{6}
\]
\begin{itemize}
    \item \(\zeta(S)\) — монотонная «скорость часов»: пока \(S \approx 0\), физическое время «почти стоит»; при росте \(S\) часы «включаются».
\end{itemize}

\subsection{Спектральный зазор и масштабный фактор}
Определим \textbf{первый ненулевой зазор}:
\[
\lambda_1(\tau) = \min \{ \lambda_k^{\rm eff}(\tau) > 0 \}, \qquad
\xi(\tau) \sim \frac{1}{\sqrt{\lambda_1(\tau)}}. \tag{7}
\]
\begin{itemize}
    \item \(\xi\) — корреляционная длина (размер областей когерентности).
    \item \textbf{Допущение:} масштабный фактор \(a \propto \xi\):
    \[
    \boxed{a(\tau) \propto \frac{1}{\sqrt{\lambda_1(\tau)}}} \ \Rightarrow \
    H \equiv \frac{\dot{a}}{a} = -\frac{1}{2} \frac{\dot{\lambda}_1}{\lambda_1}. \tag{8}
    \]
\end{itemize}
Если на фазе ZFST \(\lambda_1(\tau)\) падает \textbf{экспоненциально},
\[
\lambda_1(t) = \lambda_1(0) \, e^{-2Ht} \ \Rightarrow \ a(t) \propto e^{Ht}, \tag{9}
\]
получаем \textbf{инфляцию без инфлатона}: ускоренное расширение — чистая спектральная динамика.

\subsection{Вакуумная энергия и энтропийное подавление}
Эффективная плотность «вакуума» из нулевых энергий мод (с энтропийным весом):
\[
\boxed{\rho_{\rm vac}(S,a) = \frac{\hbar}{2 V(a)} \sum_k \tilde{\omega}_k \, F \big( \tilde{\omega}_k, S \big) \, \Theta \big( k_c(a) - k \big)} \tag{10}
\]
\begin{itemize}
    \item \(V(a) \propto a^3\) — объём;
    \item \(\frac{\hbar}{2} \tilde{\omega}_k\) — нулевая энергия моды;
    \item \(F(\tilde{\omega},S) \in [0,1]\) — \textbf{энтропийный фактор подавления} высоких частот при росте \(S\);
    \item \(\Theta\) — оконная функция с «скользящей» отсечкой \(k_c(a)\) (космологическая ко-модовость).
\end{itemize}

\textbf{Физика:} пока сумма слабо меняется \(\Rightarrow w = p/\rho \approx -1\) и инфляция идёт; по мере «выключения» подавляющих факторов инфляция останавливается, энергия перераспределяется в \textbf{локализованные моды} (нагрев).

\subsection{Спектральная размерность и поэтапное развёртывание 1D \(\to\) 3D}
\textbf{Спектральная размерность} \(d_s\) вводится через тепловой след (heat trace):
\[
K(s) = \mathrm{Tr} \, e^{-s \Lambda} \ \sim \ \frac{1}{(4\pi s)^{d_s/2}} \quad (s \to 0^+). \tag{11}
\]
\begin{itemize}
    \item При ZFST возможна стадия \(d_s \simeq 1\) (квазилинейные цепочки связей), затем потоком (5) сеть получает \textbf{три эквивалентных «направления»} связности \(\Rightarrow d_s \to 3\).
\end{itemize}

\textbf{Почему именно 3D + время?}  
(Гипотеза минимальности.) Конфигурации с \(d_s = 1\) нестабильны (слишком малые объёмы корреляций), \(d_s \ge 4\) — спектрально «дорогие» (много высоких мод без достаточного подавления). Минимум «спектрального действия» достигается при \textbf{трёх} почти равных ортогональных связях — т.е. 3D.

\textbf{Где «сидят» другие измерения?}  
В блоках \(\Lambda\) с \textbf{большими зазорами} (\(\lambda_{\rm compact} \gg \lambda_1\)) — их корреляционные длины микроскопичны, они остаются \textbf{компактифицированными}. Вклад в \(\rho_{\rm vac}\) от них подавлен \(F(\tilde{\omega},S)\), но они:
\begin{itemize}
    \item сдвигают калибровочные константы (через интегрирование высоких мод),
    \item вносят малые поправки к массам/смешиваниям,
    \item могут давать слабые «скрытые» взаимодействия.
\end{itemize}

\section{Массы, поколения и смешивания}

\subsection{«Лестница поколений»}
Эмпирически в каждом семействе видим три иерархических уровня. ZFSC моделирует это «лестницей»:
\[
\boxed{\mu = \{0, \varepsilon, c \varepsilon \}, \qquad
m_i^2 \propto \lambda_0 + \mu_i} \tag{12}
\]
\begin{itemize}
    \item \(\lambda_0 \ge 0\) — базовый сдвиг уровня (общий «фон» сектора);
    \item \(\varepsilon > 0\) — шаг;
    \item \(c > 1\) — \textbf{отношение иерархии} (ключевая характеристика семейства).
\end{itemize}

\textbf{Из двух масс \(\to c\).}  
Например, для лептонов (порядок \(e \to \mu \to \tau\)):
\[
c_\ell = \frac{m_\tau^2 - m_e^2}{m_\mu^2 - m_e^2} \approx 2.828 \times 10^2. \tag{13}
\]
Для нейтрино (в терминах разностей): \(c_\nu = \frac{\Delta m_{31}^2}{\Delta m_{21}^2} \approx 34\).

\subsection{Микромодель «поколений»: матрица \(B(\delta,r,\dots)\)}
Минимальная 3×3-версия:
\[
\boxed{
B(\delta,r;g_L) = \begin{pmatrix}
0 & g_L & 0 \\
g_L & \delta & r \\
0 & r & 0
\end{pmatrix}, \quad
\mathrm{spec}(B) = \left\{ 0, \ \frac{\delta \pm \sqrt{\delta^2 + 4(g_L^2 + r^2)}}{2} \right\}
} \tag{14}
\]
\begin{itemize}
    \item \(\delta\) — «центральный сдвиг» (асимметрия центрального узла);
    \item \(r\) — правый «плечевой» канал связи; \(g_L\) — левый канал.
\end{itemize}

Для отсортированных уровней \((\lambda_{\min} < \lambda_{\mathrm{mid}} < \lambda_{\max})\) и \(\lambda_{\mathrm{mid}} = 0\) (как в (14)) «лестничное» отношение
\[
\boxed{
c = \frac{\lambda_{\max} - \lambda_{\min}}{\lambda_{\mathrm{mid}} - \lambda_{\min}}
= \frac{2 \sqrt{\delta^2 + 4(g_L^2 + r^2)}}{\sqrt{\delta^2 + 4(g_L^2 + r^2)} - \delta}
} \tag{15}
\]
и в режиме большой \(\delta\):
\[
\boxed{c \approx \frac{\delta^2}{g_L^2 + r^2} + 2} \quad (\delta^2 \gg g_L^2 + r^2). \tag{16}
\]
\textbf{Смысл:} огромные иерархии \(c\) естественно получаются при большом центральном сдвиге \(\delta\) и узкой «горловине» связей (малые \(g_L, r\)).

\textbf{6×6 и асимметрии.} Практически мы используем расширенную 6×6-матрицу с рёбрами \(g_L, g_R\) и асимметриями \(h_{1,2,3}\), что позволяет:
\begin{itemize}
    \item поддержать разные иерархии в секторах (\(\nu, \ell, u, d\));
    \item вводить \textbf{общие параметры} (унификация) и проверять предсказательность.
\end{itemize}

\subsection{Предсказание лёгкой массы из двух тяжёлых}
Если лестница \(\{0, 1, c\}\) и мы идентифицируем \(\mu \to 1\), \(\tau \to c\), то
\[
s^2 = \frac{m_\tau^2 - m_\mu^2}{c - 1}, \qquad
\boxed{m_{\rm light}^2 = m_\mu^2 - s^2} \tag{17}
\]
\begin{itemize}
    \item \(s^2\) — общий «масштаб» сектора;
    \item \textbf{важно:} тут \(c\) — \textbf{предсказанный} моделью (из спектра \(B\)), а не вычисленный из трёх масс (иначе это тождество, а не предсказание).
\end{itemize}

\section{Гравитация и кривизна из спектра}

\subsection{Эвристика для \(G\)}
Суммарная «жёсткость» вакуума, складывающаяся из всех мод:
\[
\boxed{\frac{1}{G_{\rm eff}} \sim \sum_k \hbar \tilde{\omega}_k W_k} \tag{18}
\]
\begin{itemize}
    \item \(W_k\) — вес, зависящий от структуры мод и подавления (аналог «спектрального действия»).
\end{itemize}
Идея: чем больше высокочастотных мод задействовано (с учётом \(F(\tilde{\omega}, S)\)), тем больше «упругость» геометрии (меньше \(G\)).

\subsection{Вклад отдельной моды в кривизну}
В линейном режиме:
\[
\boxed{
\delta R_{\mu\nu}^{(k)} \simeq \frac{8\pi G}{c^4} T_{\mu\nu}^{(k)}, \qquad
T_{\mu\nu}^{(k)} \propto m_k u_\mu u_\nu
} \tag{19}
\]
\begin{itemize}
    \item \(u_\mu\) — 4-скорость носителя моды;
    \item \(m_k\) из (3); суммарно \(\sum_k \delta R_{\mu\nu}^{(k)}\) формирует наблюдаемую кривизну.
\end{itemize}
Это связывает \textbf{массы} и \textbf{кривизну} как две стороны одного спектрального «механизма» \(\Lambda\).

\section{Тёмная энергия и «почему она мала»}

\subsection{Формула вакуума и подавление}
Вернёмся к (10): малая \(\rho_\Lambda\) обеспечивается:
\begin{itemize}
    \item подавлением \(F(\tilde{\omega}, S)\) для «компактных» высоких мод (блоки с большими \(\lambda\));
    \item «скользящей» отсечкой \(k_c(a)\), уменьшающей вклад ультрафиолета при росте \(a\).
\end{itemize}

\textbf{Эффективное уравнение состояния.}
\[
w + 1 \simeq -\frac{d \ln \rho_{\rm vac}}{d \ln a} \simeq -\frac{d \ln F}{d \ln a} \quad (\text{малое}). \tag{20}
\]
Ожидается \(w \approx -1\) с крошечным дрейфом — космологически проверяемый след.

\section{Почему 1D \(\to\) 3D, а не другие измерения, и «где они сидят»}

\begin{enumerate}
    \item \textbf{Стадия 1D.} При самом начале ZFST сеть связи «тонкая», спектральный зазор \(\lambda_1\) велик, \(d_s \approx 1\). Масштаб \(a \propto 1/\sqrt{\lambda_1}\) растёт экспоненциально (9).
    \item \textbf{Развилка к 3D.} Минимум \(\mathcal{S}_{\rm spec}\) достигается при трёх почти равноправных «направлениях» связей (энтропийная эффективность): \(d_s \to 3\).
    \item \textbf{Почему не 4D?} Для \(d_s \ge 4\) характерный спектр \(\rho(\lambda)\) даёт слишком сильный ультрафиолет без достаточного подавления \(F\), что делает \(\rho_{\rm vac}\) нестабильной/слишком большой (эвристически: «дорого» в спектральном действии).
    \item \textbf{Остальные измерения} застревают в «компактных» блоках \(\Lambda\) с большими зазорами \(\lambda_{\rm compact}\):
    \begin{itemize}
        \item корреляционная длина \(\xi_{\rm compact} \sim 1/\sqrt{\lambda_{\rm compact}}\) микроскопична;
        \item их вклад в наблюдаемую динамику идёт через \textbf{ренормировку констант}, малые смещения масс/смешиваний и \(\rho_\Lambda\).
    \end{itemize}
\end{enumerate}

\section{Вычислительная программа и проверяемые следствия}

\subsection{Матрицы поколений и коэффициент \(c\)}
Мы используем матрицы \(B(\delta, r; g_L, g_R, h_{1,2,3})\) размера 3, 4 или 6. В простейшем 3×3 случае \(c\) задаётся (15)–(16); в 6×6 — численно по трём \textbf{фиксированным уровням} (важно не выбирать триплет под таргет, иначе возникает скрытая подгонка).

\textbf{Практическое правило (честность):}
\begin{itemize}
    \item выбираем одно правило триплета (напр., «три нижних уровня») и \textbf{не меняем его} между секторами;
    \item в унификационных режимах \(c\) \textbf{предсказывается}, а не подгоняется.
\end{itemize}

\subsection{Предсказание лёгких масс}
Для лептонов:
\[
m_e^{\rm pred} = \sqrt{m_\mu^2 - \frac{m_\tau^2 - m_\mu^2}{c_\ell^{\rm pred} - 1}}, \tag{21}
\]
где \(c_\ell^{\rm pred}\) извлечён из спектра \(B\) в том же \textbf{унификационном режиме}, что и для нейтрино, кварков и т.д. Аналогично можно строить предсказания для лёгких кварков (\(u,d\)) из \((c,s,t)\) или \((d,s,b)\).

\subsection{Инструментарий}
\begin{itemize}
    \item \textbf{Режимы:} \texttt{independent\_all}, \texttt{shared\_r\_all}, \texttt{shared\_delta\_all}, \texttt{full\_unify\_all}, \texttt{grand\_unify\_all}, \texttt{grand\_unify\_all\_scaled}.
    \item \textbf{Критерии «прорыва»:} в жёстких режимах (\texttt{full\_unify\_all}) одновременно
    \begin{itemize}
        \item \(z_\nu \lesssim 2\sigma\) (по \(c_\nu\)),
        \item \(z_e \lesssim 2\sigma\) (по \(m_e^{\rm pred}\) с 1\% модельной \(\sigma\)),
        \item и глобальный \(z \lesssim 2\sigma\).
    \end{itemize}
    \item \textbf{Технические замечания:}
    \begin{itemize}
        \item не использовать \(c_\ell^{\rm exp}\) при оптимизации, если цель — \textbf{предсказать} \(m_e\);
        \item выбор триплета уровней — \textbf{фиксированный} (например, «три нижних»);
        \item массы кварков сопоставлять при фиксированном \(\overline{\rm MS}\)-масштабе.
    \end{itemize}
\end{itemize}

\section{Набор уравнений ZFSC (минимальная «система» с комментариями)}

\begin{enumerate}
    \item \textbf{Собственные моды:} \(\Lambda \mathbf{v}_k = \lambda_k^{\rm eff} \mathbf{v}_k\).
    \item \textbf{Масса моды:} \(m_k = \frac{\hbar}{c^2} \sqrt{\lambda_k^{\rm eff}}\).
    \item \textbf{Смешивание:} \(U_{\alpha i} \sim \langle \alpha | \mathbf{v}_i \rangle\).
    \item \textbf{Лестница:} \(\mu = \{0, \varepsilon, c \varepsilon \}\), \(m_i^2 \propto \lambda_0 + \mu_i\).
    \item \textbf{Коэффициент иерархии (3×3):} \(c = \frac{2 \sqrt{\delta^2 + 4(g_L^2 + r^2)}}{\sqrt{\delta^2 + 4(g_L^2 + r^2)} - \delta} \simeq \frac{\delta^2}{g_L^2 + r^2} + 2\).
    \item \textbf{Предсказание лёгкой массы:} \(m_{\rm light}^2 = m_\mu^2 - (m_\tau^2 - m_\mu^2)/(c - 1)\).
    \item \textbf{Энтропийная динамика:} \(\frac{d \Lambda}{d \tau} = -\frac{\delta \mathcal{S}_{\rm spec}}{\delta \Lambda}\), \(t(\tau) = \int \zeta(S) \, d\tau\).
    \item \textbf{Зазор–масштабный фактор:} \(a \propto 1/\sqrt{\lambda_1}\), \(H = -\frac{1}{2} \frac{\dot{\lambda}_1}{\lambda_1}\).
    \item \textbf{Вакуумная энергия:} \(\rho_{\rm vac} = \frac{\hbar}{2 V} \sum_k \tilde{\omega}_k F(\tilde{\omega}_k, S) \Theta(k_c - k)\).
    \item \textbf{Гравитационная «жёсткость»:} \(G_{\rm eff}^{-1} \sim \sum \hbar \tilde{\omega}_k W_k\).
    \item \textbf{Линейная гравитация моды:} \(\delta R_{\mu\nu}^{(k)} \simeq \frac{8 \pi G}{c^4} T_{\mu\nu}^{(k)}\).
    \item \textbf{Спектральная размерность:} \(K(s) = \mathrm{Tr} \, e^{-s \Lambda} \sim (4 \pi s)^{-d_s/2}\).
\end{enumerate}

Каждый коэффициент:
\begin{itemize}
    \item \(\hbar, c\) — фундаментальные константы (масштабируют связь «частота \(\to\) масса»).
    \item \(\delta, r, g_L, g_R, h_{1,2,3}\) — \textbf{геометрия связей} в пред-геометрической сети (определяют форму спектра и, следовательно, \(c\), массы и смешивания).
    \item \(\varepsilon, \lambda_0\) — «шаг» и базовый сдвиг в лестничной аппроксимации уровня.
    \item \(F(\tilde{\omega}, S)\), \(k_c(a)\) — феноменологические подавления УФ-вклада (энтропия и масштаб), подлежащие калибровке.
    \item \(W_k\) — вес вклада мод в «жёсткость» геометрии (зависит от нормировки спектрального действия).
\end{itemize}

\section{Наблюдаемые следствия и тесты}

\begin{enumerate}
    \item \textbf{Нейтринные иерархии:} \(c_\nu\) крупный (\(\sim 34\)), устойчивый к деталям; диапазон \(m_{\beta\beta}\) для \(0\nu\beta\beta\) (мелкий \(\sim\) мЭв).
    \item \textbf{Лептоны:} предсказание \(m_e\) из \((\mu, \tau)\) при \textbf{общих} параметрах \(B\) с нейтрино (через shared-режимы).
    \item \textbf{Калибровочные константы:} через субструктуры \(\Lambda\) — возможность соотнести феноменологические константы с «средней связностью» подграфов (общая логика спектрального действия).
    \item \textbf{Инфляция:} малый тензорный сигнал \(r\) и слабый running, выражаемые через \(d \ln \lambda_1 / dt\).
    \item \textbf{Тёмная энергия:} \(w \approx -1\) с микродрейфом \(w + 1 \sim -d \ln F / d \ln a\).
    \item \textbf{Незаметные измерения:} отсутствие развёртывания прочих измерений проявляется как \textbf{малые, но коллективные} поправки к массам и константам.
\end{enumerate}

\section{Дорожная карта исследований}

\begin{itemize}
    \item \textbf{(А)} Зафиксировать архитектуру \(B\) (малое число параметров) и \textbf{одно правило} выбора триплета.
    \item \textbf{(B)} Калибровать \textbf{минимально} (например, \(\Delta m^2\) для \(\nu\) и \(\mu, \tau\) для \(\ell\)), \textbf{предсказывать} остальное (\(m_e\), \(m_{\beta\beta}\), углы PMNS/CKM, \(m_W/m_Z\)).
    \item \textbf{(C)} Считать \(\chi^2\), \textbf{z-уровни} по независимым наблюдаемым, \textbf{Global z} с учётом числа параметров.
    \item \textbf{(D)} Проверить стабильность к вариациям диапазонов/сеток (без «скрытой подгонки триплетом»).
    \item \textbf{(E)} Если нужно, \textbf{одна} новая ручка (например, слабая асимметрия) — и снова тест на предсказательность.
\end{itemize}

\section{Заключение}

ZFSC предлагает цельную картину, где \textbf{один спектр} \(\Lambda = L + M\) на нулевом вероятностном поле последовательно порождает:

\begin{itemize}
    \item \textbf{массы} (через \(\sqrt{\lambda_k^{\rm eff}}\)),
    \item \textbf{смешивания} (через собственные векторы),
    \item \textbf{гравитацию} (через суммарную «жёсткость» мод),
    \item \textbf{инфляцию} (через экспоненциальное падение спектрального зазора),
    \item \textbf{малую тёмную энергию} (энтропийное подавление нулевых мод),
    \item \textbf{3D-пространство + время} (из спектральной минимальности),
    \item и оставляет «прочие измерения» в компактных блоках \(\Lambda\), где они тонко влияют на константы.
\end{itemize}

\begin{verbatim}
@misc{prob_field_hypothesis_ru_2025,
  author       = {Евгений Монахов and ООО "VOSCOM ONLINE" Research Initiative},
  title        = {Спектральная космология нулевого поля. Теория.},
  year         = {2025},
  publisher    = {Zenodo},
  orcid        = {0009-0003-1773-5476},
  url_orcid    = {https://orcid.org/0009-0003-1773-5476},
  organization = {https://voscom.online/}
}
\end{verbatim}

\end{document}