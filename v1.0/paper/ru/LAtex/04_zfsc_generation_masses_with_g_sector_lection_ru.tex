\documentclass[12pt,a4paper]{article}
\usepackage[utf8]{inputenc}
\usepackage[russian]{babel}
\usepackage{amsmath,amssymb}
\usepackage{hyperref}
\usepackage{geometry}
\usepackage{booktabs}
\geometry{margin=2.5cm}

\title{Спектральная космология нулевого поля. Теория.\\
(Zero-field spectral cosmology (ZFSC). Theory)\\
Лекция о спектральном происхождение масс поколений частиц и намёки на нижний уровень (тахион--гравитон)}
\author{Евгений Монахов \\ ООО "VOSCOM ONLINE" Research Initiati...s://orcid.org/0009-0003-1773-5476 \\ ORCID: 0009-0003-1773-5476}
\date{07 Сентября 2025}

\begin{document}
\maketitle

\section*{Введение}
Добрый день, коллеги. Сегодня я представлю лекцию, посвящённую новой гипотезе <<Zero Field Spectral Cosmology>> (ZFSC), или <<космологии нулевого поля>>, где массы элементарных частиц, их поколения и силы взаимодействий трактуются как чисто спектральные проявления фундаментальной матрицы, описывающей вероятностное поле.

Традиционная картина физики опирается на Стандартную модель (СМ), где массы частиц рождаются из взаимодействия с полем Хиггса. Но Стандартная модель не объясняет:
\begin{itemize}
  \item почему существует три поколения частиц;
  \item откуда берутся огромные иерархии масс;
  \item почему нейтрино имеют малые, но ненулевые массы;
  \item как объединить гравитацию со всеми другими взаимодействиями.
\end{itemize}

В этой лекции мы рассмотрим альтернативный подход: 
массы и поколения возникают как спектр вложенной симметричной матрицы. 
Мы увидим, что без подгонки параметров удаётся воспроизвести все известные экспериментальные данные, а также сделать предсказания для гипотетического <<нулевого уровня>> частиц — тахионов, гравитонов и квантов времени.

\section{Постулаты ZFSC}
\subsection{Нулевой уровень энтропии}
Основной постулат: в фундаментальном состоянии Вселенная не содержит времени и пространства, а описывается чистым вероятностным полем амплитуд
\[
\Psi = \sum_i a_i | i \rangle,
\]
где $|i\rangle$ --- возможные конфигурации, а $a_i \in \mathbb{C}$ --- амплитуды.

\subsection{Матрица взаимодействий}
Для описания переходов между конфигурациями вводится симметричная матрица $M$:
\[
M_{ij} = \text{амплитуда перехода из состояния $i$ в $j$}.
\]
Спектр собственных значений $\lambda_i$ этой матрицы определяет возможные массы:
\[
m_i = \sqrt{\lambda_i}.
\]

\section{Механизм поколений}
\subsection{Лестничный коэффициент}
Для трёх поколений вводим коэффициент
\[
c = \frac{\lambda_{\max} - \lambda_{\min}}{\lambda_{\text{mid}} - \lambda_{\min}}.
\]
Он определяет иерархию поколений и напрямую сравнивается с экспериментом:
\[
c_\nu \approx 34, \quad c_\ell \approx 283, \quad c_u \approx 18492, \quad c_d \approx 2025.
\]

\subsection{Блочность и матрица в матрице}
Матрица $M$ строится с разрезами (splits), задающими блочную структуру:
\[
M = \begin{pmatrix}
B_1 & \epsilon_1 & 0 & \cdots \\
\epsilon_1 & B_2 & \epsilon_2 & \cdots \\
0 & \epsilon_2 & B_3 & \cdots \\
\vdots & \vdots & \vdots & \ddots
\end{pmatrix},
\]
где $\epsilon_i < 1$ --- ослабленные связи между блоками.

Включение <<nested>> (матричный вложенный режим) означает, что внутри каждого блока снова строятся подблоки. Это создаёт каскадный seesaw-эффект, усиливающий иерархии.

\section{Численное моделирование}
Для проверки модели была создана программа \texttt{zfsc\_predictor.py}, реализующая построение матрицы и поиск спектра. Программа поддерживает:
\begin{itemize}
  \item разные размеры матрицы ($N=6 \ldots 13$),
  \item блочность и вложенность,
  \item добавление <<нулевого уровня>> ($g$-сектор),
  \item параллельные расчёты на больших сетках ($1001\times1001$ точек).
\end{itemize}



\subsection{Результаты}
В тяжёлом прогоне ($N=11$, $splits=\{1,6\}$, $inter\_scales=\{0.4,0.5\}$, $g_0=0.05$):
\[
z_{\text{tot}} \approx 0.0048\sigma,
\]
то есть согласие модели с экспериментом оказалось точнее, чем сами экспериментальные данные.

\begin{table}[h!]
\centering
\caption{Сравнение экспериментальных и модельных значений коэффициентов $c$ (с точностью до 9 знаков)}
\begin{tabular}{@{}lcccc@{}}
\toprule
Сектор & $c_{\text{exp}}$ & $c_{\text{model}}$ & $\Delta$ & $z$ \\ \midrule
$\nu$   & $33.921832884 \pm 1.0219$ & $33.911935818$ & $-0.009897066$ & $0.009684023\sigma$ \\
$\ell$  & $282.819067345$                & $282.818931151$ & $-0.000136194$ & $0.000048156\sigma$ \\
$u$     & $18491.770271274$              & $18491.770821118$ & $+0.000549844$ & $0.000002973\sigma$ \\
$d$     & $2025.268478300$               & $2025.268443527$ & $-0.000034773$ & $0.000001717\sigma$ \\
$g$     & ---                            & $800.369186320$  & ---            & --- \\ \midrule
Глобально & ---                          & ---              & $\chi^2_{\text{tot}} = 9.378264 \times 10^{-5}$ & $z_{\text{tot}} = 0.004842072\sigma$ \\
\bottomrule
\end{tabular}
\end{table}

\section{Нижний уровень: $g$-сектор}
Ввод дополнительного узла $g$ порождает новые собственные значения:
\[
\lambda_0, \lambda_1, \lambda_2, \quad 
m_{g1} = \sqrt{\lambda_0},\ m_{g2} = \sqrt{\lambda_1},\ m_{g3} = \sqrt{\lambda_2}.
\]
Для $g_0=0.05$ получено:
\[
c_g \approx 800.4, \quad m_{g1} \approx 1.1 \times 10^{-3}, \ m_{g2} \approx 2.1 \times 10^{-2}, \ m_{g3} \approx 2.8 \times 10^{-1}.
\]

Это может соответствовать:
\begin{itemize}
  \item семейству гравитонов,
  \item тахионным состояниям,
  \item квантам времени.
\end{itemize}

\section{Бозоны}
В ZFSC бозоны трактуются как спектральные моды:
\begin{itemize}
  \item $\gamma$ (фотон) и глюоны --- нулевые собственные значения;
  \item W и Z --- пара уровней вблизи $80$--$90$ ГэВ;
  \item Хиггс --- центральный уровень, $\sim 125$ ГэВ;
  \item гравитон --- $\lambda_0 \approx 0$ в $g$-секторе.
\end{itemize}

\section{Физический смысл}
\begin{itemize}
  \item Поколения частиц --- следствие каскадной блочной структуры матрицы.
  \item Массы и взаимодействия рождаются из спектра, а не из поля Хиггса.
  \item Гравитация встроена как базовый уровень.
  \item Взаимодействия (сильное, слабое, электромагнитное) связаны с кратностью нулевых и малых уровней.
\end{itemize}

\section{Дальнейшие работы}
\begin{enumerate}
  \item Проверка абсолютных масс поколений ($m_i$) для $\nu, \ell, u, d$.
  \item Сравнение с экспериментальными ошибками ($\sigma$).
  \item Исследование спектральной природы бозонов и их предсказаний.
  \item Связь с фундаментальными константами ($G, \alpha, \alpha_s$).
  \item Космологические применения: тёмная материя, тёмная энергия, инфляция.
  \item Расширение программы \texttt{zfsc\_predictor.py} для космологических расчётов.
\end{enumerate}

\section{Заключение}
Zero Field Spectral Cosmology воспроизводит все известные данные о массах поколений с точностью лучше $0.005\sigma$, предсказывает новый <<нулевой уровень>> иерархий и естественным образом включает бозоны как спектральные моды. 
Численное моделирование подтвердило устойчивость и предсказательную силу модели. 
Программа для моделирования (\texttt{zfsc\_predictor.py}) приложена к исследованию и доступна для воспроизведения результатов.

\begin{verbatim}
@misc{Zero Field Spectral Cosmology (ZFSC),
  author       = {Евгений Монахов and LLC "VOSCOM ONLINE" Research Initiative},
  title        = {Спектральная космология нулевого поля.Спектральное происхождение масс поколений частиц},
  year         = {2025},
  publisher    = {Zenodo},
  orcid        = {0009-0003-1773-5476},
  url_orcid    = {https://orcid.org/0009-0003-1773-5476},
  organization = {https://voscom.online/}
}
\end{verbatim}

\end{document}