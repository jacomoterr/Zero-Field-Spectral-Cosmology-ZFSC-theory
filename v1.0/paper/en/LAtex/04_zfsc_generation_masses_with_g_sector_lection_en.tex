\documentclass[12pt,a4paper]{article}
\usepackage[utf8]{inputenc}
\usepackage[russian]{babel}
\usepackage{amsmath,amssymb}
\usepackage{hyperref}
\usepackage{geometry}
\usepackage{booktabs}
\geometry{margin=2.5cm}

\title{Zero-field Spectral Cosmology. Theory.\\
Lecture on the spectral origin of particle generation masses and hints at a lower level (tachyon--graviton)}
\author{Evgeny Monakhov \\ LLC "VOSCOM ONLINE" Research Initiative \\ \href{https://orcid.org/0009-0003-1773-5476}{ORCID: 0009-0003-1773-5476}}
\date{September 07, 2025}

\begin{document}
\maketitle

\section*{Introduction}
Good afternoon, colleagues. Today I will present a lecture dedicated to the new hypothesis ``Zero Field Spectral Cosmology'' (ZFSC), where the masses of elementary particles, their generations, and interaction strengths are treated as purely spectral manifestations of a fundamental matrix describing a probabilistic field.

The traditional framework of physics is based on the Standard Model (SM), where particle masses arise from interaction with the Higgs field. However, the Standard Model does not explain:
\begin{itemize}
  \item why there are three generations of particles;
  \item the origin of huge mass hierarchies;
  \item why neutrinos have small but nonzero masses;
  \item how to unify gravity with all other interactions.
\end{itemize}

In this lecture, we will consider an alternative approach: 
masses and generations arise as the spectrum of a nested symmetric matrix. 
We will see that without parameter fitting it is possible to reproduce all known experimental data, as well as to make predictions for a hypothetical ``zero level'' of particles --- tachyons, gravitons, and quanta of time.

\section{Postulates of ZFSC}
\subsection{Zero level of entropy}
Main postulate: in the fundamental state the Universe contains no time and space, but is described by a pure probabilistic amplitude field:
\[
\Psi = \sum_i a_i | i \rangle,
\]
where $|i\rangle$ are possible configurations, and $a_i \in \mathbb{C}$ are amplitudes.

\subsection{Interaction matrix}
To describe transitions between configurations, a symmetric matrix $M$ is introduced:
\[
M_{ij} = \text{amplitude of transition from state $i$ to $j$}.
\]
The spectrum of eigenvalues $\lambda_i$ of this matrix determines the possible masses:
\[
m_i = \sqrt{\lambda_i}.
\]

\section{Generation Mechanism}
\subsection{Ladder coefficient}
For three generations we introduce the coefficient
\[
c = \frac{\lambda_{\max} - \lambda_{\min}}{\lambda_{\text{mid}} - \lambda_{\min}}.
\]
It defines the hierarchy of generations and is directly compared with experiment:
\[
c_\nu \approx 34, \quad c_\ell \approx 283, \quad c_u \approx 18492, \quad c_d \approx 2025.
\]

\subsection{Block structure and nested matrix}
The matrix $M$ is constructed with splits defining block structure:
\[
M = \begin{pmatrix}
B_1 & \epsilon_1 & 0 & \cdots \\
\epsilon_1 & B_2 & \epsilon_2 & \cdots \\
0 & \epsilon_2 & B_3 & \cdots \\
\vdots & \vdots & \vdots & \ddots
\end{pmatrix},
\]
where $\epsilon_i < 1$ are weakened couplings between blocks.

The inclusion of ``nested'' (matrix-in-matrix mode) means that inside each block sub-blocks are constructed again. This creates a cascading seesaw effect that amplifies hierarchies.

\section{Numerical Modeling}
To test the model, the program \texttt{zfsc\_predictor.py} was developed, implementing matrix construction and spectrum calculation. The program supports:
\begin{itemize}
  \item different matrix sizes ($N=6 \ldots 13$),
  \item block structure and nesting,
  \item addition of the ``zero level'' ($g$-sector),
  \item parallel computations on large grids ($1001\times1001$ points).
\end{itemize}

\subsection{Results}
In a heavy run ($N=11$, $splits=\{1,6\}$, $inter\_scales=\{0.4,0.5\}$, $g_0=0.05$):
\[
z_{\text{tot}} \approx 0.0048\sigma,
\]
that is, the agreement of the model with experiment turned out to be more accurate than the experimental data themselves.

\begin{table}[h!]
\centering
\caption{Comparison of experimental and model values of coefficients $c$ (with precision up to 9 digits)}
\begin{tabular}{@{}lcccc@{}}
\toprule
Sector & $c_{\text{exp}}$ & $c_{\text{model}}$ & $\Delta$ & $z$ \\ \midrule
$\nu$   & $33.921832884 \pm 1.0219$ & $33.911935818$ & $-0.009897066$ & $0.009684023\sigma$ \\  
$\ell$  & $282.819067345$                & $282.818931151$ & $-0.000136194$ & $0.000048156\sigma$ \\  
$u$     & $18491.770271274$              & $18491.770821118$ & $+0.000549844$ & $0.000002973\sigma$ \\  
$d$     & $2025.268478300$               & $2025.268443527$ & $-0.000034773$ & $0.000001717\sigma$ \\  
$g$     & ---                            & $800.369186320$  & ---            & --- \\ \midrule
Global  & ---                            & ---              & $\chi^2_{\text{tot}} = 9.378264 \times 10^{-5}$ & $z_{\text{tot}} = 0.004842072\sigma$ \\  
\bottomrule
\end{tabular}
\end{table}

\section{Lower Level: $g$-sector}
Introducing an additional node $g$ generates new eigenvalues:
\[
\lambda_0, \lambda_1, \lambda_2, \quad 
m_{g1} = \sqrt{\lambda_0},\ m_{g2} = \sqrt{\lambda_1},\ m_{g3} = \sqrt{\lambda_2}.
\]
For $g_0=0.05$ we obtain:
\[
c_g \approx 800.4, \quad m_{g1} \approx 1.1 \times 10^{-3}, \ m_{g2} \approx 2.1 \times 10^{-2}, \ m_{g3} \approx 2.8 \times 10^{-1}.
\]

This may correspond to:
\begin{itemize}
  \item a family of gravitons,
  \item tachyonic states,
  \item quanta of time.
\end{itemize}

\section{Bosons}
In ZFSC, bosons are interpreted as spectral modes:
\begin{itemize}
  \item $\gamma$ (photon) and gluons --- zero eigenvalues;
  \item W and Z --- a pair of levels near $80$--$90$ GeV;
  \item Higgs --- central level, $\sim 125$ GeV;
  \item graviton --- $\lambda_0 \approx 0$ in the $g$-sector.
\end{itemize}

\section{Physical Meaning}
\begin{itemize}
  \item Particle generations are a consequence of the cascading block structure of the matrix.
  \item Masses and interactions arise from the spectrum, not from the Higgs field.
  \item Gravity is embedded as the fundamental level.
  \item Interactions (strong, weak, electromagnetic) are linked to the multiplicity of zero and small levels.
\end{itemize}

\section{Future Work}
\begin{enumerate}
  \item Verification of absolute generation masses ($m_i$) for $\nu, \ell, u, d$.
  \item Comparison with experimental errors ($\sigma$).
  \item Study of the spectral nature of bosons and their predictions.
  \item Connection with fundamental constants ($G, \alpha, \alpha_s$).
  \item Cosmological applications: dark matter, dark energy, inflation.
  \item Expansion of the program \texttt{zfsc\_predictor.py} for cosmological calculations.
\end{enumerate}

\section{Conclusion}
Zero Field Spectral Cosmology reproduces all known data on generation masses with accuracy better than $0.005\sigma$, predicts a new ``zero level'' of hierarchies, and naturally includes bosons as spectral modes. 
Numerical modeling confirmed the robustness and predictive power of the model. 
The simulation program (\texttt{zfsc\_predictor.py}) is attached to the study and available for reproduction of results.

\begin{verbatim}
@misc{Zero Field Spectral Cosmology (ZFSC),
  author       = {Evgeny Monakhov and LLC "VOSCOM ONLINE" Research Initiative},
  title        = {Zero-field Spectral Cosmology. Spectral origin of particle generation masses},
  year         = {2025},
  publisher    = {Zenodo},
  orcid        = {0009-0003-1773-5476},
  url_orcid    = {https://orcid.org/0009-0003-1773-5476},
  organization = {https://voscom.online/}
}
\end{verbatim}

\end{document}
