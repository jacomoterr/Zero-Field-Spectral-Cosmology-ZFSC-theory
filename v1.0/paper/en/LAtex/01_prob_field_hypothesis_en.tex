\documentclass[12pt,a4paper]{article}
\usepackage[utf8]{inputenc}
\usepackage{amsmath,amssymb}
\usepackage{hyperref}
\usepackage{geometry}
\usepackage{verbatim} % for citation block
\geometry{margin=2.5cm}

\title{The Probabilistic Field Hypothesis:\\
From Pre-Spacetime States to Emergent Physics}
\author{Evgeny Monakhov \\ LCC "VOSCOM ONLINE" Research Initiative \\ https://orcid.org/0009-0003-1773-5476}
\date{September 2025}

\begin{document}
\maketitle

\begin{abstract}
We propose a hypothesis that physical spacetime and interactions emerge from a deeper probabilistic field existing at zero entropy level. In this state there is no space or time, only amplitudes and probability fields representing potential configurations of all possible energies and interactions. We outline the basic postulates, present preliminary mathematical formulations, and sketch a research program aimed at connecting this framework with known physical laws and constants.
\end{abstract}

%------------------------------------
\section{Postulate 1: Zero-Entropy State}
We posit the existence of a fundamental level with no time and no space, characterized by vanishing entropy:
\[
S \to 0.
\]
In this regime, the universe is described by a pure probabilistic field of amplitudes:
\[
\Psi = \sum_{i} a_i |i\rangle ,
\]
where $\{|i\rangle\}$ are potential configurations (spaces, energies, interactions) and $a_i \in \mathbb{C}$ are their amplitudes.

%------------------------------------
\section{Postulate 2: Energy as Pure Potential}
Energy exists in this zero-entropy regime only as potential, not as realized dynamics. 
\[
E = \frac{1}{2}k |u|^2 ,
\]
with $u$ being a displacement in the probabilistic field and $k$ a universal stiffness coefficient.

Quantized form:
\[
E_n = \hbar \omega \left(n + \tfrac{1}{2}\right),
\]
where $\omega$ arises from the structure of the probability field rather than from classical geometry.

%------------------------------------
\section{Postulate 3: Emergence of Spacetime}
The appearance of time and space is modeled as a decoherence process:
\[
\Psi \;\xrightarrow{\text{decoherence}}\; \rho(x,t) = |\Psi(x,t)|^2 .
\]
Thus, spacetime coordinates $(x,t)$ are emergent from the underlying amplitude structure.

%------------------------------------
\section{Postulate 4: Compact and Extended Modes}
Not all modes unfold into macroscopic spacetime. Energy is distributed over dimensions as
\[
E = \sum_{D=0}^{\infty} \sum_{n_D} \hbar \omega_D \left(n_D + \epsilon_D\right),
\]
with $\omega_D$ denoting eigenfrequencies of $D$-dimensional modes and $\epsilon_D$ the zero-point shifts.
Modes with large $\omega_D$ remain compactified (e.g.\ Calabi--Yau), while modes with small $\omega_D$ extend into the observable macroscopic dimensions.

%------------------------------------
\section{Postulate 5: Entropy and Dimensional Dynamics}
The growth of entropy corresponds to the unfolding of dimensions:
\begin{itemize}
  \item At $S=0$ all dimensions exist only as potential modes.
  \item For $S>0$ a subset of modes decoheres into extended spacetime.
\end{itemize}

%------------------------------------
\section{Research Program}
\subsection*{Stage 1. Mathematical Formalization}
\begin{itemize}
  \item Build explicit models of $\Psi$ as a non-coordinate probability field.
  \item Introduce spectrum $\{\omega_D\}$ as universal modes.
  \item Relate $\omega_D$ to fundamental constants $\hbar, c, G, k_B$.
\end{itemize}

\subsection*{Stage 2. Reduction to Known Laws}
\begin{itemize}
  \item Demonstrate that projection to $3+1$ dimensions reproduces
        $E = mc^2$, Schrödinger equation, and Einstein equations.
  \item Renormalize coefficients $k, \omega_D$ to Planck units.
\end{itemize}

\subsection*{Stage 3. Numerical Experiments}
\begin{itemize}
  \item Simulate simple probabilistic fields (two or three superposed modes).
  \item Track which modes unfold into ``space'' with increasing entropy.
  \item Compare to known compactification patterns of Calabi--Yau manifolds.
\end{itemize}

\subsection*{Stage 4. Standard Model Connection}
\begin{itemize}
  \item Attempt to express interaction constants (electroweak, strong, gravity)
        in terms of $\omega_D$.
  \item Explore approximations of their observed magnitudes after renormalization.
\end{itemize}

\subsection*{Stage 5. Empirical Signatures}
\begin{itemize}
  \item Search for imprints of compactified modes in the CMB spectrum or dark energy distribution.
  \item Hypothesis: fluctuations from compactified dimensions may appear as noise or anomalies in the observed spectrum.
\end{itemize}

%------------------------------------
\section{Conclusion}
We have outlined a framework where the universe originates from a probabilistic field at zero entropy, with spacetime, energy, and interactions emerging via decoherence and unfolding of modes. This proposal suggests a new path: not from space to wave, but from wave to space. Future research will focus on renormalizing the framework to reproduce known physics and exploring potential experimental signatures.

%------------------------------------
\section*{Citation (BibTeX)}
\begin{verbatim}
@misc{prob_field_hypothesis_2025,
  author       = {Evgeny Monakhov and LCC "VOSCOM ONLINE" Research Initiative},
  title        = {The Probabilistic Field Hypothesis: From Pre-Spacetime States to Emergent Physics},
  year         = {2025},
  publisher    = {Zenodo},
  orcid        = {0009-0003-1773-5476},
  url_orcid    = {https://orcid.org/0009-0003-1773-5476},
  organization = {https://voscom.online/}
}
\end{verbatim}

\end{document}
