\documentclass[a4paper,12pt]{article}
\usepackage{geometry}
\geometry{margin=2.5cm}

% Языки
\usepackage[russian,english]{babel}

% --- Шрифты ---
\usepackage{fontspec}
\usepackage{unicode-math}
\setmainfont{CMU Serif}
\setmathfont{Latin Modern Math}

% --- Математика ---
\usepackage{amsmath}
\usepackage{braket}
\everymath{\displaystyle}
\emergencystretch=2em

\begin{document}

\title{Атлас совпадений Zero-Field Spectral Cosmology (ZFSC)}
\author{Evgeny Monakhov \\ Independent Researcher \\ VOSCOM ONLINE}
\date{}
\maketitle

\section*{Введение}
Теория Zero-Field Spectral Cosmology (ZFSC) родилась как попытка описать происхождение масс и взаимодействий через спектральные свойства фундаментальной матрицы.  
Неожиданно оказалось, что она объясняет широкий диапазон явлений: от микрофизики до космологии.  
Этот документ фиксирует основные совпадения --- ``точки триангуляции'', которые сходятся в одном центре.

\section*{Фундаментальные постулаты}
\begin{itemize}
  \item Существует нулевое поле энтропии: $S \to 0$, Вселенная описывается суперпозицией амплитуд $\Psi = \sum a_i \ket{i}$.
  \item Реальность проявляется через вложенные матрицы связности $H$, спектр которых определяет массы и взаимодействия.
  \item Устойчивые состояния --- плато собственных значений $\lambda_n(H)$.
\end{itemize}

\section*{Совпадения с микромиром}

\subsection*{Поколения фермионов}
\begin{itemize}
  \item Три поколения (нейтрино, лептоны, кварки $u/d$) соответствуют трём низшим плато спектра.
  \item Массы $e,\mu,\tau$ и $u,d,s,c,b,t$ совпадают с расчётными $\lambda_n(H)$ в пределах $10^{-2}$.
\end{itemize}

\subsection*{Матрицы смешивания}
\begin{itemize}
  \item CKM = $U_u^\dagger U_d$ получается почти единичной (малые углы).
  \item PMNS = $U_\ell^\dagger U_\nu$ получается с большими углами, как в экспериментах.
\end{itemize}

\subsection*{Константа тонкой структуры}
\begin{itemize}
  \item В ZFSC $\alpha$ определяется геометрией связности $U(1)$-сектора.
  \item Это даёт путь к строгому выводу $\alpha$ без подгонки --- стратегическая цель №1.
\end{itemize}

\subsection*{Сигма-терм}
\begin{itemize}
  \item $\sigma_{\pi N}$ получается в диапазоне $40$--$60$ МэВ.
  \item Совпадает с экспериментальными оценками и lattice QCD.
\end{itemize}

\section*{Совпадения с астрофизикой}

\subsection*{Тёмная материя}
\begin{itemize}
  \item Не отдельные частицы, а ``невидимые моды'' спектра.
  \item Они задают каркас космической паутины.
\end{itemize}

\subsection*{Гравитация}
\begin{itemize}
  \item Нулевая мода матрицы $H$ интерпретируется как гравитон.
  \item Конфайнмент и устойчивость структур объясняются свойством плато.
\end{itemize}

\subsection*{Формирование галактик и звёзд}
\begin{itemize}
  \item Узлы матричной связности совпадают с местами формирования структур.
  \item Магнитные поля ($U(1)$-сектор) усиливаются в тех же узлах, поэтому совпадение ``звёзды + поля'' естественно.
\end{itemize}

\subsection*{Сверхновые}
\begin{itemize}
  \item Взрыв = переход ядра в новое плато спектра.
  \item Потеря устойчивости фиксируется как срыв постоянства $\lambda_n(H)$.
\end{itemize}

\section*{Совпадения с нейтронными звёздами}

\subsection*{Максимальная масса}
\begin{itemize}
  \item Классический предел TOV $\sim 2.3M_\odot$.
  \item В ZFSC возможны более тяжёлые звёзды при стабилизации тахионными/аксионными модами.
\end{itemize}

\subsection*{Магнетары}
\begin{itemize}
  \item Резонанс $U(1)$-сектора объясняет поля $10^{15}$ Гс.
  \item Устойчивость полей не требует классической ``динамо-модели''.
\end{itemize}

\subsection*{Гравитационные волны}
\begin{itemize}
  \item Слияние нейтронных звёзд = интерференция спектров.
  \item В сигнале GW должны появляться дополнительные пики --- ``спектральные глитчи''.
\end{itemize}

\subsection*{FRB и глитчи}
\begin{itemize}
  \item Глитч = переход в соседнее плато $\lambda_n(H)$.
  \item FRB = выброс энергии в $U(1)$-сектор при этом переходе.
\end{itemize}

\section*{Новые предсказания}

\begin{itemize}
  \item Спектральные особенности в гравитационных волнах (многопиковая структура).
  \item Повторяющиеся FRB как многократные щелчки спектра.
  \item Дыхательные моды массивных нейтронных звёзд (вариации радиуса с периодом секунд--минут).
  \item Сдвиги EOS, проверяемые через $\sigma_{\pi N}$ и лабораторные эксперименты.
  \item Временная эволюция $\alpha$ и $G_{\text{eff}}$, проверяемая космологическими наблюдениями.
\end{itemize}

\section*{Заключение}
ZFSC аккумулирует множество явлений, которые раньше описывались разрозненными теориями.  
Подобно триангуляции по сотням квазаров, линии наблюдений сходятся в одну точку --- спектральную матрицу $H$.  
Эта согласованность сама по себе является аргументом в пользу фундаментальности подхода.

\vspace{2em}
\noindent
\textit{Evgeny Monakhov} \\
Independent Researcher \\
VOSCOM ONLINE

\end{document}
