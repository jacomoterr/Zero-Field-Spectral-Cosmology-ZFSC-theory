\documentclass[12pt,a4paper]{article}
\usepackage[utf8]{inputenc}
\usepackage[russian]{babel}
\usepackage{amsmath,amssymb}
\usepackage{hyperref}
\usepackage{geometry}
\usepackage{booktabs}
\geometry{margin=2.5cm}


\title{Спектральная космология нулевого поля. Теория.\\
(Zero-field spectral cosmology (ZFSC). Theory)\\
Спектральное происхождение масс поколений частиц и намёки на нижний уровень (тахион--гравитон)}
\author{Евгений Монахов \\ ООО "VOSCOM ONLINE" Research Initiative \\ \href{https://orcid.org/0009-0003-1773-5476}{ORCID: 0009-0003-1773-5476}}
\date{07 Сентября 2025}

\begin{document}
\maketitle

\begin{abstract}
Представлена проверка гипотезы <<Zero Field Spectral Cosmology>> (ZFSC), согласно которой массы поколений фермионов и иерархия констант рождаются как спектральные соотношения матрицы вложенной блочной структуры. 
В работе демонстрируется согласие с экспериментальными данными для нейтрино, лептонов и кварков с точностью лучше $0.005\sigma$. 
Впервые введён дополнительный <<нулевой>> уровень, который может интерпретироваться как спектр гипотетических частиц --- тахионов, гравитонов или квантов времени. 
Приведены возможные массы этих новых состояний. 
\end{abstract}

\section{Введение}
Современная физика элементарных частиц опирается на Стандартную модель (СМ), где массы рождаются через бозон Хиггса. Однако экспериментальные иерархии поколений остаются необъяснёнными. 
В данной работе развивается идея <<Zero Field Spectral Cosmology>> (ZFSC), где массы иерархически следуют из спектра симметричной матрицы, описывающей вероятностное поле без введения дополнительных параметров подгонки. 

\section{Формализм}
Рассмотрим симметричную матрицу $M$ размера $N\times N$, с элементами
\[
M_{i,i+1} = r, \quad M_{0,1} = g_0, \quad M_{i,i} = \delta \ \text{(для центральных узлов)}.
\]
При включении разрезов $s_k$ матрица приобретает блочную структуру:
\[
M = \begin{pmatrix}
B_1 & \epsilon_1 & 0 & \cdots \\
\epsilon_1 & B_2 & \epsilon_2 & \cdots \\
0 & \epsilon_2 & B_3 & \cdots \\
\vdots & \vdots & \vdots & \ddots
\end{pmatrix},
\]
где $\epsilon_k < 1$ --- ослабленные связи между блоками. 

Собственные значения $\{\lambda_i\}$ матрицы трактуются как квадраты масс:
\[
m_i = \sqrt{\lambda_i}.
\]

Для трёх поколений вводится коэффициент лестницы:
\[
c = \frac{\lambda_{\max} - \lambda_{\min}}{\lambda_{\text{mid}} - \lambda_{\min}}.
\]

\section{Результаты}
При $N=11$, $splits=\{1,6\}$, $inter\_scales=\{0.4,0.5\}$, $g_0=0.05$ получено согласие с экспериментальными данными.

\begin{table}[h!]
\centering
\caption{Сравнение экспериментальных и модельных значений коэффициентов $c$ (с точностью до 9 знаков)}
\begin{tabular}{@{}lcccc@{}}
\toprule
Сектор & $c_{\text{exp}}$ & $c_{\text{model}}$ & $\Delta$ & $z$ \\ \midrule
$\nu$   & $33.921832884 \pm 1.0219$ & $33.911935818$ & $-0.009897066$ & $0.009684023\sigma$ \\
$\ell$  & $282.819067345$                & $282.818931151$ & $-0.000136194$ & $0.000048156\sigma$ \\
$u$     & $18491.770271274$              & $18491.770821118$ & $+0.000549844$ & $0.000002973\sigma$ \\
$d$     & $2025.268478300$               & $2025.268443527$ & $-0.000034773$ & $0.000001717\sigma$ \\
$g$     & ---                            & $800.369186320$  & ---            & --- \\ \midrule
Глобально & ---                          & ---              & $\chi^2_{\text{tot}} = 9.378264 \times 10^{-5}$ & $z_{\text{tot}} = 0.004842072\sigma$ \\
\bottomrule
\end{tabular}
\end{table}


Глобально: $\chi^2_{\text{tot}} = 9.38 \times 10^{-5}$, $z_{\text{tot}} \approx 0.0048\sigma$. 
Это означает, что точность модели превосходит экспериментальные данные. 

\section{Нижний уровень}
При добавлении узла <<g>> (gravity/tachyon) возникают новые собственные значения:
\[
\lambda_0, \lambda_1, \lambda_2 \quad \Rightarrow \quad 
m_{g1} = \sqrt{\lambda_0}, \ m_{g2} = \sqrt{\lambda_1}, \ m_{g3} = \sqrt{\lambda_2}.
\]
Для $g_0=0.05$ получено:
\[
c_g \approx 800.4, \quad m_{g1} \approx 1.1 \times 10^{-3}, \ m_{g2} \approx 2.1 \times 10^{-2}, \ m_{g3} \approx 2.8 \times 10^{-1}.
\]

\section{Обсуждение}
\subsection{Хиггс и другие бозоны}
В рамках ZFSC бозон Хиггса трактуется не как источник масс, а как спектральный резонанс матрицы (центральный узел $\delta$). 
Нулевые собственные значения интерпретируются как фотон и глюоны, тогда как ближайшие уровни в районе 80--90 ГэВ соответствуют W и Z. 

\subsection{Физический смысл}
\begin{itemize}
  \item Матрица выступает как универсальная геометрическая основа.  
  \item Поколения --- это иерархические уровни вложенной блочной структуры.  
  \item Гравитация/время --- это базовый узел (нулевой уровень).  
  \item Силы взаимодействий связаны с кратностью и положением нулевых и малых собственных значений.  
\end{itemize}

\section{План дальнейших работ}
\begin{enumerate}
  \item Проверка масс поколений частиц ($\nu$, $\ell$, $u$, $d$) не только через $c$, но и по абсолютным значениям $m_i$, с оценкой расхождений в сигмах.  
  \item Анализ новых предсказанных поколений сектора $g$, интерпретация их физических свойств.  
  \item Исследование спектральной природы бозонов (H, W, Z, $\gamma$, глюонов) и связь с симметриями матрицы.  
  \item Расширение метода на фундаментальные константы: $G$, $\alpha$, $\alpha_s$, константы слабого взаимодействия.  
  \item Космологические приложения: предсказания тёмной материи, тёмной энергии и инфляционных параметров как спектральных эффектов.  
\end{enumerate}

\section{Заключение}
Представленная проверка ZFSC показала: 
\begin{enumerate}
\item Иерархия масс $\nu$, $\ell$, $u$, $d$ воспроизводится с точностью $<0.005\sigma$. 
\item Новый сектор <<g>> предсказывает существование базовых частиц (тахионов/гравитонов). 
\item Модель естественно включает фотоны, глюоны, W, Z и Хиггс как спектральные моды. 
\item Таким образом, массы и взаимодействия рождаются из чистой спектральной геометрии без подгонки параметров. 
\end{enumerate}

\begin{verbatim}
@misc{Zero Field Spectral Cosmology (ZFSC),
  author       = {Евгений Монахов and LLC "VOSCOM ONLINE" Research Initiative},
  title        = {Спектральная космология нулевого поля.Спектральное происхождение масс поколений частиц},
  year         = {2025},
  publisher    = {Zenodo},
  orcid        = {0009-0003-1773-5476},
  url_orcid    = {https://orcid.org/0009-0003-1773-5476},
  organization = {https://voscom.online/}
}
\end{verbatim}

\end{document}