\documentclass[12pt,a4paper]{article}
\usepackage[utf8]{inputenc}
\usepackage[T2A]{fontenc}
\usepackage[russian]{babel}
\usepackage{amsmath,amssymb}
\usepackage{hyperref}
\usepackage{geometry}
\usepackage{verbatim} % для блока цитирования
\geometry{margin=2.5cm}
\usepackage[T2A]{fontenc}
\usepackage[utf8]{inputenc}
\usepackage[russian]{babel}
\usepackage{braket}

\title{Zero-Field Spectral Cosmology (ZFSC): \\
Хроника исследовательского диалога человека и искусственного интеллекта}
\author{Евгений Монахов и ИИ-партнёр \\ VOSCOM Research Initiative}
\date{Сентябрь 2025}

\begin{document}
\maketitle

\begin{abstract}
Данный текст представляет собой хронику исследовательского диалога между человеком и искусственным интеллектом, в ходе которого была сформулирована и численно проверена нулевополевая спектральная космология (ZFSC).
Человек внёс аксиомы и интуитивную логику, а ИИ помог перевести их в строгую математическую форму и провести численные проверки.
Этот документ фиксирует путь открытия и может рассматриваться как логическая база ZFSC.
\end{abstract}

\section{Введение}
\subsection{Контекст}
Определение проблемы, которую классическая физика не решала: происхождение поколений фермионов, инфляции, тёмной энергии, сингулярностей.

\subsection{Роль человека и ИИ}
- Человек: аксиомы, логика, постановка вопросов.
- ИИ: формулы, спектральные матрицы, вычисления.
- Результат: синтез даёт новое видение космологии.

\section{Постановка аксиом}
\begin{enumerate}
  \item Нулевой уровень энтропии ($S\to 0$).
  \item Матрица связей как фундамент реальности.
  \item Собственные значения матрицы $\leftrightarrow$ массы и энергии.
  \item Луковично-фрактальная структура матрицы.
  \item Инварианты спектра = фундаментальные константы.
\end{enumerate}

\section{Поколения фермионов из геометрии}
Диалог: почему три поколения, как они возникают из первых трёх положительных собственных значений.
Численные проверки коэффициентов $c_\nu, c_\ell, c_u, c_d$.

\section{CKM и PMNS как следствие геометрии}
Диалог: почему CKM почти единичная, а PMNS даёт большие углы.
Результаты численных расчётов.

\section{Гравитон и тахион как нулевые и отрицательные моды}
Идея: нулевой мод = гравитон, отрицательный мод = тахион.
Пояснение, почему нет сингулярности.

\section{Энергодоли Вселенной}
Диалог: видимое вещество, тёмная материя, тёмная энергия как доли спектра.
Численные результаты: $\Omega_{\rm vis},\Omega_{\rm dark},\Omega_{\rm grav}$.

\section{Инфляция как раскалывание спектра}
Формулы:
\[
N(t) \sim \varphi^{t/\tau}, \qquad a(t)\propto N(t)^\alpha.
\]
Число e-folds без инфлатонного поля.

\section{Хиггс как центральная диагональ}
Диалог: где сидит бозон Хиггса в матрице.
Вывод: особый диагональный элемент бозонного блока $\Delta_H$.

\section{Скорость света как спектральный инвариант}
Формула:
\[
c \approx \sqrt{\kappa_{U(1)}}.
\]
Пояснение, почему $c$ одинаков для всех наблюдателей.

\section{Чёрные дыры и отсутствие сингулярности}
Вложенные подматрицы, отрицательные моды, сохранение информации.

\section{Будущее Вселенной}
Диалог: два сценария — насыщение спектра или фрактальные циклы.
Почему нет тепловой смерти.

\section{Заключение}
ZFSC как результат исследовательского диалога.
Список прорывных численных подтверждений.

\end{document}
