\documentclass[a4paper,12pt]{article}
\usepackage{geometry}
\geometry{margin=2.5cm}

% Языки
\usepackage[russian,english]{babel}

% --- Шрифты ---
\usepackage{fontspec}
\usepackage{unicode-math}
\setmainfont{CMU Serif}
\setmathfont{Latin Modern Math}

% --- Математика ---
\usepackage{amsmath}
\usepackage{braket}
\everymath{\displaystyle}
\emergencystretch=2em

\begin{document}

\section*{Стратегический план контактов (ZFSC)}

\subsection*{Фаза 1. Подготовка (сейчас $\to$ 6 месяцев)}
\textbf{Цель:} укрепить фундамент ZFSC, не раскрывая лишнего.
\begin{itemize}
  \item Завершить расчёты:
    \begin{itemize}
      \item массы поколений ($u,d,s,c,b,t; e,\mu,\tau; \nu$),
      \item матрицы смешивания (CKM, PMNS),
      \item $\sigma_{\pi N}$ в диапазоне $40\text{--}60\,\text{МэВ}$,
      \item постоянную тонкой структуры $\alpha$ из геометрии.
    \end{itemize}
  \item Подготовить 2--3 статьи в формате \texttt{arXiv}.
  \item Следить за смежными публикациями (например, поляронные модели).
\end{itemize}

\textbf{Коммуникации:}
\begin{itemize}
  \item Прямых контактов пока не заводить.
  \item Допустима короткая нейтральная реакция на статьи, без раскрытия деталей.
\end{itemize}

\subsection*{Фаза 2. Первые контакты (6--12 месяцев)}
\textbf{Цель:} мягко показать идею и собрать реакцию.
\begin{itemize}
  \item Адресаты:
    \begin{itemize}
      \item небольшие исследовательские группы (например, Афонин, Тулуб),
      \item локальные математики через личные контакты,
      \item сообщество \texttt{arXiv} через первую публикацию.
    \end{itemize}
  \item Формат писем:
    \begin{itemize}
      \item без ``жирных костей'',
      \item демонстрация: ``у нас есть спектральный метод, который даёт массы и смешивания'',
      \item вложения строго ограничены.
    \end{itemize}
\end{itemize}

\subsection*{Фаза 3. Выход в большую игру (12--24 месяца)}
\textbf{Цель:} заявить ZFSC как серьёзную альтернативу стандартной модели.
\begin{itemize}
  \item Адресаты:
    \begin{itemize}
      \item крупные коллаборации (ATLAS, CMS, IceCube, lattice QCD),
      \item ведущие теоретики по flavor physics,
      \item журналы (Nature Physics, PRD) и конференции (Moriond, QCD workshops).
    \end{itemize}
  \item С чем выходить:
    \begin{itemize}
      \item полный пакет результатов (массы, CKM/PMNS, $\alpha$, $\sigma_{\pi N}$),
      \item статьи на \texttt{arXiv},
      \item визуализации (спектры, матричные анимации).
    \end{itemize}
  \item Роль ранних контактов: возможное вовлечение в диалог, но уже с позиции силы.
\end{itemize}

\end{document}
