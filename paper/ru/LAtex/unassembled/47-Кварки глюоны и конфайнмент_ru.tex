\documentclass[12pt,a4paper]{article}
\usepackage[utf8]{inputenc}
\usepackage[russian]{babel}
\usepackage{amsmath,amssymb,amsfonts}
\usepackage{bm}
\usepackage{physics}
\usepackage{siunitx}
\usepackage{booktabs}
\usepackage{hyperref}
\usepackage{geometry}
\geometry{margin=2.3cm}
\hypersetup{colorlinks=true,linkcolor=blue,citecolor=blue,urlcolor=blue}
\usepackage[utf8]{inputenc}
\usepackage[russian]{babel}
\usepackage{tikz}
\usetikzlibrary{arrows.meta}
\usetikzlibrary{positioning}

\title{Кварки, глюоны и конфайнмент в спектральной космологии нулевого поля (ZFSC)}
\author{Евгений Монахов \\ VOSCOM ONLINE Research Initiative}
\date{\today}

\begin{document}
\maketitle

\section*{Введение}
В стандартной модели физики элементарных частиц кварки и глюоны описываются квантовой хромодинамикой (КХД), основанной на калибровочной группе $SU(3)$. 
Основным свойством КХД является \emph{конфайнмент} --- невозможность наблюдать кварки и глюоны в изолированном состоянии.  

В рамках Zero-Field Spectral Cosmology (ZFSC) мы предлагаем альтернативное объяснение этих свойств: 
кварки и глюоны интерпретируются как спектральные моды фундаментальной матрицы $H$, а конфайнмент возникает как следствие структуры спектра и запутанности между модами.

\section*{КХД: традиционная картина}
\begin{enumerate}
  \item Кварки несут ``цветной заряд'' и взаимодействуют через обмен глюонами.
  \item Глюоны (8 типов) сами несут цвет, поэтому они взаимодействуют друг с другом.
  \item Потенциал между кварками растёт с расстоянием:
  \[
  V(r) \sim \sigma \, r, \quad \sigma > 0,
  \]
  что приводит к конфайнменту.
\end{enumerate}

\section*{ZFSC: спектральная интерпретация}
\subsection*{Массы и моды}
В ZFSC фундаментальное утверждение:
\[
m = \lambda(H),
\]
где $\lambda(H)$ --- собственные значения матрицы $H$.

\begin{itemize}
  \item \textbf{Кварки:} низшие положительные собственные значения в секторе $SU(3)$.
  \item \textbf{Глюоны:} почти нулевые моды в том же секторе, отвечающие за связи между узлами.
  \item \textbf{Адроны:} коллективные состояния, энергия которых в основном обусловлена связностью спектра:
  \[
  M_{\text{адрона}} \approx \sum_i \lambda_{q_i}(H) + E_{\text{связи}}(H).
  \]
\end{itemize}

\subsection*{Конфайнмент как спектральное свойство}
Вместо ``растягивающейся струны'' КХД, в ZFSC имеем:
\begin{enumerate}
  \item Собственные значения $\lambda_q$ не существуют как изолированные мода.  
  \item Они реализуются только в комбинациях (триплеты $qqq$ или дублеты $q\bar q$).  
  \item Спектральное условие:
  \[
  \lambda_q \notin \mathrm{Spec}(H) \quad \text{изолированно}, \qquad
  \lambda_{qqq}, \lambda_{q\bar q} \in \mathrm{Spec}(H).
  \]
  \item Попытка вынести кварк ``наружу'' ведёт к перестройке спектра $H$ и рождению новой пары кварк-антикварк.
\end{enumerate}

\subsection*{Роль запутанности}
В ZFSC важен вклад квантовой запутанности между модами:
\[
\Delta E_s = \alpha \, I_{AB} + \beta \, I_{\mathrm{intra}},
\]
где $I_{AB}$ --- взаимная информация между слоями спектра, $I_{\mathrm{intra}}$ --- запутанность внутри слоя.

Для кварков и глюонов:
\begin{itemize}
  \item Запутанность усиливает коллективные состояния и подавляет изолированные.  
  \item Конфайнмент в ZFSC --- это не ``сила натяжения струны'', а \emph{устойчивое распределение запутанности}, которое ``склеивает'' кварки и глюоны в адроны.  
\end{itemize}

\section*{Сравнение КХД и ZFSC}
\begin{tabular}{|l|l|l|}
\hline
Свойство & КХД & ZFSC \\ \hline
Кварки & Фундаментальные частицы & Спектральные моды $H$ \\ \hline
Глюоны & Носители $SU(3)$, несут цвет & Почти нулевые моды (связи спектра) \\ \hline
Конфайнмент & Линейный потенциал $V(r)\sim \sigma r$ & Отсутствие изолированных мод, только комбинации \\ \hline
Массы адронов & Динамика КХД + энергия поля & Сумма собственных значений + энергия запутанности \\ \hline
\end{tabular}

\section*{Заключение}
ZFSC предоставляет новую картину кварков, глюонов и конфайнмента:
\begin{itemize}
  \item Массы кварков и глюонов определяются как спектральные значения матрицы $H$.
  \item Конфайнмент возникает не как ``струна'', а как условие спектральной связности и распределения запутанности.
  \item Адроны --- устойчивые коллективные состояния спектра.
\end{itemize}

Таким образом, свойства КХД находят естественное объяснение в рамках спектральной космологии нулевого поля.

\section*{Иллюстрация конфайнмента в ZFSC}

\begin{center}
\begin{tikzpicture}[node distance=3cm, thick, >=stealth]

% Nodes
\node[circle, draw, minimum size=1cm, fill=blue!15] (q1) {$q$};
\node[circle, draw, minimum size=1cm, fill=blue!15, right of=q1] (q2) {$q$};
\node[circle, draw, minimum size=1cm, fill=blue!15, below right of=q2, xshift=1.5cm, yshift=-1.5cm] (q3) {$q$};

% Proton triangle bonds
\draw (q1) -- (q2);
\draw (q2) -- (q3);
\draw (q3) -- (q1);

\node[above of=q2, yshift=1.5cm] (text1) {Адронное состояние $qqq$};

% Arrow to separation
\draw[->, thick] (q3) -- ++(3,-2) node[right] {Попытка вынести кварк};

% New pair
\node[circle, draw, minimum size=1cm, fill=red!15, right of=q3, xshift=4cm, yshift=-1.5cm] (q4) {$q$};
\node[circle, draw, minimum size=1cm, fill=red!15, below of=q4, yshift=-2cm] (qbar) {$\bar q$};

\draw (q4) -- (qbar);

\node[right of=q4, xshift=2.5cm, align=left] (text2) 
{Разрыв спектра $\Rightarrow$ \\
рождение новой пары $q\bar q$};

\end{tikzpicture}
\end{center}

\section*{Электрон и опыт с двумя щелями в ZFSC}

\subsection*{1. Электрон как спектральная мода}
В рамках ZFSC электрон трактуется не как ``частица-шарик'' и не как 
``волна вероятности'', а как \emph{стационарная спектральная мода} 
фундаментальной матрицы $H$ в лептонном секторе:
\[
m_e = \lambda_e(H), \qquad 
H |\psi_e\rangle = \lambda_e |\psi_e\rangle .
\]

Таким образом:
- масса электрона $m_e$ определяется спектром $H$;
- состояние электрона описывается собственным вектором $|\psi_e\rangle$.

\subsection*{2. Закон устойчивости моды}
По закону устойчивости спектральных мод:
\[
\frac{d\lambda_e}{dt} = 0 \quad \Rightarrow \quad 
E_e = \lambda_e c^2 = \text{const}.
\]

Энергия электрона не рассеивается и не затухает, в отличие от колебательной струны или механического осциллятора.  
Электрон сохраняет свою спектральную идентичность сколь угодно долго.

\subsection*{3. Опыт с двумя щелями}
Когда электрон проходит через барьер с двумя щелями, его спектральная мода $|\psi_e\rangle$ раскладывается на две компоненты, соответствующие геометрическим путям:

\[
|\psi_e\rangle \;\to\; |\psi_1\rangle + |\psi_2\rangle ,
\]

где $|\psi_1\rangle$ и $|\psi_2\rangle$ --- амплитуды прохождения через первую и вторую щель соответственно.  

На экране наблюдается плотность вероятности:
\[
P(x) = |\psi_1(x) + \psi_2(x)|^2 
= |\psi_1(x)|^2 + |\psi_2(x)|^2 + 2\mathrm{Re}\{\psi_1^*(x)\psi_2(x)\}.
\]

Интерференционный член $2\mathrm{Re}\{\psi_1^*\psi_2\}$ отражает \emph{внутреннюю спектральную структуру} моды электрона, а не ``мистику двойственности''.

\subsection*{4. Интерпретация ZFSC}
\begin{itemize}
  \item В традиционной КМ интерферирует ``вероятность''.  
  \item В ZFSC интерферирует сама \emph{спектральная мода}, которая обязана проявляться как волновая структура, поскольку она является решением уравнения $H|\psi\rangle=\lambda|\psi\rangle$.
  \item Экран не фиксирует ``частицу-шарик'', а регистрирует место, где мода $|\psi_e\rangle$ схлопывается в акте взаимодействия.
\end{itemize}

\subsection*{5. Рассеяние электронов}
При рассеянии электронов друг на друге (или на ядрах) мы имеем взаимодействие спектральных мод:
\[
\sigma \sim |\langle \psi_{e1} | \psi_{e2} \rangle|^2,
\]
где $\sigma$ --- эффективное сечение рассеяния, а перекрытие собственных векторов 
$\langle \psi_{e1}|\psi_{e2}\rangle$ определяет вероятность взаимодействия.

\subsection*{6. Вывод}
\begin{itemize}
  \item Электрон --- стационарная спектральная мода ($m_e=\lambda_e(H)$).
  \item Устойчивость: $\tfrac{d\lambda_e}{dt}=0 \Rightarrow E_e = \text{const}$.
  \item Интерференция в двух щелях --- прямое проявление спектральной природы электрона.
  \item Рассеяние --- результат перекрытия собственных векторов спектра.
\end{itemize}

Таким образом, ZFSC снимает ``волново-корпускулярную загадку'' электрона: 
он всегда является спектральной модой, а его волновые свойства следуют 
не из ``двойственности'', а из фундаментальной матричной структуры Вселенной.

\section*{ZFSC Poster: Electron}

\begin{center}
\framebox{
\begin{minipage}{0.9\linewidth}
\begin{align*}
m_e &= \lambda_e(H) 
&& \text{(Electron = spectral mode)} \\[8pt]
\frac{d\lambda_e}{dt} &= 0 
&& \text{(Stability law: $E_e=\lambda_e c^2$ const)} \\[8pt]
|\psi_e\rangle &\to |\psi_1\rangle + |\psi_2\rangle 
&& \text{(Two-slit decomposition)} \\[8pt]
P(x) &= |\psi_1(x) + \psi_2(x)|^2 
&& \text{(Interference pattern)} \\[8pt]
\sigma &\sim |\langle \psi_{e1}|\psi_{e2}\rangle|^2 
&& \text{(Scattering as overlap)}
\end{align*}
\end{minipage}
}
\end{center}

\section*{Закон устойчивости спектральных мод (ZFSC)}

\textbf{Постулат.} 
Собственные значения фундаментальной матрицы $H$ являются 
\emph{стационарными спектральными модами}, которые не теряют энергию во времени.  

\subsection*{Формулировка}
Пусть 
\[
H \, |\psi_n\rangle = \lambda_n |\psi_n\rangle ,
\]
где $\lambda_n \in \mathbb{R}$ --- собственные значения, а $|\psi_n\rangle$ --- собственные векторы.

Тогда энергия элементарного состояния выражается как
\[
E_n = \lambda_n c^2 .
\]

\subsection*{Вывод}
\begin{enumerate}
  \item Для колебательных систем (струна, колокол) собственные частоты $\omega_k$ связаны с внешней средой: 
  \[
  E(t) = E_0 e^{-\gamma t}, \qquad \gamma > 0,
  \]
  где $\gamma$ --- коэффициент затухания.
  
  \item Для фундаментальной матрицы $H$ затухание отсутствует, поскольку $H$ описывает \emph{замкнутую спектральную систему}.  
  Собственные значения $\lambda_n$ инвариантны во времени:
  \[
  \frac{d\lambda_n}{dt} = 0.
  \]
  
  \item Следовательно, энергия состояния:
  \[
  E_n(t) = \lambda_n c^2 = \text{const}.
  \]
\end{enumerate}

\subsection*{Интерпретация}
\begin{itemize}
  \item Струна теряет энергию $\Rightarrow$ открытая система, обмен с внешней средой.  
  \item Частица (электрон, кварк, глюон) в ZFSC $\Rightarrow$ стационарная мода спектра, замкнутая на $H$, не имеющая каналов утечки.
\end{itemize}

\textbf{Вывод:} 
Устойчивость элементарных частиц есть прямое следствие 
неизменности спектра матрицы $H$. 
Частицы — это «вечные ноты» Вселенной, которые звучат, пока существует сам фундаментальный оператор $H$.



\end{document}
