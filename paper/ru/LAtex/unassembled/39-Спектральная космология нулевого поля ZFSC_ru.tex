\documentclass[12pt,a4paper]{article}
\usepackage[utf8]{inputenc}
\usepackage[russian]{babel}
\usepackage{amsmath,amssymb,amsfonts}
\usepackage{bm}
\usepackage{physics}
\usepackage{siunitx}
\usepackage{booktabs}
\usepackage{hyperref}
\usepackage{geometry}
\geometry{margin=2.3cm}
\hypersetup{colorlinks=true,linkcolor=blue,citecolor=blue,urlcolor=blue}
\usepackage[utf8]{inputenc}
\usepackage[russian]{babel}
\usepackage{tikz}
\usetikzlibrary{arrows.meta}
\usetikzlibrary{positioning}
\title{Спектральная космология нулевого поля (ZFSC) \\
\large Теоретическая основа происхождения масс поколений частиц и фундаментальных взаимодействий}

\author{Евгений Монахов \\
ООО ``VOSCOM ONLINE'' Research Initiative \\
\href{https://orcid.org/0009-0003-1773-5476}{ORCID: 0009-0003-1773-5476}}

\date{Сентябрь 2025}

\begin{document}
\maketitle

\begin{abstract}
Zero-Field Spectral Cosmology (ZFSC) --- новая концептуальная теория, в которой массы поколений частиц и фундаментальные взаимодействия возникают из спектральных свойств дискретных многослойных матриц, определённых на нулевом уровне энтропии.
Теория воспроизводит иерархии масс фермионов, матрицы смешивания CKM и PMNS, а также предсказывает бозонные моды, включая кандидата на гравитон. 
В работе приведены основные постулаты, математическая формулировка и первые совпадения с экспериментальными данными.
\end{abstract}

%------------------------------------
\section{Введение}
Современная физика опирается на Стандартную модель (СМ), успешно описывающую фундаментальные взаимодействия. 
Однако происхождение масс частиц, иерархия поколений и структура смешивания остаются открытыми вопросами. 
Предлагаемая теория спектральной космологии нулевого поля (ZFSC) рассматривает Вселенную как спектральную матричную систему на нулевом уровне энтропии, где собственные значения соответствуют наблюдаемым массам частиц и свойствам взаимодействий.

%------------------------------------
\section{Постулат 1: Нулевой уровень энтропии}
Предполагается существование фундаментального уровня, на котором отсутствуют время и пространство, а энтропия стремится к нулю:
\[
S \to 0.
\]
На этом уровне Вселенная описывается чистым вероятностным полем амплитуд:
\[
\Psi = \sum_{i} a_i |i\rangle ,
\]
где $\{|i\rangle\}$ --- потенциальные конфигурации, а $a_i \in \mathbb{C}$ --- их амплитуды.

%------------------------------------
\section{Постулат 2: Матричная структура}
Реальность представляется как вложенные ``луковичные'' уровни матриц:
\[
H^{(n)} =
\begin{bmatrix}
H^{(n-1)} & V \\
V^\dagger & H^{(n-1)}
\end{bmatrix},
\]
где $V$ --- операторы связи между слоями.
Спектр собственных значений $H^{(n)}$ формирует физические массы и взаимодействия.

%------------------------------------
\section{Фермионный спектр}
Массы поколений фермионов (нейтрино, лептоны, кварки up/down) соответствуют первым трём положительным собственным значениям в разных секторах:
\[
m^{(f)}_k = \lambda^{(f)}_k, \quad f \in \{ \nu, \ell, u, d \}, \ k=1,2,3.
\]
Разные геометрические трансформации секторов приводят к CKM- и PMNS-матрицам смешивания.

%------------------------------------
\section{Бозонный слой}
Нижние собственные моды спектра интерпретируются как бозоны:
\begin{itemize}
  \item $\lambda_0 \approx 0$ --- кандидат на гравитон;
  \item $\lambda_0 < 0$ --- тахионная мода (нестабильность поля);
  \item остальные низкие значения --- фотоны, глюоны, W/Z, Хиггс.
\end{itemize}

%------------------------------------
\section{Совпадения с экспериментом}
\begin{itemize}
  \item плато масс трёх поколений совпадают с экспериментальными данными (точность $\sim 10^{-2}$);
  \item иерархии $c_\nu, c_\ell, c_u, c_d$ воспроизводят наблюдаемые соотношения;
  \item CKM-матрица близка к единичной, PMNS имеет большие углы;
  \item получены нулевые и отрицательные моды (гравитон, тахион).
\end{itemize}

%------------------------------------
\section{Заключение}
ZFSC демонстрирует, что массы частиц и структура взаимодействий могут иметь чисто спектральное происхождение. 
Теория даёт ряд совпадений со Стандартной моделью и предсказывает новые эффекты, требующие проверки.
В дальнейшем планируется расширение вычислений, построение предсказаний для тёмной материи и проверка бозонных мод.

%------------------------------------
\section*{Благодарности}
Автор выражает признательность коллегам и сообществу VOSCOM ONLINE за поддержку и обсуждение.

%------------------------------------
\section*{Лицензия}
Документ распространяется по лицензии Creative Commons CC-BY 4.0.  
Код и расчёты — MIT License.  

\end{document}
