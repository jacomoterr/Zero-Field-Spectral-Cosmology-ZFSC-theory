\documentclass[12pt,a4paper]{article}
\usepackage[utf8]{inputenc}
\usepackage[russian]{babel}
\usepackage{amsmath,amssymb,amsfonts}
\usepackage{bm}
\usepackage{physics}
\usepackage{siunitx}
\usepackage{booktabs}
\usepackage{hyperref}
\usepackage{geometry}
\geometry{margin=2.3cm}
\hypersetup{colorlinks=true,linkcolor=blue,citecolor=blue,urlcolor=blue}
\usepackage[utf8]{inputenc}
\usepackage[russian]{babel}
\usepackage{tikz}
\usetikzlibrary{arrows.meta}
\usetikzlibrary{positioning}

\title{Zero-Field Spectral Cosmology (ZFSC): \\
Вселенная как квантовый компьютер, квантовая запутанность и тёмная материя}

\author{Евгений Монахов \\
ООО ``VOSCOM ONLINE'' Research Initiative \\
\href{https://orcid.org/0009-0003-1773-5476}{ORCID: 0009-0003-1773-5476}}

\date{Сентябрь 2025}

\begin{document}
\maketitle

\begin{abstract}
Zero-Field Spectral Cosmology (ZFSC) --- новая теория, рассматривающая массы поколений частиц и фундаментальные взаимодействия как спектральные свойства дискретных многослойных матриц, определённых на нулевом уровне энтропии. 
Показано, что Вселенную можно интерпретировать как многомерный квантовый компьютер особого рода. 
Квантовая запутанность описывается как проекция единого собственного вектора на несколько частиц, что объясняет феномен ``страшного дальнодействия''. 
Обсуждаются кандидаты на тёмную материю как высокие моды спектра, а также предлагаются экспериментальные способы проверки предсказаний ZFSC.
\end{abstract}

%------------------------------------
\section{Введение}
Стандартная модель (СМ) успешно описывает взаимодействия частиц, однако происхождение масс, иерархия поколений и природа тёмной материи остаются загадками. 
ZFSC предлагает новый подход: все массы и взаимодействия возникают из спектральных свойств дискретных матриц, заданных на фундаментальном уровне $S \to 0$.

%------------------------------------
\section{Постулат 1: Нулевой уровень энтропии}
На фундаментальном уровне отсутствуют время и пространство, а энтропия стремится к нулю:
\[
S \to 0.
\]
Состояние описывается как вероятностное поле амплитуд:
\[
\Psi = \sum_{i} a_i |i\rangle, \quad a_i \in \mathbb{C}.
\]

%------------------------------------
\section{Постулат 2: Матричная структура}
Реальность имеет иерархическую матричную структуру:
\[
H^{(n)} =
\begin{bmatrix}
H^{(n-1)} & V \\
V^\dagger & H^{(n-1)}
\end{bmatrix}.
\]
Собственные значения $\lambda_k$ формируют массы частиц, собственные векторы $u_k$ задают их взаимодействия.

%------------------------------------
\section{Фермионный спектр}
Массы поколений фермионов (нейтрино, лептоны, кварки) соответствуют первым трём положительным собственным значениям:
\[
m^{(f)}_k = \lambda^{(f)}_k, \quad f \in \{\nu, \ell, u, d\}, \ k=1,2,3.
\]

%------------------------------------
\section{Бозонный слой}
Нижние собственные моды интерпретируются как бозоны:
\begin{itemize}
  \item $\lambda_0 \approx 0$ --- кандидат на гравитон,
  \item $\lambda_0 < 0$ --- тахионная мода,
  \item малые положительные $\lambda_k$ --- фотоны, глюоны, W, Z, Хиггс.
\end{itemize}

%------------------------------------
\section{Вселенная как квантовый компьютер}
ZFSC позволяет рассматривать Вселенную как квантовый компьютер:  
\begin{itemize}
  \item пространство состояний $\Psi$ играет роль гигантского квантового регистра,
  \item матрицы $H^{(n)}$ --- логические преобразования,
  \item собственные значения $\lambda_k$ --- результаты вычислений, устойчивые физические массы.
\end{itemize}

---

\section{Квантовая запутанность в ZFSC}
Запутанность интерпретируется как проекция единого собственного вектора на разные частицы.  
Если $u = (u_A, u_B)$, то
\[
|\Psi_{AB}\rangle = \sum_k u_{A,k} |k_A\rangle \otimes u_{B,k} |k_B\rangle.
\]
Мгновенные корреляции объясняются единством вектора $u$, существующего вне пространства и времени.

\begin{figure}[h!]
\centering
\begin{tikzpicture}[node distance=2cm, thick]
\node[draw, rectangle, minimum width=4cm, minimum height=2cm, fill=blue!5] (matrix) {Спектральный слой $H^{(n)}$};
\draw[-{Latex[length=3mm]}, thick, red] (matrix) -- ++(0,-2) node[midway, right] {$u_k$};
\node[draw, circle, minimum size=1cm, below left=2cm and 1cm of matrix, fill=green!10] (A) {A};
\node[draw, circle, minimum size=1cm, below right=2cm and 1cm of matrix, fill=green!10] (B) {B};
\draw[-{Latex[length=3mm]}, dashed] (matrix.south west) .. controls +(0,-1) and +(0,1) .. (A.north);
\draw[-{Latex[length=3mm]}, dashed] (matrix.south east) .. controls +(0,-1) and +(0,1) .. (B.north);
\draw[<->, thick, blue, dotted] (A.east) -- (B.west) node[midway, below] {Мгновенная корреляция};
\end{tikzpicture}
\caption{Запутанные частицы как проекции одного собственного вектора.}
\end{figure}

---

\section{У каких частиц запутанность выражена сильнее}
\begin{itemize}
  \item Нейтрино --- максимальная запутанность (лёгкие моды, сильная связность).
  \item Фотоны, глюоны --- высокая запутанность (безмассовые, низкие моды).
  \item Электроны и лептоны --- умеренная.
  \item Тяжёлые кварки --- слабая запутанность (локализованные моды).
\end{itemize}

---

\section{Высокие моды и тёмная материя}
Высокие $\lambda_k$ почти ортогональны нижним слоям:
\[
H^{(n)} u_k = \lambda_k u_k, \quad \lambda_k \gg m_{\mathrm{известные}}.
\]
Свойства:
\begin{itemize}
  \item стабильность,
  \item слабое взаимодействие,
  \item большой вклад в плотность энергии.
\end{itemize}
Это естественные кандидаты на тёмную материю.

\begin{figure}[h!]
\centering
\begin{tikzpicture}[xscale=2, yscale=1.2, thick]
\draw[->] (0,0) -- (5,0) node[right] {$\lambda_k$};
\foreach \x/\label in {0.5/нейтрино, 1.0/лептоны, 1.5/кварки, 2.0/бозоны} {
  \draw[fill=green!60] (\x,0) circle (3pt);
  \node[below] at (\x,-0.1) {\small \label};
}
\foreach \x in {3.5, 4.0, 4.5} {
  \draw[fill=purple!70] (\x,0) circle (3pt);
}
\node[below] at (4.0,-0.1) {\small Тёмная материя};
\end{tikzpicture}
\caption{Низкие моды = известные частицы; высокие = тёмная материя.}
\end{figure}

---

\section{Прогнозы для экспериментов}
\subsection{Запутанность нейтрино}
\[
S_\nu = - \mathrm{Tr}(\rho_\nu \ln \rho_\nu),
\]
энтропия фон Неймана должна быть выше, чем у лептонов/кварков.  
Проверка: осцилляции (JUNO, IceCube).  

\subsection{Запутанность фотонов и глюонов}
Аномально устойчивая корреляция. Проверка: эксперименты Белла, коллайдеры (струи).  

\subsection{Высокие моды}
\begin{itemize}
  \item космология (CMB, $\Omega_{\mathrm{DM}}$),
  \item гравитационное линзирование,
  \item резонансы на ускорителях:
  \[
  \sigma(E) \propto \frac{1}{(E^2 - \lambda_k^2)^2 + \Gamma_k^2},
  \]
  \item подземные детекторы (WIMP-like события).
\end{itemize}

\subsection{Сравнительная таблица}

\begin{table}[h!]
\centering
\begin{tabular}{|c|c|c|}
\hline
Эффект & ZFSC & Стандартная модель \\
\hline
Запутанность нейтрино & Очень высокая & Не описывается явно \\
Запутанность фотонов & Аномально устойчивая & Ограничена Беллами \\
Тёмная материя & Высокие моды спектра & Вводится феноменологически \\
Гравитон & Нулевая мода $\lambda_0 \approx 0$ & Гипотетический квант \\
\hline
\end{tabular}
\caption{Сравнение предсказаний ZFSC и СМ.}
\end{table}

---

\section{Заключение}
ZFSC показывает, что:
\begin{itemize}
  \item Вселенная может рассматриваться как квантовый компьютер особого рода;
  \item квантовая запутанность естественно объясняется единством собственных векторов;
  \item у нейтрино и безмассовых бозонов запутанность выше, у тяжёлых кварков ниже;
  \item высокие моды спектра --- естественные кандидаты на тёмную материю.
\end{itemize}

\section*{Лицензия}
Документ --- CC-BY 4.0.  
Код и расчёты --- MIT License.  

\end{document}
