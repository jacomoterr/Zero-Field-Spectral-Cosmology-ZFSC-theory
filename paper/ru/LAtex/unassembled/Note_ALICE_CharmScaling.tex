\documentclass[a4paper,12pt]{article}
\usepackage{geometry}
\geometry{margin=2.5cm}

% Языки
\usepackage[russian,english]{babel}

% --- Шрифты ---
\usepackage{fontspec}
\usepackage{unicode-math}
\setmainfont{CMU Serif}          % текст (кириллица + латиница)
\setmathfont{Latin Modern Math}  % строгие формулы

% --- Математика ---
\usepackage{amsmath}
\usepackage{braket}
\everymath{\displaystyle}
\emergencystretch=2em

\title{Note on a Scaling Relation in ALICE Charm Hadronization}
\author{}
\date{\today}

\begin{document}
\maketitle
\selectlanguage{english}

Dear ALICE Heavy-Flavor Group,

We suggest a simple baryon--meson scaling involving charm hadrons in Pb--Pb collisions:
\[
v_2(\Lambda_c) - v_2(D^0) \ \propto\  \log\!\left(\frac{\Lambda_c}{D^0}\right),
\]
where $\Lambda_c/D^0$ is the yield ratio in the same $(p_T,\,$centrality$)$ bin. The proportionality factor is expected to be slowly varying with centrality.

\section*{How it can be tested on existing ALICE data}
\begin{itemize}
  \item \textbf{Inputs:} published $v_2$ measurements for $D^0$ and $\Lambda_c$ vs. $p_T$ and centrality at $\sqrt{s_{NN}}=5.02$ TeV; corresponding $\Lambda_c/D^0$ yield ratios in matched bins.
  \item \textbf{Binning and matching:} harmonize centrality and $p_T$ binning; when necessary, interpolate within published ranges with conservative systematics.
  \item \textbf{Correlation test:} for each bin, compute $\Delta v_2 \equiv v_2(\Lambda_c)-v_2(D^0)$ and $Y \equiv \log(\Lambda_c/D^0)$. Perform a linear fit $\Delta v_2 = A + B\,Y$ with full covariance of statistical and systematic uncertainties.
  \item \textbf{Cross-checks:} repeat for different centrality classes; verify that the slope $B$ is stable within uncertainties. Compare to $p$--Pb and $pp$ baselines (where coalescence is weaker) as a control.
  \item \textbf{Systematics:} test sensitivity to feed-down, selection efficiencies, and non-flow corrections; the relation should persist within combined uncertainties if it is genuine.
\end{itemize}

\section*{Why this may be important}
\begin{itemize}
  \item \textbf{Universality hint:} a logarithmic link between baryon/meson composition and elliptic flow would indicate a common underlying mechanism (e.g., coalescence-driven degrees of freedom) governing both hadronization and collectivity for charm.
  \item \textbf{Model constraints:} such scaling can discriminate between fragmentation-dominated and coalescence-dominated pictures and reduce parameter degeneracies in heavy-flavor transport models.
  \item \textbf{Compact observable:} $\Delta v_2$ vs. $\log(\Lambda_c/D^0)$ gives a one-curve summary across centralities and $p_T$ that is easy to compare with theoretical calculations.
\end{itemize}

\bigskip
Best regards, \\
Evgeny Monakhov \\
Independent Researcher \\
VOSCOM ONLINE \\
evgeny.monakhov@voscom.online

\end{document}
