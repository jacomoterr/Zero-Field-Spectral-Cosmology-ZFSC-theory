\documentclass[12pt,a4paper]{article}
\usepackage[utf8]{inputenc}
\usepackage[russian]{babel}
\usepackage{amsmath,amssymb,amsfonts}
\usepackage{bm}
\usepackage{physics}
\usepackage{siunitx}
\usepackage{booktabs}
\usepackage{hyperref}
\usepackage{geometry}
\geometry{margin=2.3cm}
\hypersetup{colorlinks=true,linkcolor=blue,citecolor=blue,urlcolor=blue}
\usepackage[utf8]{inputenc}
\usepackage[russian]{babel}
\usepackage{tikz}
\usetikzlibrary{arrows.meta}
\usetikzlibrary{positioning}

\title{Spectral Origin of Mass: \\ 
From Zero-Field Spectral Cosmology to Einstein's Equation}
\author{Evgeny Monakhov \\ VOSCOM ONLINE Research Initiative}
\date{\today}

\begin{document}
\maketitle

\section*{Introduction}
Albert Einstein’s famous equation
\[
E = mc^2
\]
became the iconic symbol of 20th century physics, showing the deep equivalence between mass and energy.  
However, in this formulation, the mass $m$ is assumed as a given property of matter.  

In the framework of \textbf{Zero-Field Spectral Cosmology (ZFSC)} we propose a deeper principle:  
\emph{mass itself is not fundamental, but arises as a spectral eigenvalue of a fundamental matrix $H$.}  
This idea is summarized by a simple equation:
\[
m = \lambda(H),
\]
where $\lambda(H)$ denotes the eigenvalues of $H$, which encodes the hidden structure of the zero-entropy field.

\section*{ZFSC Postulates}
\begin{enumerate}
  \item \textbf{Zero-entropy level:}
  \[
  S \to 0, \qquad 
  \Psi = \sum_i a_i |i\rangle .
  \]
  The Universe at its most fundamental level is described as a probabilistic amplitude field.

  \item \textbf{Mass as spectrum:}
  \[
  m = \lambda(H).
  \]

  \item \textbf{Particle generations:}
  \[
  m^{(n)}_f = \lambda_n(H), \quad n=1,2,3.
  \]

  \item \textbf{Mixing rule:}
  \[
  \mathrm{Mix} = U_A^\dagger U_B.
  \]
\end{enumerate}

\section*{Core Law: $m = \lambda(H)$}
The formula states that mass is not a primitive attribute but a result of spectral properties of $H$.  
Just as sounds emerge from a bell, or notes from a piano, particle masses emerge from the hidden matrix structure.

\subsection*{Everyday Analogies}
\begin{itemize}
  \item  \textbf{Piano:} instrument $H$ $\rightarrow$ notes $\lambda(H)$ $\rightarrow$ masses as sounds.
  \item  \textbf{Bell:} geometry $H$ $\rightarrow$ resonant tones $\lambda(H)$ $\rightarrow$ masses as frequencies.
  \item  \textbf{Prism:} structure $H$ $\rightarrow$ spectrum $\lambda(H)$ $\rightarrow$ masses as colors.
\end{itemize}

\section*{Connection with Einstein}
Einstein’s formula:
\[
E = mc^2
\]
becomes a corollary when substituting $m = \lambda(H)$:
\[
E = \lambda(H) \, c^2.
\]
Thus, particle energy is directly determined by its spectral eigenvalue.

\section*{Stepwise Derivation}
\begin{enumerate}
  \item Fundamental law: $m = \lambda(H)$.
  \item Substitution: $E = \lambda(H)c^2$.
  \item Quantum link: $E = \hbar \omega$ $\;\Rightarrow\;$ eigenvalues as frequency quanta.
  \item Geometry and gauge: structure of $H$ encodes $SU(3)\times SU(2)\times U(1)$.
  \item Classical limit: effective large-scale physics.
\end{enumerate}

\section*{Hypotheses and Extensions}
\begin{itemize}
  \item Zero mode: $\lambda_0 \approx 0 \Rightarrow$ graviton candidate.
  \item Negative modes: $\lambda<0 \Rightarrow$ tachyonic sectors.
  \item Entanglement corrections: spectral shifts via mutual information.
  \item Time evolution of couplings: $G_\mathrm{eff}(t)$, $\alpha(t)$ varying slowly with cosmological time.
\end{itemize}

\section*{Conclusion}
The equation $m=\lambda(H)$ is more fundamental than $E=mc^2$, because it explains the \emph{origin of mass itself}.  
Einstein demonstrated that mass and energy are equivalent; ZFSC shows where mass comes from.  

\section*{Research Outlook}
\begin{enumerate}
  \item Numerical spectral scans of large matrices $H$.
  \item Fitting particle masses and mixing matrices to experimental data.
  \item Study of tachyonic modes and their cosmological implications.
  \item Modeling time evolution of effective couplings $G_\mathrm{eff}(t)$.
\end{enumerate}

\end{document}
