\section{Сравнение: ZFSC v1.0 и ZFSC v2.0}

Для ясности зафиксируем основные отличия между первой формулировкой ZFSC и расширенной версией v2.0.

\begin{table}[h!]
\centering
\begin{tabular}{@{}p{4cm}p{5.5cm}p{5.5cm}@{}}
\toprule
 & \textbf{ZFSC v1.0} & \textbf{ZFSC v2.0} \\
\midrule
\textbf{Пространство состояний} & Только материальные моды $|i_{\mathrm{matter}}\rangle$ & Тензорное произведение материальных и геометрических мод $|i_{\mathrm{matter}}\rangle \otimes |j_{\mathrm{geom}}\rangle$ \\
\textbf{Геометрия} & Рассматривалась как фиксированный фон & Квантована через фибоначчиевские, golden Laplacian, q-Фибоначчи, фрактальные или случайные матрицы \\
\textbf{Параметры} & $(\Delta, r, g_L, g_R, h_1,h_2,h_3)$ & Расширено: $(\Delta, r, g_L, g_R, h_1,h_2,h_3, \alpha, \beta, n_*(t))$ \\
\textbf{Запутанность} & Не учитывалась & Введены явные поправки $\Delta E_s = \alpha I_{AB} + \beta I_{\mathrm{intra}}$ \\
\textbf{Возраст Вселенной} & Не связан со спектром & Закодирован как глубина аппроксимации Фибоначчи $n_*(t)$ \\
\textbf{Обратная задача} & Только прямой расчёт спектров & Обратная задача: реконструкция $H_{\mathrm{geom}}$ по экспериментальным массам и матрицам смешивания \\
\textbf{Закон спектра} & $d\lambda_n/dt = 0$ (устойчивость) & То же, но для совместного спектра материя+геометрия \\
\bottomrule
\end{tabular}
\caption{Эволюция концепции ZFSC: от версии 1.0 к версии 2.0.}
\end{table}

\section{Преимущества ZFSC v2.0}
\begin{itemize}
    \item \textbf{Унификация:} материя и геометрия трактуются на равных основаниях как квантованные состояния.  
    \item \textbf{Гибкость:} внутри одной модели можно тестировать разные геометрии (Фибоначчи, случайные, фрактальные).  
    \item \textbf{Привязка к наблюдениям:} реконструкция напрямую из экспериментальных данных (массы фермионов, CKM/PMNS).  
    \item \textbf{Космологическая связь:} возраст Вселенной естественно кодируется глубиной аппроксимации $n_*(t)$.  
    \item \textbf{Фальсифицируемость:} гипотеза о Фибоначчи проверяема: если природа не использует золотое квантование, вес $c_{\mathrm{Fib}}$ обнуляется.  
\end{itemize}

\section{Нерешённые проблемы}
\begin{enumerate}
    \item \textbf{Численные алгоритмы:} эффективные методы решения обратной задачи для больших матриц ещё не отработаны.  
    \item \textbf{Физическая интерпретация:} требуется уточнение роли отрицательных собственных значений (тахионов) в космологии.  
    \item \textbf{Параметры запутанности:} константы $\alpha, \beta$ в $\Delta E_s$ нуждаются в физическом выводе или экспериментальной оценке.  
    \item \textbf{Космическая эволюция:} необходимо откалибровать функцию $n_*(t)$ на данных по CMB и формированию структур.  
    \item \textbf{Универсальность:} остаётся вопрос, уникальна ли структура Фибоначчи или это лишь один из возможных квазипериодических/фрактальных кандидатов.  
    \item \textbf{Экспериментальные сигналы:} важно найти дополнительные наблюдаемые (например, в спектрах бозонов), выходящие за рамки масс и смешиваний фермионов.  
\end{enumerate}
