\documentclass[a4paper,12pt]{article}
\usepackage{geometry}
\geometry{margin=2.5cm}

% Языки
\usepackage[russian,english]{babel}

% --- Шрифты ---
\usepackage{fontspec}
\usepackage{unicode-math}
\setmainfont{CMU Serif}
\setmathfont{Latin Modern Math}

% --- Математика ---
\usepackage{amsmath}
\usepackage{braket}
\everymath{\displaystyle}
\emergencystretch=2em

\begin{document}

\title{Нейтронные звёзды в рамках Zero-Field Spectral Cosmology (ZFSC)}
\author{Evgeny Monakhov \\ Independent Researcher \\ VOSCOM ONLINE}
\date{}
\maketitle

\section*{Введение}
Нейтронные звёзды --- уникальные астрофизические объекты, в которых проявляется физика экстремальных плотностей и сильных полей. Они представляют собой ``пограничные состояния'' материи: от устойчивых конфигураций плотного ядерного вещества до переходов в чёрные дыры.  

Современные наблюдения показывают ряд аномалий:
\begin{itemize}
  \item наличие нейтронных звёзд с массами $\gtrsim 2.3\,M_\odot$, выше классического предела Толмана--Оппенгеймера--Волкова (TOV);
  \item сверхсильные магнитные поля ($10^{14}$--$10^{15}$ Гс) у магнетаров;
  \item быстрые радиовсплески (FRB) и глитчи --- внезапные скачки вращения;
  \item детали уравнения состояния (EOS) плотной материи остаются неизвестными.
\end{itemize}

ZFSC предлагает интерпретацию этих эффектов через спектральные свойства фундаментальной матрицы Вселенной.

\section*{Классическая картина}
Структура нейтронной звезды описывается уравнением Толмана--Оппенгеймера--Волкова:
\[
\frac{dP(r)}{dr} = -\frac{G}{r^2}\frac{\left[\rho(r)+\frac{P(r)}{c^2}\right]\left[M(r)+4\pi r^3 \frac{P(r)}{c^2}\right]}{1-\frac{2GM(r)}{rc^2}},
\]
где $P(r)$ --- давление, $\rho(r)$ --- плотность, $M(r)$ --- масса внутри радиуса $r$.  

Для завершения задачи требуется уравнение состояния $P(\rho)$. В классической астрофизике оно неизвестно при сверхядерных плотностях; возможны гиперонные, кварковые или смешанные фазы.  

Максимальная масса ограничена пределом TOV:
\[
M_{\text{max}} \approx 2.0\text{--}2.3\,M_\odot,
\]
в зависимости от выбранного EOS. Однако наблюдения намекают на более массивные объекты.

\section*{ZFSC-картина}
В ZFSC масса и устойчивость системы связаны с собственными значениями базовой матрицы $H$:
\[
m = \lambda_n(H)c^2 + \Delta E_s,
\]
где $\Delta E_s$ --- поправки от запутанности:
\[
\Delta E_s = \alpha\,I_{\text{inter}} + \beta\,I_{\text{intra}}.
\]

\subsection*{Устойчивость и масса}
Нейтронная звезда соответствует конфигурации, где несколько низших мод находятся на устойчивом плато.  
\begin{itemize}
  \item Когда $\lambda_n(H)$ перестаёт быть постоянным, звезда теряет устойчивость $\Rightarrow$ коллапс в чёрную дыру.
  \item Наличие тахионных или аксионных мод может временно стабилизировать систему и допустить массы $M \sim 2.5\,M_\odot$.
\end{itemize}

\subsection*{Магнитные поля}
Магнитные поля возникают как возбуждения $U(1)$-сектора связности. При нейтронной плотности реализуется резонанс, что ведёт к устойчивым полям:
\[
B \sim 10^{14}\text{--}10^{15}\,\text{Гс}.
\]

\subsection*{Уравнение состояния}
ZFSC трактует EOS как выбор ветви спектрального плато:
\[
P = f(\lambda_i(H), \Delta E_s).
\]
Разные варианты (нейтронная, гиперонная, кварковая материя) соответствуют разным наборам собственных значений и их комбинациям.

\section*{Эффекты и наблюдаемые явления}

\subsection*{Максимальная масса}
ZFSC объясняет существование нейтронных звёзд с массами $>2.3 M_\odot$ за счёт стабилизирующего вклада дополнительных мод (тахионных/аксионных).  

\subsection*{Гравитационные волны}
Слияние двух нейтронных звёзд соответствует интерференции спектров двух узлов матрицы. В спектре гравитационных волн должны проявляться дополнительные пики --- ``спектральные глитчи'', отражающие перестройку мод.

\subsection*{FRB и глитчи}
\begin{itemize}
  \item \textbf{Глитч:} резкий переход системы в соседнее спектральное плато $\Rightarrow$ изменение момента инерции и частоты вращения.
  \item \textbf{FRB:} коллективный выброс энергии в $U(1)$-сектор при таком переходе $\Rightarrow$ радиовсплеск.
\end{itemize}

\subsection*{Магнетары}
Сверхсильные магнитные поля объясняются резонансом $U(1)$-сектора. Их устойчивость --- следствие спектрального механизма, а не классической динамо-модели.

\section*{Новые предсказания ZFSC}

\begin{itemize}
  \item \textbf{Спектральные глитчи в гравитационных волнах:} дополнительные пики в сигнале при слиянии нейтронных звёзд.
  \item \textbf{Повторяющиеся FRB:} переходы туда-обратно между соседними плато $\Rightarrow$ многократные всплески.
  \item \textbf{Медленные дыхательные моды:} массивные нейтронные звёзды должны испытывать долгопериодические вариации радиуса (порядка секунд--минут), связанные с колебаниями нижних мод.
  \item \textbf{Сдвиги EOS:} возможны измеримые различия в зависимости $\sigma_{\pi N}$, которые можно проверять через лабораторные эксперименты по рассеянию.
\end{itemize}

\section*{Заключение}
ZFSC даёт единый язык для описания нейтронных звёзд.  
Эффекты, вызывающие вопросы в классической астрофизике (массы $>2.3M_\odot$, магнетары, FRB, глитчи), естественно объясняются через спектральные плато, дополнительные моды и резонансы в $U(1)$-секторе.  

Теория предсказывает новые наблюдаемые явления: спектральные особенности гравитационных волн, повторные FRB, дыхательные моды массивных звёзд. Их обнаружение может стать прямым тестом ZFSC.

\vspace{2em}
\noindent
\textit{Evgeny Monakhov} \\
Independent Researcher \\
VOSCOM ONLINE

\end{document}
