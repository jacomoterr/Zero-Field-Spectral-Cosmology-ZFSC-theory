\documentclass[a4paper,12pt]{article}
\usepackage{geometry}
\geometry{margin=2.5cm}

% Языки
\usepackage[russian,english]{babel}

% --- Шрифты ---
\usepackage{fontspec}
\usepackage{unicode-math}
\setmainfont{CMU Serif}
\setmathfont{Latin Modern Math}

% --- Математика ---
\usepackage{amsmath}
\usepackage{braket}
\everymath{\displaystyle}
\emergencystretch=2em

\begin{document}

\section*{Как собираются звёзды, галактики и Вселенная: взгляд астрофизики и ZFSC}

\subsection*{Космическая паутина}
Когда мы смотрим на распределение галактик во Вселенной, оно напоминает гигантскую паутину: тонкие нити, узлы и пустоты. Эта структура возникла из крошечных квантовых флуктуаций в первые мгновения после Большого взрыва, которые постепенно усиливались за счёт гравитации. В местах, где плотность была чуть выше средней, гравитация собирала больше вещества, и такие ``узлы'' становились зародышами будущих галактик.

\subsection*{Роль гравитации}
Гравитация --- главный архитектор Вселенной. Именно она заставляет газовые облака сжиматься, формировать звёзды, скопления и галактики. Без гравитации вещество осталось бы равномерным туманом. Но гравитация умеет усиливать даже малейшие различия в плотности, превращая их в космические города из миллиардов звёзд.

\subsection*{Магнитные поля: спутники структуры}
В современной космологии есть гипотеза: зарождение звёзд и галактик связано с областями сильных магнитных полей. В действительности магнитные поля --- не строители, а скорее дирижёры, которые помогают оркестру гравитации.  
Они направляют плазму, формируют вращающиеся диски, помогают выбрасывать джеты --- узкие струи вещества, уходящие в космос. Наблюдения показывают, что нити космической паутины и области сильных магнитных полей часто совпадают. Поэтому иногда создаётся впечатление, что именно поля управляют сборкой. Но глубже это один и тот же процесс: узлы структуры порождают и звёзды, и поля.

\subsection*{Тёмная материя: невидимый каркас}
Особая роль принадлежит тёмной материи. Хотя мы её не видим, именно она создаёт ``скелет'' Вселенной. Тёмная материя формирует огромные гало и нити, в которых потом собирается обычное вещество --- газ и звёзды.  
Без тёмной материи галактики не могли бы образоваться: гравитация обычного вещества слишком слаба, чтобы удержать звёзды вместе на таких масштабах. То есть звёзды и галактики сидят на невидимом каркасе, который прячет тёмная материя.

\subsection*{Как это объясняет ZFSC}
Zero-field Spectral Cosmology предлагает другой взгляд на эти процессы:
\begin{itemize}
  \item В основе --- матричная связность, огромная сеть возможных состояний, где каждый узел --- потенциальный каркас будущей структуры.
  \item Гравитация --- проявление самой глубокой, почти нулевой моды этой матрицы. Именно она стягивает вещество в узлы.
  \item Магнитные поля --- это проекции $U(1)$-сектора связности. Они появляются там, где сеть сама задаёт особые узлы. Поэтому совпадение областей сильных магнитных полей и рождений звёзд не случайно: это разные стороны одного механизма.
  \item Тёмная материя в ZFSC --- не ``невидимая частица'', а побочный эффект дополнительного спектра мод, которые не взаимодействуют напрямую с обычным веществом, но формируют геометрию пространства и поддерживают каркас структуры.
\end{itemize}

\subsection*{На человеческом языке}
Представьте строительные леса. Сначала возводится невидимый каркас --- это роль тёмной материи. Потом к нему прикрепляются балки --- обычное вещество, которое собирается под действием гравитации. И, наконец, проводка и кабели --- магнитные поля --- которые помогают распределять энергию и организуют рабочее пространство.  
И хотя провода и балки видны, главный архитектор всё равно --- невидимый каркас.

\subsection*{Будущее структур}
Что ждёт галактики и звёзды в далёком будущем?
\begin{itemize}
  \item Звёзды будут постепенно выгорать, превращаясь в белые карлики, нейтронные звёзды и чёрные дыры.
  \item Галактики будут сливаться друг с другом, образуя всё более крупные системы.
  \item Тёмная материя продолжит удерживать эти структуры, хотя обычного вещества станет всё меньше в активной форме.
  \item Магнитные поля будут сохраняться как ``отголоски'' связности, но без ярких источников плазмы они станут слабее и рассеяннее.
\end{itemize}

В модели ZFSC это выглядит как постепенное ``замораживание'' спектра: матрица сохраняет свою структуру, но активные возбуждения уходят. Вселенная остаётся паутиной связей, но она всё более пуста в энергетическом смысле.

\subsection*{Итог}
\begin{itemize}
  \item Звёзды и галактики образуются не в магнитных полях, а в узлах гравитационной структуры.
  \item Магнитные поля --- верный спутник и помощник, но не причина.
  \item Тёмная материя --- фундаментальный каркас, без которого архитектура Вселенной не состоялась бы.
  \item В теории ZFSC все эти явления объединяются: и гравитация, и магнитные поля, и тёмная материя --- это проявления разных слоёв единой матричной структуры, которая разворачивается во времени.
\end{itemize}

\end{document}
