\documentclass[12pt,a4paper]{article}
\usepackage[utf8]{inputenc}
\usepackage[T2A]{fontenc}
\usepackage[russian]{babel}
\usepackage{amsmath,amssymb}
\usepackage{hyperref}
\usepackage{geometry}
\usepackage{verbatim} % для блока цитирования
\geometry{margin=2.5cm}
\usepackage[T2A]{fontenc}
\usepackage[utf8]{inputenc}
\usepackage[russian]{babel}
\usepackage{braket}

\title{Эволюция идей: от Стандартной модели к теории струн и ZFSC}
\author{Евгений Монахов \\ LCC ``VOSCOM ONLINE'' Research Initiative}
\date{Сентябрь 2025}

\begin{document}
\maketitle

\section*{Сравнительная схема}

\begin{tabular}{p{4cm}p{5cm}p{5cm}}
\toprule
\textbf{Стандартная модель (SM)} & \textbf{Теория струн} & \textbf{ZFSC (Zero-Field Spectral Cosmology)} \\
\midrule
Феноменологическая теория, массы и углы задаются параметрами & Струны как фундаментальные объекты, спектр из колебаний & Фрактальная эрмитова матрица $H$, спектр из собственных значений \\
\midrule
Калибровочные группы $SU(3)\times SU(2)\times U(1)$ --- постулируются & Естественно возникают из симметрий струн и D-brane конфигураций & Симметрии $\approx$ распределения связности матрицы (графовые инварианты) \\
\midrule
CKM/PMNS вводятся «сверху» как параметры & Возникают из геометрии пересечений D-brane, форм CY & CKM/PMNS = разные геометрии связей одного $H$ \\
\midrule
Массы фермионов из Юкав, подгоняются & Моды колебаний струн в CY-компактификациях & Устойчивые \emph{плато} собственных значений $H$ \\
\midrule
Нет гравитации в рамках SM & Гравитация = возбуждение струны (гравитон) & Гравитон = почти нулевая мода матрицы $H$ \\
\midrule
Тёмная материя и энергия вводятся извне & Возможны моды из скрытых измерений, но не наблюдаются & Тёмная материя = аксионные моды («золотая лестница»), тёмная энергия = вклад тахионов и асимметрии \\
\midrule
Нет топологии в явном виде & Используются многообразия CY, топологические инварианты, mirror symmetry & Фрактальная дискретная топология матрицы; аналог потоков Риччи (Перельман) стабилизирует плато \\
\bottomrule
\end{tabular}

\section*{Итог}
ZFSC можно рассматривать как \emph{дискретизированный каркас} струнной теории:  
без необходимости постулировать струны и 10 измерений, но с сохранением математики (CY, модули, топология).  
Это объединяет точность SM, геометрию струн и простоту спектральной модели.

\end{document}
