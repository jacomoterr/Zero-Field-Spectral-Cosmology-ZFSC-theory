\documentclass[12pt,a4paper]{article}
\usepackage[utf8]{inputenc}
\usepackage[T2A]{fontenc}
\usepackage[russian]{babel}
\usepackage{amsmath,amssymb}
\usepackage{hyperref}
\usepackage{geometry}
\usepackage{verbatim} % для блока цитирования
\geometry{margin=2.5cm}
\usepackage[T2A]{fontenc}
\usepackage[utf8]{inputenc}
\usepackage[russian]{babel}
\usepackage{braket}

\title{Возникновение массы и гравитации \\
из гипотезы вероятностного поля}
\author{Евгений Монахов \\ VOSCOM Research Initiative}
\date{Сентябрь 2025}

\begin{document}
\maketitle

\begin{abstract}
Предлагается описание механизма, при котором масса и гравитация не постулируются отдельно, а возникают как следствия спектральной структуры вероятностного поля на нулевом уровне энтропии. Масса трактуется как ``замороженная'' энергия отдельных мод, а гравитация --- как глобальное искажение эффективной метрики (сети связей), вызванное совокупным вкладом этих мод. Обсуждаются формулы-игрушки и возможные пути проверки.
\end{abstract}

%------------------------------------
\section{Масса как энергия моды}
В рамках гипотезы каждая мода графа имеет собственную частоту:
\[
\omega_k = \sqrt{\lambda_k},
\]
где $\lambda_k$ --- собственные значения лапласиана $L$ на графе связей (CY-Links).

Энергия моды:
\[
E_k = \hbar \omega_k \left(n_k + \tfrac{1}{2}\right).
\]

При разворачивании в пространство-время эта энергия реализуется как масса:
\[
m_k = \frac{E_k}{c^2}.
\]

Таким образом, масса частицы не вводится вручную, а является результатом декогеренции и ``замораживания'' энергии моды.

%------------------------------------
\section{Иерархия масс}
Почти вырожденные моды ($\omega_i \approx \omega_j \approx \omega_k$) могут соответствовать поколениям частиц. Небольшие расщепления $\Delta\omega$ приводят к экспоненциальным различиям в эффективных массах, что согласуется с наблюдаемой иерархией:
\[
m_i \sim \frac{\hbar \omega_i}{c^2}, 
\qquad
\Delta m \sim \exp(-\alpha \, \Delta \omega).
\]

%------------------------------------
\section{Гравитация как коллективный эффект}
Пространство возникает как сеть связей. Массы (энергии мод) влияют на структуру сети, изменяя спектр лапласиана. Это эквивалентно искривлению метрики.

Эффективное тензорное уравнение:
\[
R_{\mu\nu} - \tfrac{1}{2} R g_{\mu\nu} \;\;\sim\;\; 
\sum_{k} \langle T_{\mu\nu}^{(k)} \rangle ,
\]
где $\langle T_{\mu\nu}^{(k)} \rangle$ --- вклад моды $\omega_k$ в локальную структуру сети.

%------------------------------------
\section{Гравитационная постоянная}
Гравитационная константа может быть связана с суммарным спектром:
\[
G^{-1} \;\sim\; \sum_{k} \hbar \omega_k ,
\]
что напоминает зависимость $G$ от спектра в теориях струн. Таким образом, $G$ --- не фундаментальная величина, а коллективный параметр, зависящий от набора развёрнутых мод.

%------------------------------------
\section{Интерпретация}
\begin{itemize}
  \item Масса = энергия развёрнутой моды.
  \item Иерархия масс = эффект почти вырожденных мод.
  \item Гравитация = глобальная реакция сети (эффективной метрики) на распределение энергий мод.
  \item $G$ = коллективная константа, зависящая от спектра.
\end{itemize}

%------------------------------------
\section{Возможные проверки}
\begin{enumerate}
  \item Попытка вывести порядок величин $m_{\rm Higgs}, \alpha, \alpha_s$ через спектр $\omega_k$.
  \item Проверка устойчивости $G$ к изменениям в спектре (связь с космологическими данными о постоянстве $G$).
  \item Численное моделирование спектральных графов для проверки: 
  \begin{itemize}
    \item появляется ли триплет почти вырожденных мод (три поколения),
    \item стабилизируется ли $D_{\rm eff}\approx 3$,
    \item воспроизводится ли подавление $\rho_\Lambda$.
  \end{itemize}
\end{enumerate}

%------------------------------------
\section{Заключение}
В гипотезе вероятностного поля масса и гравитация возникают не как отдельные постулаты, а как естественные следствия спектральной структуры. Массы рождаются из энергии отдельных мод, а гравитация --- как коллективное искажение метрики. Данный подход позволяет рассматривать фундаментальные константы как производные от более глубокой вероятностной структуры.
\end{document}
