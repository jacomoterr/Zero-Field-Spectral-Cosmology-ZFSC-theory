%------------------------------------
\section{Первые проверки: три поколения и коэффициенты иерархии $c$}
\label{sec:first-checks}

\subsection{Определение сектора и отображение спектра в массы}
Пусть в секторе $f\in\{\nu,\ell,u,d\}$ матрица связей/гамильтониан $H_f$ имеет
три устойчивые положительные собственные значения
\[
\lambda^{(f)}_1 < \lambda^{(f)}_2 < \lambda^{(f)}_3,
\]
которые соответствуют трём поколениям фермионов сектора $f$.
Мы рассматриваем аффинное отображение уровня спектра в «массу-квадрат»:
\[
m^{(f)\,2}_k \;=\; A_f\,\lambda^{(f)}_k + B_f,\qquad k=1,2,3,
\]
где $A_f>0$ — масштабный множитель (единицы, калибровка),
а $B_f$ — возможный секторный сдвиг (регуляризация нулевого уровня).
Это допускает модели, где для нейтрино непосредственно сопоставимы
\(\lambda_k \leftrightarrow m_k^2\) (измеримы именно разности \(m^2\)).

\vspace{4pt}
\noindent\textbf{Иерархические «зазоры» спектра.}
Определим \emph{последовательные зазоры} (в любом из представлений \(\lambda\) или \(m^2\)):
\[
\Delta^{(f)}_1 := \lambda^{(f)}_2 - \lambda^{(f)}_1,\qquad
\Delta^{(f)}_2 := \lambda^{(f)}_3 - \lambda^{(f)}_2.
\]
Тогда при аффинном преобразовании \(\lambda\mapsto A_f \lambda + B_f\) имеем
\(\Delta \mapsto A_f \Delta\). Это означает, что \emph{отношение зазоров}
\[
\boxed{\quad c_f \;:=\; \frac{\Delta^{(f)}_2}{\Delta^{(f)}_1}
\;=\; \frac{\lambda^{(f)}_3 - \lambda^{(f)}_2}{\lambda^{(f)}_2 - \lambda^{(f)}_1}
\;=\; \frac{m^{(f)\,2}_3 - m^{(f)\,2}_2}{m^{(f)\,2}_2 - m^{(f)\,2}_1}\quad}
\]
\emph{инвариантно} относительно любых аффинных перенормировок (\(A_f,B_f\)).
Именно эта аффинная инвариантность делает \(c_f\) удобным «маркером геометрии» сектора $f$.

\subsection{Дискретная «кривизна» спектра и физический смысл $c_f$}
Определим дискретную вторую разность (аналог кривизны последовательности уровней):
\[
\kappa_f \;:=\; \frac{\lambda^{(f)}_3 - 2\lambda^{(f)}_2 + \lambda^{(f)}_1}{\lambda^{(f)}_2 - \lambda^{(f)}_1}
\;=\; \frac{\Delta^{(f)}_2 - \Delta^{(f)}_1}{\Delta^{(f)}_1}
\;=\; \frac{\Delta^{(f)}_2}{\Delta^{(f)}_1} - 1
\;=\; c_f - 1.
\]
Тем самым
\[
\boxed{\quad c_f \,=\, 1+\kappa_f \quad}
\]
и величина \(c_f\) непосредственно измеряет «ускорение» разрежения уровней:
\(c_f>1\) означает, что верхний зазор существенно шире нижнего.
Физически это читается как \emph{луковичная} (вложенная) геометрия сектора:
чем больше \(c_f\), тем круче возрастает «масштаб» при переходе от первого поколения к третьему.
Для нейтрино \(c_\nu\) умеренно велико — «мягкая иерархия»;
для кварков up/down \(c_u,c_d\) огромны — «жёсткая иерархия»;
для заряженных лептонов \(c_\ell\) — промежуточно-жёсткая.

\subsection{Экспериментальные значения \(c_f^{\rm exp}\)}
Для \emph{заряженных лептонов} используем точные массы
\((m_e, m_\mu, m_\tau)\) и определение \(c_\ell\) через квадраты масс:
\[
c_\ell^{\rm exp} \;=\; \frac{m_\tau^2 - m_\mu^2}{m_\mu^2 - m_e^2}
\;\approx\; 281.82.
\]
(Число приведено при современных значениях PDG.)\footnote{PDG (2024): «Particle Properties» и сводный обзор RPP.}

Для \emph{нейтрино} доступны именно \(\Delta m^2\), поэтому естественно положить
\[
\boxed{\quad c_\nu^{\rm exp} \;:=\; 
\frac{\lvert\Delta m^2_{31}\rvert}{\Delta m^2_{21}} 
\;\approx\; 33.9 \quad}
\]
(нормальная иерархия; аналогично в инвертированной иерархии через \(\Delta m^2_{32}\)).\footnote{NuFIT, глобальные подгонки параметров осцилляций; см. также PDG (2023/2024) «Neutrino Masses, Mixing, and Oscillations».}

Для \emph{up-кварков} (схема $\overline{\rm MS}$, общепринятая шкала) оценка
\[
c_u^{\rm exp} \;=\; \frac{m_t^2 - m_c^2}{m_c^2 - m_u^2}
\;\approx\; 1.85\times 10^4,
\]
для \emph{down-кварков}
\[
c_d^{\rm exp} \;=\; \frac{m_b^2 - m_s^2}{m_s^2 - m_d^2}
\;\approx\; (1.9\text{--}2.1)\times 10^3
\]
в зависимости от выбранных центральных значений $m_s$ (сильная чувствительность к $m_s$).\footnote{PDG (2024): «Quark masses» (обзор), учёт схемы/шкалы; разброс по $m_s$ даёт вклад в интервал \(c_d^{\rm exp}\).}

\subsection{Результаты модели (ZFSC v6.2) и сравнение}
При общих структурных параметрах (для всех четырёх секторов)
\[
g_L=5.0,\qquad g_R=0.1,\qquad h_1=1.5,\quad h_2=-1.0,\quad h_3=0.7,
\]
мы получили\footnote{Числа из наших серий расчётов (режим {\tt independent\_all}, плотные сетки по \(\Delta,r\)).}:
\[
\begin{aligned}
c_\nu^{\rm model} &\,=\, 33.92 \pm 1.02,\\
c_\ell^{\rm model} &\,=\, 282.819067345,\\
c_u^{\rm model} &\,=\, 1.849177\times 10^4,\\
c_d^{\rm model} &\,=\, 2025.268478300.
\end{aligned}
\]
Сопоставление с экспериментом (центральные оценки):
\[
\begin{array}{c|c|c|c}
\text{Сектор} & c^{\rm exp} & c^{\rm model} & \text{Отн. отклонение}\\\hline
\nu      & 33.9 & 33.92\ (\pm 1.02) & < 1\%\\
\ell     & 281.82 & 282.82 & \sim 0.35\%\\
u        & 1.85\times 10^4 & 1.849\times 10^4 & \lesssim 0.1\%\\
d        & (1.9\text{--}2.1)\times 10^3 & 2.025\times 10^3 & \text{в пределах схемной неопределённости}
\end{array}
\]
Важно, что \emph{все четыре} \(c_f\) воспроизводятся \emph{одними и теми же}
структурными параметрами \(g_L,g_R,h_i\), что указывает на общую геометрию «луковичных уровней»,
а не на подгонку по секторам.

\subsection{Устойчивость и инвариантность}
\paragraph{Аффинная инвариантность.}
По определению \(c_f = \Delta_2/\Delta_1\) не чувствителен ни к \(A_f\), ни к \(B_f\). 
Это означает: независимо от абсолютной калибровки шкалы масс (или от сдвига вакуумного уровня),
иерархический профиль спектральных зазоров (а значит и геометрия «слоёв») измеряется \(c_f\).

\paragraph{Ренормгрупповая чувствительность.}
Для кварков \(c_u,c_d\) чувствительны к выбору схемы и шкалы масс
(например, \(\overline{\rm MS}\) при \(\mu=2\) GeV или другая).
Однако, будучи \emph{отношениями зазоров} по \(m^2\), величины \(c_f\) 
в окуляре ZFSC отражают геометрию низших мод, а не их абсолютные значения.
Практически: изменения в \(m_s\) на несколько процентов способны сдвигать \(c_d\) на десятки процентов,
что и объясняет наблюдаемый интервал для \(c_d^{\rm exp}\) в литературе.

\subsection{Физические выводы для ZFSC}
\begin{enumerate}
  \item Наличие трёх устойчивых положительных мод \(\lambda^{(f)}_1,\lambda^{(f)}_2,\lambda^{(f)}_3\) естественно объясняет \emph{три поколения}.
  \item Величины \(c_f\) кодируют геометрию низших слоёв: 
  \(c_\nu\) умеренно велико (мягкая иерархия), \(c_\ell\) — промежуточное, \(c_{u,d}\) — крайне велики (жёсткие иерархии).
  \item Совпадение \(c_f^{\rm model}\) с \(c_f^{\rm exp}\) при общих параметрах \(g_L,g_R,h_i\) — 
  сильный тест \emph{геометрической} природы спектра \(H\).
  \item Инвариантность \(c_f\) к аффинной калибровке объясняет устойчивость результата при различных способах нормировки масс.
\end{enumerate}

\subsection{Примечания и потенциальные источники}
\begin{itemize}
  \item Сводные данные по массам лептонов/кварков и обзоры см.: 
  PDG, \emph{Review of Particle Physics} (2024), разделы «Particle Properties», «Quark Masses», «CKM matrix».%
  \footnote{Particle Data Group (RPP 2024), включая страницы \emph{Particle Properties} и обзор по массам кварков; см. также обзор по CKM.}
  \item Параметры осцилляций нейтрино (соотношение \(\Delta m^2\)) — глобальные подгонки NuFIT, а также соответствующие разделы PDG.%
  \footnote{NuFIT (актуальная версия), сводные таблицы \(\Delta m^2_{21}, \Delta m^2_{31}\), углы смешивания; см. также PDG «Neutrino masses, mixing, and oscillations».}
\end{itemize}
