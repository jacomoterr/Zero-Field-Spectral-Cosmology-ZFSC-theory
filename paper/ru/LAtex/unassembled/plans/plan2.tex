\documentclass[a4paper,12pt]{article}
\usepackage{geometry}
\geometry{margin=2.5cm}

\usepackage{fontspec}
\usepackage{unicode-math}
\setmainfont{CMU Serif}
\setmathfont{Latin Modern Math}

\usepackage{mathtools}
\usepackage{physics}
\everymath{\displaystyle}

\title{ZFSC: План работ по подтверждени теории.\\ Часть 2. Развёрнутый план исследований для LHC и CMB.}
\author{Евгений Монахов}
\date{}

\begin{document}
\maketitle

\section*{Общая цель}
Подтвердить Zero-Field Spectral Cosmology (ZFSC) через независимые эксперименты:
\begin{enumerate}
  \item Микромир (массы, CKM/PMNS, $\alpha$).  
  \item Макромир (реликтовое излучение, космология).  
\end{enumerate}

---

\section*{I. Работа со стандартной моделью и LHC}
\subsection*{1. Численные расчёты в ZFSC}
\begin{itemize}
  \item Построить спектры для всех секторов ($u,d,\ell,\nu$).  
  \item Проверить трёхуровневую иерархию масс поколений.  
  \item Получить CKM и PMNS матрицы из собственных векторов.  
  \item Вынести геометрическое определение $\alpha$.
\end{itemize}

\textit{Простыми словами:}  
Мы ищем в матрице $H$ те же закономерности, которые физики видят в экспериментах.

\subsection*{2. Что показываем коллаборациям}
\begin{itemize}
  \item Соотношения масс ($m_t/m_c/m_u$, $m_\tau/m_\mu/m_e$).  
  \item Структура CKM (малые углы) и PMNS (большие углы).  
  \item Предсказания по лептонной универсальности.  
\end{itemize}

\textit{Простыми словами:}  
Мы им не даём всю теорию, а только «кусочки совпадений». Если они подтверждаются — это зацепка.

\subsection*{3. Потенциальные «вау»-результаты}
\begin{itemize}
  \item Естественное происхождение трёх поколений фермионов.  
  \item Геометрическая формула для $\alpha$.  
  \item Намёки на тахионные моды (новая физика).
\end{itemize}

---

\section*{II. Работа с космологами (CMB, реликтовое излучение)}
\subsection*{1. Предсказания ZFSC}
\begin{itemize}
  \item Возраст Вселенной больше, чем в $\Lambda$CDM:  
  $t(z)=t_{\Lambda CDM}(z)+\gamma \ln(1+z)$.  
  \item Слабая эволюция $G_{\text{eff}}(t)$ (гравитационная константа).  
  \item Фрактальные паттерны в CMB.  
\end{itemize}

\textit{Простыми словами:}  
По нашей теории Вселенная чуть старше, $G$ немного меняется, а в CMB остаются «узоры», которые стандартная модель не объясняет.

\subsection*{2. Что делаем сами}
\begin{itemize}
  \item Скачиваем карты Planck, ACT, SPT (доступны открыто).  
  \item Строим свои статистики: мультипольные моменты, корреляции, фрактальные признаки.  
  \item Проверяем отклонения от $\Lambda$CDM.
\end{itemize}

\subsection*{3. Что показываем коллаборациям}
\begin{itemize}
  \item Известные аномалии (ось зла, холодное пятно, низкий диполь).  
  \item Новые паттерны, если найдём при нашем анализе.  
  \item Указание: «здесь стандартная космология не объясняет данные».
\end{itemize}

\subsection*{4. Потенциальные «вау»-результаты}
\begin{itemize}
  \item Независимое подтверждение «растянутого» возраста Вселенной.  
  \item Обнаружение структуры в CMB, совпадающей с матричными предсказаниями ZFSC.  
  \item Новая интерпретация аномалий (они не случайность, а следствие спектральной структуры).
\end{itemize}

---

\section*{III. Тактика писем}
\begin{itemize}
  \item Отправляем по 1--2 письма в день, чередуя коллаборации.  
  \item Формат: «У нас есть математическая модель, которая предсказывает интересные совпадения».  
  \item Не называем ZFSC напрямую, пока не появится серия подтверждений.  
  \item Ждём независимой обратной проверки.
\end{itemize}

\textit{Простыми словами:}  
Мы «закидываем кости» — пусть их заинтересует, пусть проверят.  
Главное — не раскрывать всю теорию сразу.

---

\section*{IV. Концепт общей работы}
\begin{enumerate}
  \item Сначала микромир: массы, CKM/PMNS, $\alpha$.  
  \item Потом космология: CMB, возраст, $G_{\text{eff}}$.  
  \item Параллельно письма в CERN и космологические коллаборации.  
  \item Когда совпадения становятся устойчивыми — публикуем ZFSC как целое.
\end{enumerate}

\textit{Простыми словами:}  
Идём от малого к большому. Сначала частицы, потом космос.  
Сначала сами проверяем, потом — мировые коллаборации.

---

\section*{Финальная цель}
\begin{itemize}
  \item Подтверждение $\alpha$ как геометрической константы.  
  \item Три поколения и смешивания как естественный результат матрицы $H$.  
  \item Космологические аномалии объяснены в рамках ZFSC.  
\end{itemize}

\textbf{Если всё это сойдётся — мы действительно сделали шаг к Теории всего.}

\end{document}
