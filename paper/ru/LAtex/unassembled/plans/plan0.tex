\documentclass[a4paper,12pt]{article}
\usepackage{geometry}
\geometry{margin=2.5cm}

\usepackage{fontspec}
\usepackage{unicode-math}
\setmainfont{CMU Serif}
\setmathfont{Latin Modern Math}

\usepackage{mathtools}
\usepackage{physics}
\everymath{\displaystyle}

\title{ZFSC: Общий план работ по подтверждени теории.}
\author{Евгений Монахов}
\date{}

\begin{document}
\maketitle

\section*{Главная цель}
Построить и доказать Zero-Field Spectral Cosmology (ZFSC) через эксперименты и численные расчёты.  
Ключевой ориентир: вывести массы поколений частиц, матрицы смешивания CKM/PMNS, силы взаимодействий и строгое геометрическое определение постоянной тонкой структуры $\alpha$.  
Всё остальное (CMB, космология, $H_0$, возраст Вселенной) служит дополнительными проверками.

---

\section*{Этап I. Микромир — «лестница поколений»}
\subsection*{A. Спектры и массы}
\begin{itemize}
  \item Построить базовую матрицу $H(N, \Delta, r, g_L, g_R, h_1,h_2,h_3)$.
  \item Получить собственные значения $\lambda_n(H)$.
  \item Интерпретировать первые три устойчивые положительные моды как массы поколений (u/d, e/μ/τ, нейтрино).
\end{itemize}

\textit{Простыми словами:}  
Мы проверяем, что в матрице естественным образом рождаются «три этажа» масс частиц.

\subsection*{B. CKM и PMNS}
\[
\mathrm{CKM} = U_u^\dagger U_d, 
\quad
\mathrm{PMNS} = U_\ell^\dagger U_\nu .
\]

\textit{Простыми словами:}  
Берём «направления» (собственные векторы) из разных секторов.  
Если они не совпадают, рождается смешивание. Мы ожидаем:
малые углы для кварков (CKM), большие углы для нейтрино (PMNS).

\subsection*{C. Геометрическое $\alpha$}
\[
\alpha^{-1} = \mathcal{F}(\text{геометрия $H$ и слоёв})
\]

\textit{Простыми словами:}  
Хотим вывести $1/\alpha \approx 137$ как математическую константу.  
Это будет главный «вау-результат» всей работы.

---

\section*{Этап II. Сотрудничество с CERN и LHC}
\subsection*{A. Письма и «кости»}
\begin{itemize}
  \item Не раскрывая всю теорию, отправлять запросы в коллаборации (ATLAS, CMS, LHCb).  
  \item Ставить акценты: «есть математический метод, который предсказывает иерархии масс и смешивания».  
  \item Просить проверить отдельные численные совпадения.
\end{itemize}

\subsection*{B. Проверки на LHC}
\begin{itemize}
  \item Сравнение иерархий поколений ($m_t/m_c/m_u$, $m_\tau/m_\mu/m_e$).
  \item Углы CKM и PMNS.  
  \item Возможные аномалии в редких распадах ($B$-мезоны, лептонная универсальность).
\end{itemize}

\textit{Простыми словами:}  
Мы хотим, чтобы LHC подтвердил: «да, предсказания масс и смешивания совпадают с нашими данными».

---

\section*{Этап III. Космология и реликтовое излучение (CMB)}
\subsection*{A. Наш прогноз}
ZFSC предсказывает:
\begin{itemize}
  \item слегка изменённый возраст Вселенной ($t(z) = t_{\Lambda CDM}(z) + \gamma\ln(1+z)$),
  \item медленную эволюцию $G_{\text{eff}}(t)$,
  \item фрактальные паттерны в анизотропиях CMB.
\end{itemize}

\textit{Простыми словами:}  
Если мы правы, в CMB должны быть «необъяснимые шероховатости» — они уже есть, но стандартная космология их игнорирует.

\subsection*{B. Данные коллабораций}
\begin{itemize}
  \item Планк (Planck, ESA).  
  \item ACT (Atacama Cosmology Telescope).  
  \item SPT (South Pole Telescope).  
\end{itemize}

Мы можем скачать их открытые карты и проверить нашу гипотезу напрямую.

---

\section*{Этап IV. Тактика работы}
\begin{enumerate}
  \item \textbf{Самостоятельные проверки.} Сначала считаем сами: массы, CKM/PMNS, α, CMB-анализ.  
  \item \textbf{«Кости» коллаборациям.} Посылаем частичные результаты, не раскрывая полной картины.  
  \item \textbf{Независимая верификация.} Когда несколько независимых коллабораций подтверждают совпадения — публикуем ZFSC.  
\end{enumerate}

\textit{Простыми словами:}  
Сначала сами убеждаемся, что работает. Потом даём учёным проверять куски. И только когда совпадения устойчивы — открываем всю теорию.

---

\section*{Финальная цель}
\begin{itemize}
  \item Математическое и экспериментальное подтверждение ZFSC.  
  \item Главный результат: $\alpha$ как геометрическая константа.  
  \item Побочный бонус: новая космология с объяснением CMB и $H_0$.  
\end{itemize}

\textbf{Итог:}  
Если всё это подтверждается — мы сделали шаг к «Теории всего».  

\end{document}
