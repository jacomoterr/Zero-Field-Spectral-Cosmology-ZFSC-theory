

\begin{table}[h!]
\centering
\small
\setlength{\tabcolsep}{6pt}
\renewcommand{\arraystretch}{1.25}
\begin{tabular}{p{3.4cm} p{3.6cm} p{6.2cm} p{4.2cm}}
\toprule
\textbf{Параметр (SM)} & \textbf{Как задаётся в SM} & \textbf{Происхождение в ZFSC} & \textbf{Как считаем в сканах (кратко)}\\
\midrule

Массы кварков $u,d,s,c,b,t$ &
Юкавовы константы + Хиггс (подгонка) &
Собственные значения секторных матриц $H_{u},H_{d}$ (геометрические трансформации $T_{u/d}$ базовой $H$); плато = устойчивые кластеры $\lambda_i$ &
Решить $H_{u/d}\psi=\lambda\psi$; сопоставить плато с экспериментом, минимизируя невязку по массам иерархий \\

Массы лептонов $e,\mu,\tau$ &
Юкавовы (подгонка) &
Собственные значения $H_{\ell}$; плато лептонов как фрактальные ступени одного $H$ &
Аналогично кваркам; совместный фит с нейтрино по единому $H$ \\

Нейтринные массы/сплиттинги $\Delta m^2$ &
Вводятся феноменологически (за рамками SM) &
Нижние собственные значения $H_\nu$; $\Delta m^2$ = разности соседних $\lambda_i$ (с учётом $c$) &
Точная экстракция малых зазоров; регуляризация отрицательных мод; проверка осцилляционных данных \\

CKM: углы $\theta_{12},\theta_{23},\theta_{13}$ и фаза $\delta_{\rm CKM}$ &
Параметры матрицы смешивания (подгонка) &
$V_{\rm CKM} = U_u^\dagger U_d$, где $U_{u/d}$ — матрицы собственных векторов $H_{u/d}$; «почти единичность» из близких геометрий $T_u\approx T_d$ &
Диагонализовать $H_{u/d}$, построить $U_{u/d}$, затем CKM; минимизировать невязку углов и фазы \\

PMNS: углы и фазы (в т.ч. Majorana) &
Феноменологически &
$U_{\rm PMNS}=U_\ell^\dagger U_\nu$; большие углы из существенно разных геометрий $T_\ell$ и $T_\nu$ &
Диагонализовать $H_{\ell},H_{\nu}$; построить PMNS и фит осцилляций \\

Константы калибровочных групп $g_s,g,g'$ &
Фиксируются на масштабе (ренормгруппа) &
Из \emph{статистики связности} фрактальной матрицы: распределения координационных чисел/мультисвязей $\to$ эффективные SU(3), SU(2), U(1) купплинги &
Оценка через графовые метрики (степени/кластерность) + калибровка на известных процессах \\

Угол $\theta_{\rm QCD}$ (strong CP) &
Свободный параметр (почти ноль) &
Малая фаза из слабого C/CP-нарушения верхних слоёв (тахион/аксион) $\phi_{\tau,a}\to$ эффективная $\theta_{\rm eff}$, подавленная запутанностью &
Экстрагировать фазовые вклады из блочного $H_{\rm ext}$; обеспечить малость в фитах \\

Масса Хиггса и VEV $v$ &
Потенциал Хиггса (постулируется) &
Низкая бозонная мода (бозонный слой) и масштаб когерентности: $v$ как эффективный параметр связности/конденсата слоя; $m_H$ — ближайший «бозонный» собственный уровень &
Выделить бозонный слой, найти низшие $\lambda$; совместно откалибровать с массами векторных бозонов \\

Массы $W,Z$ &
Из $v$ и купплингов &
Бозонные моды $H_{\rm bos}$ с геометрическими проекциями на SU(2)$\times$U(1) подпространства &
Проецирование спектра на подалгебры; согласование с $\sin^2\theta_W$ \\

ЭМ-константа $\alpha$ &
Эксп. вход + рг-эволюция &
Эффективная плотность связей U(1)-подпространства (статистика длин/мощностей связей) &
Калибровка на низкоэнергетических процессах; проверка рг-потока как функции масштаба фрактала \\

Фазы CP в CKM/PMNS &
Свободные &
Геометрические фазы собственных векторов из различий $T_s$ + вклад верхних слоёв через $\Delta_X$ &
Из $U_s$ извлечь инварианты Джарлскога; фит к распадам/осцилляциям \\

Плотность тёмной материи $\Omega_{\rm DM}$ &
Вводится за рамками SM &
Вклад \emph{высоких мод} и аксионного слоя (фрактальная «золотая лестница»), проекции на фотон/нуклонные каналы &
По $\Delta\lambda_k\sim \Lambda_0^2\varphi^{2k}$ прикинуть $m_a$ и плотности; сравнить с окнами экспериментов \\

Космологическая константа $\Lambda$ (вакуумная энергия) &
Вводится феноменологически &
Квадратичные вклады $\Delta\rho_{\rm vac}$ от расщеплений $\,\delta_g^{(\tau,a)}$ (слабое C/CP-нарушение) + «залечивание» отрицательных мод &
Посчитать $\Delta\rho_{\rm vac}$ из $H_{\rm ext}$ с отсечкой; зафиксировать в космологическом окне \\

Нейтринные Majorana-фазы &
Неопределены в SM &
Фазовая структура $U_\nu$ из спектра $H_\nu$ и межслойной запутанности; возможные связи с аксионным слоем &
Из $U_\nu$ извлечь фазы; проверка в ββ0ν-поисках (в модели — через проекции) \\

Спектр гравитона (нулевая мода) &
Нет в SM &
Почти нулевая собственная мода бозонного слоя (граница когерентного переноса) &
Идентифицировать минимальную $\lambda\approx0$ и её устойчивость в сканах \\
\bottomrule
\end{tabular}

%------------------------------------
\section{Почему простая матрица даёт совпадения}

Одним из удивительных фактов, выявленных в первых численных экспериментах, было то, что даже \emph{простая кубическая матрица} без дополнительных усложняющих факторов (запутанности, тахионных поколений, асимметрий и т.д.) уже демонстрировала \textbf{устойчивые плато собственных значений}, совпадающие с наблюдаемыми массами поколений фермионов.

\subsection*{1. Локальное приближение фрактала}
Фрактал обладает самоподобием: его локальные куски воспроизводят общую структуру в упрощённом виде.  
Если рассматривать малый участок фрактальной матрицы $H$, то он может быть приближен регулярной решёткой (кубом).  
Таким образом, простая матрица в ограниченном диапазоне размеров $N$ \emph{эффективно ведёт себя как кусок фрактала}.

\[
H_{\text{fractal}} \quad\longrightarrow\quad H_{\text{cube}} \quad \text{(локальное приближение)}.
\]

\subsection*{2. Устойчивость спектральных плато}
Ключевое свойство спектра фрактальных и высоко-симметричных матриц --- появление \emph{кластеров собственных значений}:
\[
\lambda_{i},\;\lambda_{i+1},\;\dots,\;\lambda_{i+k} \;\approx\; \Lambda_{\text{plateau}}.
\]
Это объясняется тем, что при высокой симметрии большое число состояний оказывается вырожденным или почти вырожденным.  
Видимое плато масс частиц в ZFSC является прямым проявлением этой устойчивости.  

Даже простая кубическая решётка, обладая симметрией, создаёт группировки собственных значений, которые \emph{в первом приближении совпадают с экспериментальными массами поколений}.

\subsection*{3. Бог как инженер: кубы и фракталы}
Можно рассматривать это как принцип дизайна:
\begin{itemize}
  \item На фундаментальном уровне \textbf{Бог использует простые ``кирпичики''} --- кубические структуры как локальные приближения.
  \item На глобальном уровне эти кирпичики \textbf{собираются в фрактальную архитектуру}, создавая золотые пропорции, самоподобие и гармонию.
\end{itemize}
Это аналогично тому, как из простых молекул льда рождаются сложные фрактальные снежинки: локально структура кубическая, но глобально --- эстетически сложный узор.

\subsection*{4. Совпадения и ограничения}
Таким образом:
\begin{enumerate}
  \item \emph{Совпадения масс в первых численных расчётах} --- естественное следствие устойчивости плато даже в простейших матрицах.
  \item Однако \emph{тонкие эффекты} (углы PMNS, вклад тахионов и аксионов, вакуумная энергия) \textbf{не могут быть объяснены только кубической структурой}. Для этого необходим полный фрактальный аппарат ZFSC.
  \item Кубическая матрица можно рассматривать как \emph{низкоуровневое приближение} фрактальной геометрии --- рабочую модель, подтверждающую базовую правильность теории.
\end{enumerate}

\subsection*{5. Итог}
То, что совпадения с экспериментом появляются уже на уровне кубической матрицы, подтверждает: \textbf{ZFSC не является искусственной подгонкой}. Даже простейшее приближение воспроизводит ключевые спектральные особенности, а дальнейшие ``навороты'' (запутанность, тахионные поколения, C/CP-асимметрия) лишь уточняют картину и объясняют тонкие детали.

Это можно выразить формулой:
\[
H_{\text{cube}} \;\approx\; H_{\text{fractal}} \quad \Rightarrow \quad
\text{правильные массы} \;+\; \mathcal{O}(\text{тонкие эффекты}),
\]
где $\mathcal{O}(\text{тонкие эффекты})$ — поправки, учитываемые в расширенной ZFSC.

\caption{Ключевые параметры SM и их происхождение в ZFSC. В ZFSC все численные значения не постулируются, а выводятся из спектра и геометрии базовой фрактальной матрицы $H$ с учётом запутанности и верхних слоёв (тахион/аксион) через блочный гамильтониан $H_{\rm ext}$.}
\end{table}
