\documentclass[12pt,a4paper]{article}
\usepackage[utf8]{inputenc}
\usepackage[T2A]{fontenc}
\usepackage[russian]{babel}
\usepackage{amsmath,amssymb}
\usepackage{hyperref}
\usepackage{geometry}
\usepackage{verbatim} % для блока цитирования
\geometry{margin=2.5cm}

\title{Фибоначчиевские структуры и предгеометрия в спектральной космологии нулевого поля (ZFSC)}
\author{Евгений Монахов \\ ООО «VOSCOM ONLINE» Research Initiative \\ ORCID: 0009-0003-1773-5476}
\date{Сентябрь 2025}

\begin{document}
\maketitle

\section*{Аннотация}
В статье предлагается расширение теории \emph{Zero-Field Spectral Cosmology} (ZFSC), включающее квантование самой геометрии через фибоначчиевские матрицы и квазипериодические структуры. Показано, что золотое сечение и последовательность Фибоначчи естественным образом возникают как устойчивые масштабы спектра, а также как основа для реконструкции предгеометрической матрицы. Обсуждаются связи с экспериментальными массами поколений фермионов и матрицами смешивания CKM/PMNS. Формулируется обратная спектральная задача: восстановление предгеометрической структуры по экспериментальным данным.

\section{Введение}
Теория ZFSC постулирует существование фундаментального предгеометрического уровня, где отсутствуют время и пространство, а энтропия стремится к нулю:
\[
S \to 0.
\]
На этом уровне Вселенная описывается вероятностным полем амплитуд:
\[
\Psi = \sum_{i} a_i |i\rangle ,
\]
где $\{|i\rangle\}$ — возможные конфигурации, а $a_i\in \mathbb{C}$ — их амплитуды.

В исходной формулировке геометрия трактовалась как фиксированная «сцена» для спектральных мод. В данной работе мы предлагаем расширение: квантование самой геометрии через \emph{фибоначчиевские матрицы}, что придаёт теории дополнительную согласованность и позволяет объяснить ряд наблюдаемых соотношений.

\section{Фибоначчи и золотое сечение как универсальные структуры}
Последовательность Фибоначчи
\[
F_{n+1} = F_n + F_{n-1}, \qquad F_0=0, F_1=1,
\]
обладает фундаментальным свойством:
\[
\lim_{n\to\infty} \frac{F_{n+1}}{F_n} = \varphi = \frac{1+\sqrt{5}}{2}.
\]

Золотое сечение $\varphi$ возникает во множестве физических систем:
\begin{itemize}
    \item устойчивые спирали роста в биологии и квазикристаллах,
    \item спектры квазипериодических гамильтонианов (фибоначчиева цепочка, модель Обри–Андре),
    \item самоподобные иерархии устойчивости в динамических системах.
\end{itemize}

В ZFSC это свойство проявляется как \emph{естественный масштаб для устойчивых плато собственных значений матрицы предгеометрии}.

\section{Фибоначчиевские матрицы}
Минимальная матрица, генерирующая ряд Фибоначчи, имеет вид:
\[
F = \begin{pmatrix} 1 & 1 \\ 1 & 0 \end{pmatrix}, \qquad
F^n = \begin{pmatrix} F_{n+1} & F_n \\ F_n & F_{n-1} \end{pmatrix}.
\]

Расширенные конструкции включают:
\begin{enumerate}
    \item \textbf{Фибоначчиев гамильтониан:}
    \[
    (H\psi)_n = \psi_{n+1} + \psi_{n-1} + V \chi_{\text{Fib}}(n)\psi_n ,
    \]
    где $\chi_{\text{Fib}}(n)$ — индикатор слова Фибоначчи.
    \item \textbf{Золотой граф-Лапласиан:} вложенные слои графа с числом узлов $\sim F_n$ и степенями вершин $\deg_{L+1}/\deg_L \to \varphi$.
    \item \textbf{q-Фибоначчи:} деформированные последовательности, позволяющие описывать отклонения от идеала $\varphi$.
\end{enumerate}

\section{Квантование геометрии}
Расширим пространство состояний:
\[
\Psi = \sum_{i,j} a_{ij}\, |i\rangle \otimes |j_{\text{geom}}\rangle ,
\]
где $|j_{\text{geom}}\rangle$ — квантованные состояния геометрии, описываемые фибоначчиевскими матрицами.

Тогда гамильтониан системы принимает вид:
\[
H_{\text{tot}} = H_{\text{matter}} \otimes I + I \otimes H_{\text{geom}} + H_{\text{int}}.
\]

Условие устойчивости спектра:
\[
H_{\text{tot}} \Psi = \Lambda \Psi ,
\]
где $\Lambda$ задаёт совокупный спектр, включающий как материальные, так и геометрические моды.

\section{Обратная спектральная задача}
Пусть заданы экспериментальные данные: массы поколений фермионов $\{m_k^{\rm exp}\}$ и матрицы смешивания $U_{\rm CKM}, U_{\rm PMNS}$.  
Ставим задачу: найти матрицу $H_{\text{geom}}$ такую, что
\[
H_{\text{tot}} | \psi_k \rangle = \lambda_k | \psi_k \rangle, \qquad \lambda_k \equiv (m_k^{\rm exp})^2,
\]
и собственные векторы дают требуемые перекрытия.

Это \emph{inverse eigenvalue problem}, известная в математике как задача восстановления матрицы Якоби по спектру. В нашем случае параметризация через фибоначчиевские структуры существенно сужает пространство решений.

\section{Возраст Вселенной и глубина аппроксимации}
Последовательность аппроксимантов $F_n/F_{n-1}$ естественно связывается с конечным возрастом Вселенной. Пусть $n_*(t)$ — глубина развёртки:
\[
n_*(t) = n_0 + \eta \log_{\varphi}\!\Big(\frac{a(t)}{a(t_{\rm ref})}\Big).
\]

Сегодняшний возраст $t_0$ фиксирует $n_*(t_0)$, а значит — глубину аппроксимации золотого отношения, на которой стабилизировались спектры масс и смешиваний.

\section{Практические проверки}
Для верификации предлагаем:
\begin{enumerate}
    \item Сравнить отношения $\lambda_{k+1}/\lambda_k$ с $\varphi$ на разных секторах (u, d, $\ell$, $\nu$).
    \item Оценить фрактальные размерности интегральной плотности состояний (IDOS) и их зависимость от $V/J$.
    \item Проверить, что $n_*(t_0)$ соответствует минимуму ошибки фитинга экспериментальных данных.
    \item Использовать q-Фибоначчи для моделирования малых отклонений, приводящих к CKM/PMNS матрицам.
\end{enumerate}

\section{Заключение}
Включение фибоначчиевских структур в ZFSC позволяет:
\begin{itemize}
    \item естественным образом объяснить устойчивость плато спектра,
    \item связать экспериментальные массы и смешивания с универсальным масштабом золотого сечения,
    \item интерпретировать возраст Вселенной как глубину фибоначчиевой аппроксимации,
    \item сформулировать обратную спектральную задачу для реконструкции предгеометрической матрицы.
\end{itemize}

Таким образом, теория приобретает дополнительную красоту, согласованность и предсказательную силу.

\bibliographystyle{plain}
\begin{thebibliography}{9}
\bibitem{Suto} A. Sütő, \emph{The spectrum of a quasiperiodic Schrödinger operator}, Comm. Math. Phys. (1989).
\bibitem{Damanik} D. Damanik, \emph{Fibonacci Hamiltonian}, in: Mathematics of Aperiodic Order (2015).
\bibitem{AubryAndre} S. Aubry, G. André, \emph{Analyticity breaking and Anderson localization in incommensurate lattices}, Ann. Israel Phys. Soc. (1980).
\end{thebibliography}

\end{document}
