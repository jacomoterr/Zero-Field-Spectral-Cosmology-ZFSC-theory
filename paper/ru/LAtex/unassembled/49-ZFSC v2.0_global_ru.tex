\documentclass[12pt,a4paper]{article}
\usepackage[utf8]{inputenc}
\usepackage[russian]{babel}
\usepackage{amsmath,amssymb,amsfonts}
\usepackage{bm}
\usepackage{physics}
\usepackage{siunitx}
\usepackage{booktabs}
\usepackage{hyperref}
\usepackage{geometry}
\geometry{margin=2.3cm}
\hypersetup{colorlinks=true,linkcolor=blue,citecolor=blue,urlcolor=blue}
\usepackage[utf8]{inputenc}
\usepackage[russian]{babel}
\usepackage{tikz}
\usetikzlibrary{arrows.meta}
\usetikzlibrary{positioning}

\title{Глобальная система квантовых уравнений в спектральной космологии нулевого поля (ZFSC v2.0)}
\author{Евгений Монахов \\ ООО «VOSCOM ONLINE» Research Initiative \\ ORCID: 0009-0003-1773-5476}
\date{Сентябрь 2025}

\begin{document}
\maketitle

\section*{Аннотация}
Представлена глобальная система квантовых уравнений теории \emph{Zero-Field Spectral Cosmology} (ZFSC v2.0). Включены квантованные состояния как материальных мод, так и самой геометрии, параметризованной через фибоначчиевские и квазипериодические матрицы. Введены поправки на запутанность, космологическую эволюцию и возраст Вселенной как глубину аппроксимации золотого отношения. Сформулирована обратная спектральная задача для реконструкции предгеометрии на основе экспериментальных данных (массы фермионов, CKM/PMNS). Показано, что гипотеза о «фибоначчиевой геометрии» естественным образом добавляет красоту и простоту теории, но не является обязательной — теория вмещает разные варианты квантования геометрии.

\section{Постулаты ZFSC}
\begin{enumerate}
    \item \textbf{Нулевой уровень энтропии:}
    \[
    S \to 0.
    \]
    Вселенная описывается суперпозицией:
    \[
    \Psi = \sum_{i,j} a_{ij}\, |i_{\text{matter}}\rangle \otimes |j_{\text{geom}}\rangle .
    \]

    \item \textbf{Матричная структура:}  
    Реальность реализуется как вложенные матрицы связей («луковичные слои»).  
    Размерность и структура задаются параметрами \((\Delta, r, g_L, g_R, h_1,h_2,h_3)\).

    \item \textbf{Закон спектральной устойчивости:}
    \[
    \frac{d \lambda_n}{dt} = 0 \quad \Rightarrow \quad E_n = \lambda_n c^2 = \text{const}.
    \]
\end{enumerate}

\section{Гамильтониан глобальной системы}
Общее уравнение:
\[
H_{\mathrm{tot}} = H_{\mathrm{matter}} \otimes I + I \otimes H_{\mathrm{geom}} + H_{\mathrm{int}}.
\]

\subsection{Материальный гамильтониан}
Матрица $H_{\mathrm{matter}}$ параметризуется как:
\[
H_{\mathrm{matter}} = H(\Delta, r, g_L, g_R, h_1,h_2,h_3).
\]

Собственные значения:
\[
m^{(s)}_n = \lambda_n(H_{\mathrm{matter}}), \quad s \in \{u, d, \ell, \nu\}.
\]

\subsection{Геометрический гамильтониан}
Геометрия трактуется как квантованная:
\[
H_{\mathrm{geom}} \in \{ H_{\mathrm{Fib}}, H_{\mathrm{GoldenLap}}, H_{q\text{-Fib}}, H_{\text{Rand}}, \dots \}.
\]

Примеры:
\begin{itemize}
    \item \emph{Фибоначчиев гамильтониан:}
    \[
    (H\psi)_n = \psi_{n+1} + \psi_{n-1} + V\chi_{\text{Fib}}(n)\psi_n.
    \]
    \item \emph{Golden Laplacian:} вложенные графы с ростом узлов $\sim F_n$.
    \item \emph{q-Фибоначчи:} мягкая деформация для CKM/PMNS.
\end{itemize}

\subsection{Взаимодействие и запутанность}
Поправки на межслойную и внутрислойную корреляцию:
\[
\Delta E_s = \alpha I_{AB} + \beta I_{\mathrm{intra}},
\]
где $I$ — мера взаимной информации.

\section{Основное уравнение устойчивости}
Суперпозиция подчиняется:
\[
H_{\mathrm{tot}} \Psi = \Lambda \Psi ,
\]
где $\Lambda$ — совокупный спектр (массы, бозонные моды, тахионные состояния).

\section{Привязка к наблюдаемым данным}
\begin{enumerate}
    \item \textbf{Массы фермионов:}
    \[
    m_f^{(n)} = \lambda_n(H_{\mathrm{tot}}).
    \]
    \item \textbf{Матрицы смешивания:}
    \[
    U_{\mathrm{CKM}} = U_u^\dagger U_d, \quad U_{\mathrm{PMNS}} = U_\ell^\dagger U_\nu.
    \]
    \item \textbf{Бозонные моды:}  
    \(\lambda \approx 0\) — гравитон, \(\lambda < 0\) — тахион.
    \item \textbf{Возраст Вселенной и глубина аппроксимации:}
    \[
    n_*(t) = n_0 + \eta \log_{\varphi}\!\Big(\frac{a(t)}{a(t_{\rm ref})}\Big).
    \]
\end{enumerate}

\section{Обратная спектральная задача}
Формулировка:
\[
\min_{H_{\mathrm{geom}}} \; \mathcal{L} = \sum_k \left( \lambda_k^{\mathrm{exp}} - \lambda_k(H_{\mathrm{tot}}) \right)^2
+ \gamma \left\| U_{\mathrm{exp}} - U(H_{\mathrm{tot}}) \right\|_F^2 .
\]

Здесь $\lambda_k^{\mathrm{exp}}$ — экспериментальные массы, $U_{\mathrm{exp}}$ — CKM/PMNS.

Это \emph{inverse eigenvalue problem} для предгеометрической матрицы.

\section{Абсолютизация теории}
Общий вид:
\[
H_{\mathrm{geom}} = \sum_{p} c_p H_p ,
\]
где $H_p$ — кандидаты (Фибоначчи, случайные, фрактальные), $c_p$ — веса.

Если гипотеза о Фибоначчи неверна, веса $c_{\mathrm{Fib}} \to 0$, и теория сама обнуляет лишнее.

\section{Заключение}
В версии ZFSC v2.0:
\begin{itemize}
    \item учтены квантовые состояния геометрии;
    \item включены параметры $(\Delta, r, g_L, g_R, h_1,h_2,h_3, \alpha, \beta, n_*(t))$;
    \item сформулирована обратная задача реконструкции предгеометрии;
    \item гипотеза о фибоначчиевской геометрии становится проверяемой и естественно вплетается в модель.
\end{itemize}

Таким образом, теория достигает уровня абсолютизма: она охватывает все возможные геометрические структуры и позволяет как подтвердить, так и опровергнуть гипотезу о золотом квантовании Вселенной.

\section{Сравнение: ZFSC v1.0 и ZFSC v2.0}

Для ясности зафиксируем основные отличия между первой формулировкой ZFSC и расширенной версией v2.0.

\begin{table}[h!]
\centering
\begin{tabular}{@{}p{4cm}p{5.5cm}p{5.5cm}@{}}
\toprule
 & \textbf{ZFSC v1.0} & \textbf{ZFSC v2.0} \\
\midrule
\textbf{Пространство состояний} & Только материальные моды $|i_{\mathrm{matter}}\rangle$ & Тензорное произведение материальных и геометрических мод $|i_{\mathrm{matter}}\rangle \otimes |j_{\mathrm{geom}}\rangle$ \\
\textbf{Геометрия} & Рассматривалась как фиксированный фон & Квантована через фибоначчиевские, golden Laplacian, q-Фибоначчи, фрактальные или случайные матрицы \\
\textbf{Параметры} & $(\Delta, r, g_L, g_R, h_1,h_2,h_3)$ & Расширено: $(\Delta, r, g_L, g_R, h_1,h_2,h_3, \alpha, \beta, n_*(t))$ \\
\textbf{Запутанность} & Не учитывалась & Введены явные поправки $\Delta E_s = \alpha I_{AB} + \beta I_{\mathrm{intra}}$ \\
\textbf{Возраст Вселенной} & Не связан со спектром & Закодирован как глубина аппроксимации Фибоначчи $n_*(t)$ \\
\textbf{Обратная задача} & Только прямой расчёт спектров & Обратная задача: реконструкция $H_{\mathrm{geom}}$ по экспериментальным массам и матрицам смешивания \\
\textbf{Закон спектра} & $d\lambda_n/dt = 0$ (устойчивость) & То же, но для совместного спектра материя+геометрия \\
\bottomrule
\end{tabular}
\caption{Эволюция концепции ZFSC: от версии 1.0 к версии 2.0.}
\end{table}

\section{Преимущества ZFSC v2.0}
\begin{itemize}
    \item \textbf{Унификация:} материя и геометрия трактуются на равных основаниях как квантованные состояния.  
    \item \textbf{Гибкость:} внутри одной модели можно тестировать разные геометрии (Фибоначчи, случайные, фрактальные).  
    \item \textbf{Привязка к наблюдениям:} реконструкция напрямую из экспериментальных данных (массы фермионов, CKM/PMNS).  
    \item \textbf{Космологическая связь:} возраст Вселенной естественно кодируется глубиной аппроксимации $n_*(t)$.  
    \item \textbf{Фальсифицируемость:} гипотеза о Фибоначчи проверяема: если природа не использует золотое квантование, вес $c_{\mathrm{Fib}}$ обнуляется.  
\end{itemize}

\section{Нерешённые проблемы}
\begin{enumerate}
    \item \textbf{Численные алгоритмы:} эффективные методы решения обратной задачи для больших матриц ещё не отработаны.  
    \item \textbf{Физическая интерпретация:} требуется уточнение роли отрицательных собственных значений (тахионов) в космологии.  
    \item \textbf{Параметры запутанности:} константы $\alpha, \beta$ в $\Delta E_s$ нуждаются в физическом выводе или экспериментальной оценке.  
    \item \textbf{Космическая эволюция:} необходимо откалибровать функцию $n_*(t)$ на данных по CMB и формированию структур.  
    \item \textbf{Универсальность:} остаётся вопрос, уникальна ли структура Фибоначчи или это лишь один из возможных квазипериодических/фрактальных кандидатов.  
    \item \textbf{Экспериментальные сигналы:} важно найти дополнительные наблюдаемые (например, в спектрах бозонов), выходящие за рамки масс и смешиваний фермионов.  
\end{enumerate}

\end{document}
