\documentclass[12pt,a4paper]{article}
\usepackage[utf8]{inputenc}
\usepackage[T2A]{fontenc}
\usepackage[russian]{babel}
\usepackage{amsmath,amssymb}
\usepackage{hyperref}
\usepackage{geometry}
\usepackage{verbatim} % для блока цитирования
\geometry{margin=2.5cm}
\usepackage[T2A]{fontenc}
\usepackage[utf8]{inputenc}
\usepackage[russian]{babel}
\usepackage{braket}

\title{Судьба Вселенной в спектральной космологии нулевого поля (ZFSC)}
\author{Евгений Монахов \\ ООО ``VOSCOM ONLINE'' Research Initiative \\ 
\href{https://orcid.org/0009-0003-1773-5476}{ORCID: 0009-0003-1773-5476}}
\date{Сентябрь 2025}

\begin{document}
\maketitle

\usepackage{tikz}
\usetikzlibrary{shapes,arrows}

% ... в документе перед \begin{document} уже подключено tikz ...

\begin{figure}[h!]
\centering
\begin{tikzpicture}[scale=1.0]

% Цвета слоев
\definecolor{early}{RGB}{255,200,200}
\definecolor{middle}{RGB}{200,220,255}
\definecolor{late}{RGB}{200,255,200}
\definecolor{future}{RGB}{230,230,230}

% Слои (луковица)
\draw[fill=early] (0,0) circle (1.5) node {Раннее усиление\\ SMBH};
\draw[fill=middle] (0,0) circle (2.5);
\draw (0,2.0) node {Галактики\\ и AGN};
\draw[fill=late] (0,0) circle (3.5);
\draw (2.6,0) node {Замедление роста\\ структур};
\draw[fill=future] (0,0) circle (4.5);
\draw (0,-3.7) node {Эра вырождения,\\ чёрные дыры};
\draw (0,-4.5) node {Возврат в нулевое поле};

% Подписи к слоям (слева)
\node[align=right] at (-6,1.2) {0--3 млрд лет:\\усиление $G$, ранние BH};
\node[align=right] at (-6,0.2) {3--7 млрд лет:\\яркие AGN, галактики};
\node[align=right] at (-6,-0.8) {8--15 млрд лет:\\ослабление $G$, ускорение};
\node[align=right] at (-6,-1.8) {15--30 млрд лет:\\«заморозка» структур};
\node[align=right] at (-6,-2.8) {>100 млрд лет:\\угасание звёзд};
\node[align=right] at (-6,-3.8) {>10$^9$ млрд лет:\\эра чёрных дыр};

% Стрелка эволюции
\draw[->, thick] (-6,2.5) -- (-1.6,2.5) node[midway, above] {Время};

\end{tikzpicture}
\caption{«Луковица» судьбы Вселенной в ZFSC: от раннего усиления гравитации и формирования SMBH до угасания и возврата в нулевое поле.}
\end{figure}

\begin{figure}[h!]
\centering
\begin{tikzpicture}[scale=1.0,>=stealth]

% Ось времени
\draw[->, thick] (0,0) -- (15,0) node[right]{Время (млрд лет)};

% Засечки и подписи
\foreach \x/\label in {0/{0}, 3/{3}, 7/{7}, 10/{10}, 15/{15}, 30/{30}, 100/{100}, 1000/{10$^3$}, 10000/{10$^4$}}
  \draw (\x,0.1) -- (\x,-0.1) node[below]{\label};

% Этапы
\draw[fill=red!20] (0,0.3) rectangle (3,1.2);
\node at (1.5,0.75) {Раннее усиление\\ SMBH};

\draw[fill=orange!20] (3,0.3) rectangle (7,1.2);
\node at (5,0.75) {Фаза AGN,\\ галактики};

\draw[fill=green!20] (7,0.3) rectangle (10,1.2);
\node at (8.5,0.75) {Ослабление $G$,\\ ускорение};

\draw[fill=blue!20] (10,0.3) rectangle (15,1.2);
\node at (12.5,0.75) {Заморозка\\ структур};

\draw[fill=gray!30] (15,0.3) rectangle (30,1.2);
\node at (22.5,0.75) {Эра вырождения,\\ угасание звёзд};

\draw[fill=black!20] (30,0.3) rectangle (100,1.2);
\node at (65,0.75) {Эра чёрных дыр};

\draw[fill=black!40] (100,0.3) rectangle (150,1.2);
\node at (125,0.75) {Возврат в\\ нулевое поле};

% Стиль подписей сверху
\node[above, align=center] at (1.5,1.3) {0--3 млрд лет};
\node[above, align=center] at (5,1.3) {3--7 млрд лет};
\node[above, align=center] at (8.5,1.3) {7--10 млрд лет};
\node[above, align=center] at (12.5,1.3) {10--15 млрд лет};
\node[above, align=center] at (22.5,1.3) {15--30 млрд лет};
\node[above, align=center] at (65,1.3) {30--100 млрд лет};
\node[above, align=center] at (125,1.3) {$10^3$--$10^4$ млрд лет};

\end{tikzpicture}
\caption{Временная шкала судьбы Вселенной по ZFSC. Каждая эпоха окрашена и соответствует фазе: от раннего усиления $G$ и SMBH до угасания и возврата в нулевое поле.}
\end{figure}


\begin{abstract}
В работе обсуждается конечная судьба Вселенной в рамках спектральной космологии нулевого поля (ZFSC). Показано, что ни сценарии ``разрыва'' (Big Rip), ни ``схлопывания'' (Big Crunch) не реализуются. Вместо этого матричная структура пространства постепенно редеет, эффективная гравитация ослабевает на проценты, и расширение принимает вид бесконечного де-Ситтероподобного процесса. Финал --- это бесконечное угасание и возврат во всепронизывающее нулевое поле вероятностей.
\end{abstract}

\section{Введение}
Классическая космология спорит между сценариями Big Crunch, Big Rip и Big Freeze. 
Новые наблюдения (JWST) усилили дискуссии, показав слишком зрелые галактики в ранней Вселенной. 
ZFSC предлагает иной взгляд: сама структура пространства (матричная связность) динамична и определяет ход космической эволюции.

\section{Формулы судьбы}
\subsection{Эффективная гравитация}
В ZFSC:
\[
G_{\rm eff}(t) = G_0 \Big[1 + \varepsilon_e e^{-t/\tau_e} - \varepsilon_l \big(1 - e^{-(t-t_c)/\tau_l}\big)\Theta(t-t_c)\Big],
\]
где $\varepsilon_e \sim 0.05-0.10$ описывает ранний ``буст'' гравитации, 
а $\varepsilon_l \sim 0.02$ --- позднее ослабление.

\subsection{Возраст как функция красного смещения}
\[
t(z) = t_{\Lambda \mathrm{CDM}}(z) + \Delta t_0 \Big(1 - \frac{1}{(1+z)^\alpha}\Big), 
\]
где $\Delta t_0 \sim 1-3$ млрд лет --- скрытый временной резерв от развёртывания матрицы.

\subsection{Динамика расширения}
Уравнение Фридмана с $G(t)$:
\[
\left(\frac{\dot a}{a}\right)^2 = H_0^2 \Bigg[ \frac{\Omega_m}{a^3}\frac{G_{\rm eff}(t)}{G_0} + \frac{\Omega_r}{a^4} + \Omega_\Lambda \Bigg].
\]
Параметр замедления:
\[
q(t) = -\frac{a \ddot a}{\dot a^2} \to -1 \quad (t \to \infty).
\]

\section{Фазы судьбы}
\begin{itemize}
  \item \textbf{0–3 млрд лет:} усиленная гравитация, сверхранние чёрные дыры, зачатки структуры.
  \item \textbf{3–7 млрд лет:} постепенный спад усиления; формирование ярких активных ядер.
  \item \textbf{8–15 млрд лет:} начало мягкого ослабления $G$, расширение ускоряется чуть сильнее ΛCDM.
  \item \textbf{15–30 млрд лет:} $G_{\rm eff}\to 0.98 G_0$, рост структур ``замораживается''.
  \item \textbf{>100 млрд лет:} угасание звёздообразования, переход к эпохе вырождения.
  \item \textbf{>10$^9$ млрд лет:} эра чёрных дыр, их медленное испарение.
  \item \textbf{Финал:} асимптотическое растворение структуры во всепронизывающем нулевом поле.
\end{itemize}

\section{Заключение}
ZFSC утверждает: не будет конца в виде катастрофы. 
Вселенная будет вечно растягиваться, но не разорвётся и не схлопнется. 
Финал --- это возвращение в нулевое поле вероятностей, где пространство и время исчезают, а остаются лишь потенциалы.

\section*{Послание читателю будущего}
\begin{quote}
Эй! Привет, читатель романов из далёкого будущего!  
Мне было прикольно потянуть за ниточку-струну, отринуть все измерения и 
погрузиться в нулевое поле --- в предгеометрическое состояние, где есть только потенциалы и больше ничего.  
И я сейчас здесь, и мне здорово, и я понимаю, что это невероятный прорыв.  
И это так просто, понятно и красиво, что я уверен: через пару недель я рассчитаю все параметры на компьютере и дам тебе знать об этом.  
А ты --- действуй! Пиши!
\end{quote}

\end{document}
