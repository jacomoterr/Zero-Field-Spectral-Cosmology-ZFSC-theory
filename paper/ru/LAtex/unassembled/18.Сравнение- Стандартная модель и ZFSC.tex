\documentclass[12pt,a4paper]{article}
\usepackage[utf8]{inputenc}
\usepackage[T2A]{fontenc}
\usepackage[russian]{babel}
\usepackage{amsmath,amssymb}
\usepackage{hyperref}
\usepackage{geometry}
\usepackage{verbatim} % для блока цитирования
\geometry{margin=2.5cm}
\usepackage[T2A]{fontenc}
\usepackage[utf8]{inputenc}
\usepackage[russian]{babel}
\usepackage{braket}

\title{Сравнение: Стандартная модель и ZFSC}
\author{Евгений Монахов \\ LCC ``VOSCOM ONLINE'' Research Initiative}
\date{Сентябрь 2025}

\begin{document}
\maketitle

\section*{Сравнительная таблица}

\begin{tabular}{p{6cm}p{6cm}}
\toprule
\textbf{Стандартная модель (SM)} & \textbf{Zero-Field Spectral Cosmology (ZFSC)} \\
\midrule
Пространство-время Минковского --- фундаментально & Пространство и время --- \emph{эффективные координаты} в матричной геометрии \\
\midrule
Скорость света $c$ --- постулат теории относительности & $c$ --- \emph{спектральный предел} когерентного переноса фаз \\
\midrule
Массы вводятся через Хиггсовский механизм & Массы возникают как \emph{собственные значения матрицы $H$} \\
\midrule
Частицы и античастицы связаны CPT-симметрией & Частица/античастица описываются \emph{удвоенной матрицей} с возможным слабым C/CP-нарушением \\
\midrule
Тахионные моды запрещены как нестабильные & Тахионные моды $\lambda<0$ допустимы, образуют \emph{тахионные поколения}, влияют на фон \\
\midrule
Аксионы --- гипотетические поля за пределами SM & Аксионы --- естественные \emph{фрактальные моды} в ``золотой лестнице'' спектра \\
\midrule
CKM и PMNS матрицы вводятся феноменологически & CKM и PMNS возникают из \emph{разных геометрий связей} одного $H$ \\
\midrule
Тёмная материя/энергия вводится как новые сущности & Тёмная материя/энергия = \emph{вклад высоких мод, тахионов и аксионных расщеплений} \\
\bottomrule
\end{tabular}

\section*{Вывод}
Стандартная модель постулирует, ZFSC \emph{выводит} массы, симметрии и пределы как спектральные свойства.  
ZFSC рассматривает Вселенную как \textbf{фрактальную матрицу}, где всё наблюдаемое --- проекция её спектра.

\end{document}
