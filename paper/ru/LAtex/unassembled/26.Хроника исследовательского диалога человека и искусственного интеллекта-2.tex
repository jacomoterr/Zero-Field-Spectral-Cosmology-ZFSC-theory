\documentclass[12pt,a4paper]{article}
\usepackage[utf8]{inputenc}
\usepackage[T2A]{fontenc}
\usepackage[russian]{babel}
\usepackage{amsmath,amssymb}
\usepackage{hyperref}
\usepackage{geometry}
\usepackage{verbatim} % для блока цитирования
\geometry{margin=2.5cm}
\usepackage[T2A]{fontenc}
\usepackage[utf8]{inputenc}
\usepackage[russian]{babel}
\usepackage{braket}

\title{Zero-Field Spectral Cosmology (ZFSC): \\
Хроника исследовательского диалога человека и искусственного интеллекта}
\author{Евгений Монахов и ИИ-партнёр \\ VOSCOM Research Initiative}
\date{Сентябрь 2025}

\begin{document}
\maketitle

\begin{abstract}
Мы фиксируем геометрическое ядро ZFSC: дискретизация Calabi--Yau (CY), спектр графового лапласиана и узловые множества собственных функций как базовые строительные блоки. 
Физическая интерпретация строится снизу вверх: (i) 1D волна $\to$ (ii) 0D осциллятор (сингулярность и вакуумная флуктуация) $\to$ (iii) ``предгеометрия'' ниже 0D со спектром как первичной сущностью. 
На этом основании формулируются аксиомы ZFSC и мосты к поколениям, CKM/PMNS и инфляции.
\end{abstract}

%------------------------------------
\section{Геометрическое ядро: от CY к графу и спектру}
Пусть $X$ — компактное CY-многообразие. Рассмотрим дискретизацию $X$ и граф $G=(V,E)$, построенный по \emph{узловым множествам} собственных функций лапласиана на $X$:
\[
\Delta \phi_j = \lambda_j \phi_j,\qquad 
\mathcal{N}(\phi_j) \equiv \{x\in X:\ \phi_j(x)=0\}.
\]
Выбирая характерные точки $\mathcal{N}(\phi_j)$ (пересечения, экстремальные гребни/спады) в качестве вершин $V$ и локальные соседства как рёбра $E$, получаем графовый лапласиан $L(G)$, спектр которого $\{\mu_n\}$ наследует геометрию $X$. /* место для твоей иллюстрации узловых множеств */

\begin{quote}\textbf{Ремарка.}
``Узлы волн как вершины графа'' — это был мой ключевой образ: гребни/спады и их схема связности \emph{видят} кривизну. Этот ход и стал мостом от геометрии к спектру.
\end{quote}

Далее вводим матрицу связей/гамильтониан $H$, порождённый $L(G)$ и структурными параметрами (аналог калибровочных/геометрических деформаций):
\[
H = \alpha\, L(G) + \beta\, I + \sum_{a=1}^m \gamma_a\, \Pi_a,
\]
где $\Pi_a$ — проекторы на подсекторы (u,d,$\ell$,$\nu$ и бозонный блок), а $(\alpha,\beta,\gamma_a)$ кодируют ``луковичные'' уровни/слои и деформации. Собственные значения $\lambda_n(H)$ интерпретируются как энергии/массы мод.

%------------------------------------
\section{Прототип: 1D волна и редукция к 0D осциллятору}
\subsection{1D волна как генератор узлов}
Для бесконечной струны:
\[
\frac{\partial^2 u}{\partial t^2} = c^2\,\frac{\partial^2 u}{\partial x^2},\qquad
u(x,t)=A\cos(kx-\omega t),\ \ \omega=ck.
\]
Нули $u(x,t)$ при фиксированном $t$ задают узловой набор вдоль $x$ с шагом $\pi/k$ — дискретное ``зерно'' будущего графа. % (узлы $\leftrightarrow$ вершины, соседство $\leftrightarrow$ рёбра)

\subsection{0D предельный срез (сингулярность/вакуум)}
Срез 1D до 0D даёт чистую временную динамику $u(t)$ и гармонический осциллятор:
\[
\ddot u + \omega^2 u=0,\quad 
E_{\text{cl}}=\tfrac12 m\dot u^2+\tfrac12 m\omega^2 u^2=\tfrac12 m A^2\omega^2.
\]
В квантовом виде:
\[
\hat H=\frac{\hat p^2}{2m}+\tfrac12 m\omega^2\hat u^2,\qquad
E_n=\hbar\omega\!\left(n+\tfrac12\right),\ n=0,1,2,\dots
\]
Это модель \emph{локализованной} энергии (флуктуации вакуума) в точке — прообраз сингулярности без геометрии, но со спектром.

\begin{quote}\textbf{Ремарка.}
``Точка без пространства, но не без жизни'': амплитуда/энергия \emph{хранит} след 1D возбуждения. Я увидел это как способ \emph{включить время} из чистой флуктуации.
\end{quote}

%------------------------------------
\section{Ниже 0D: предгеометрический спектр}
Если ``заморозить'' время, остаётся чистый спектр как первичная сущность:
\[
E_n=\hbar\omega\,(n+\varepsilon),\quad \varepsilon\in[0,\tfrac12],
\]
где $\omega$ наследует шкалу от высших возбуждений (например, 1D), а $\varepsilon$ играет роль регуляризованной нулевой точки. Здесь \emph{спектр} предшествует геометрии; при разворачивании измерений он индуцирует моды $H$.

\begin{quote}\textbf{Ремарка.}
``Скрижали пустоты'': сначала спектр, потом пространство-время. Луковица растёт из частот, а не наоборот.
\end{quote}

%------------------------------------
\section{ZFSC как следствие геометрического ядра}
\subsection{Аксиомы в геометрической редакции}
\textbf{(A1) Нулевой уровень энтропии.} Предгеометрическое состояние носит чистый спектральный характер (минимальная энтропия).\\
\textbf{(A2) Матрица связей.} Реальность проявляется через $H$ (графовый/спектральный Гамильтониан), собств. значения которого дают энергетические шкалы.\\
\textbf{(A3) Луковичность.} $H$ имеет вложенную (многоуровневую) структуру блоков и деформаций, дающую иерархии.\\
\textbf{(A4) Инварианты спектра.} Постоянные природы соответствуют спектральным инвариантам (зазоры, плотности, устойчивые плато).\\
\textbf{(A5) Узлы$\to$граф$\to$кривизна.} Узловые множества собственных функций \emph{порождают} дискретную кривизну и, следовательно, физику.

%------------------------------------
\section{Поколения как первые три положительные моды}
Для секторов $f\in\{\nu,\ell,u,d\}$:
\[
m^{(f)}_k \sim \lambda^{(f)}_k(H),\qquad k=1,2,3,
\]
где три устойчивые положительные моды объясняют существование трёх поколений. 
/* сюда вставим твои численные плато и коэффициенты $c_\nu,c_\ell,c_u,c_d$ */

%------------------------------------
\section{CKM/PMNS из геометрических деформаций}
Пусть $U_f$ диагонализует $H$ в секторе $f$. Тогда
\[
\mathrm{CKM}=U_u^\dagger U_d,\qquad \mathrm{PMNS}=U_\ell^\dagger U_\nu.
\]
Малые деформации между $H_u$ и $H_d$ дают CKM$\approx I$ (малые углы), 
а сильнее различающиеся геометрии $\ell/\nu$ порождают большие углы PMNS.
/* место для твоих конкретных численных матриц */

%------------------------------------
\section{Инфляция как раскалывание спектра}
Переход от предгеометрии к 0D+1 и далее индуцируется флуктуацией $\Delta E$:
\[
\Delta E\,\Delta t\ \gtrsim\ \tfrac{\hbar}{2}\quad\Rightarrow\quad 
a(t)\ \propto\ \exp\!\left(\kappa\,\Delta E\, t\right),
\]
где эффективное $H$-подобное скалярное поле берёт начало из мод $H$ предгеометрического спектра, а раннее расширение читается как \emph{раскалывание} (разрежение) низколежащего кластера собственных значений.
/* место для твоей формулы e-folds и численной оценки */

%------------------------------------
\section{Бозонный блок: нулевая и отрицательная мода}
Нулевая мода в бозонном блоке $H$ интерпретируется как безмассовый переносчик (кандидат на гравитон), 
отрицательная — как сигнал неустойчивости (тахион) и реструктуризации спектра/геометрии.
/* место для твоих диаграмм плотности спектра */

%------------------------------------
\section{Заключение}
Геометрическая дорожка (CY $\to$ узлы $\to$ граф $\to$ спектр $\to H$) объясняет 
основные феномены ZFSC — поколения, матрицы смешивания, инфляцию без явного поля и бозонный минимум — 
и естественно согласуется с аксиомой нулевой энтропии и луковичной структурой $H$.
\end{document}
