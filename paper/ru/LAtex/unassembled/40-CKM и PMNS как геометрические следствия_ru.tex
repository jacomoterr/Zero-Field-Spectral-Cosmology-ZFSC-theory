\documentclass[12pt,a4paper]{article}
\usepackage[utf8]{inputenc}
\usepackage[russian]{babel}
\usepackage{amsmath,amssymb,amsfonts}
\usepackage{bm}
\usepackage{physics}
\usepackage{siunitx}
\usepackage{booktabs}
\usepackage{hyperref}
\usepackage{geometry}
\geometry{margin=2.3cm}
\hypersetup{colorlinks=true,linkcolor=blue,citecolor=blue,urlcolor=blue}
\usepackage[utf8]{inputenc}
\usepackage[russian]{babel}
\usepackage{tikz}
\usetikzlibrary{arrows.meta}
\usetikzlibrary{positioning}

\title{CKM и PMNS как геометрические следствия\\
Zero-Field Spectral Cosmology (ZFSC)}
\author{Евгений Монахов \\ VOSCOM ONLINE Research Initiative}
\date{8 сентября 2025}

\begin{document}
\maketitle

\begin{abstract}
Мы показываем, что в рамках ZFSC смешивание поколений кварков и лептонов (матрицы CKM и PMNS) возникает как чисто геометрический эффект из одной базовой эрмитовой матрицы $H$, представляющей дискретный слой связности ``луковичной'' матрицы реальности. Разные физические сектора ($u,d,\ell,\nu$) получаются действием на $H$ различающихся геометрических преобразований (граничные условия, твисты, перестановки, локальные деформации), после чего диагонализация секторных матриц порождает собственные векторы $U_s$ и массы как устойчивые ``плато'' собственных значений. Мы выводим: 
\[
V_{\rm CKM}=U_u^\dagger U_d,\qquad U_{\rm PMNS}=U_\ell^\dagger U_\nu,
\]
обосновываем малость смешивания кварков (близость $T_u$ и $T_d$) и большие углы у нейтрино (сильное отличие $T_\ell$ и $T_\nu$), а также даём пертурбативные формулы для углов, фаз и инварианта Ярлскога через элементы возмущения $\delta H$. Показаны тестируемые предсказания и численная процедура подбора параметров $(\Delta,r; g_L,g_R; h_1,h_2,h_3)$.
\end{abstract}

\section*{Введение}
ZFSC постулирует фундаментальный уровень $S\to 0$, где Вселенная задаётся чистым вероятностным полем амплитуд и дискретной матричной структурой связности. На фиксированном слое задаётся базовая эрмитова матрица $H=H^\dagger$, чьи спектральные плато соответствуют наблюдаемым иерархиям масс. Физические сектора ($u,d,\ell,\nu$) индуцируются геометрическими преобразованиями $T_s$ над $H$, после чего смешивание есть относительная ориентация собственных баз этих секторных матриц.

\section{Базовая матрица и секторные преобразования}
Пусть $H(\Delta,r;g_L,g_R;\bm{h})\in\mathbb{C}^{N\times N}$ — эрмитова, где параметры $(\Delta,r)$ кодируют мезомасштабную дискретную геометрию, $g_L,g_R$ — асимметрии левых/правых связностей, а $\bm{h}=(h_1,h_2,h_3)$ — локальные деформации/твисты.

Определим секторные матрицы
\begin{equation}
H_s \equiv T_s[H], \qquad s\in\{u,d,\ell,\nu\},
\end{equation}
где $T_s$ — геометрические операции над графом/решёткой: 
\begin{itemize}
\item граничные условия (Dirichlet/Neumann/mixed),
\item перестановки узлов и перетасовки подрешёток,
\item локальные твисты (анизотропные повороты фаз вдоль циклов),
\item слабые деформации весов рёбер.
\end{itemize}
\noindent
Важно: $T_s$ не обязаны быть унитарными сопряжениями одного и того же оператора; обычно
\(
H_s = W_s^\dagger\, H\, W_s + \varepsilon_s\, D_s,
\)
где $W_s$ — унитарная геометрия (перестановка/фазовый твист), а $D_s=D_s^\dagger$ — малые диагональные/локальные деформации. 

\section{Массы как плато собственных значений}
Диагонализация
\begin{equation}
H_s = U_s\,\Lambda_s\, U_s^\dagger,\qquad 
\Lambda_s=\mathrm{diag}(\lambda_{s,1}\le\lambda_{s,2}\le\ldots),
\end{equation}
даёт собственные значения $\lambda_{s,i}$, среди которых устойчивые \emph{плато} интерпретируются как три поколения. Массы сектора $s$:
\begin{equation}
m_{s,i} = \alpha_s\, \lambda_{s,i}^{(+)}\quad (i=1,2,3),
\end{equation}
где $\lambda^{(+)}$ — первые три положительных устойчивых значения, $\alpha_s$ — секторный масштаб (перевод единиц/нормировка). Нулевая мода $\approx 0$ трактуется как безмассовый бозон (кандидат на гравитон), отрицательная — как тахионный индикатор неустойчивости слоя.

\section{Смешивание как относительная ориентация собственных баз}
Определения:
\begin{equation}
V_{\rm CKM} = U_u^\dagger U_d,\qquad U_{\rm PMNS}=U_\ell^\dagger U_\nu.
\end{equation}
Если $T_d$ мало отличается от $T_u$, то $U_d \approx U_u$ и $V_{\rm CKM}\approx \mathbb{I}$ (малые углы). Если же $T_\nu$ сильно отличается от $T_\ell$, то $U_\nu$ существенно повёрнута относительно $U_\ell$ (большие углы PMNS).

\section{Пертурбативный вывод малых углов CKM}
Пусть
\begin{equation}
H_d = H_u + \delta H,\qquad \norm{\delta H}\ll \min_{i\neq j}|\lambda_{u,j}-\lambda_{u,i}|.
\end{equation}
Пусть $H_u = U_u\Lambda_u U_u^\dagger$, и перейдём в базис $U_u$:
\(
\widetilde{\delta H} \equiv U_u^\dagger\, \delta H\, U_u.
\)
Для собственных векторов известна формула первого порядка (невырожденные уровни):
\begin{equation}
K_{ij} \equiv \left(U_u^\dagger U_d - \mathbb{I}\right)_{ij} 
\simeq \frac{\widetilde{\delta H}_{ij}}{\lambda_{u,j}-\lambda_{u,i}},\quad i\neq j, 
\qquad K_{ii}=0,
\end{equation}
где $K=-K^\dagger$ (антиэрмитова генерация поворота). Тогда
\begin{equation}
V_{\rm CKM} = U_u^\dagger U_d \simeq e^{-K} \simeq \mathbb{I}-K+\mathcal{O}(\delta H^2).
\end{equation}
Следовательно,
\begin{equation}
\theta_{ij}^{(q)} \simeq \frac{\abs{\widetilde{\delta H}_{ij}}}{\abs{\lambda_{u,j}-\lambda_{u,i}}},\qquad
\delta_{\rm CKM}\ \text{задаётся аргументами компонент }K_{ij}.
\end{equation}
Инвариант Ярлскога в первом ненулевом порядке:
\begin{equation}
J_{\rm CKM} \simeq \Im\!\left( K_{12}K_{23}K_{13}^* \right) + \mathcal{O}(\delta H^4).
\end{equation}
Итого, \emph{малость углов} естественна при маленьких $\norm{\delta H}$ или больших спектральных зазорах $\abs{\lambda_{u,j}-\lambda_{u,i}}$.

\section{Непертурбативно большие углы PMNS}
Для лептонов предполагаем, что $T_\nu$ существенно отличается от $T_\ell$ (другие граничные условия/твисты, возможно, топологически иные циклы), так что пертурбативная схема неприменима. Тогда
\begin{equation}
U_{\rm PMNS} = U_\ell^\dagger U_\nu,\qquad 
\text{с большими углами } \theta_{ij}^{(\ell)} \sim \mathcal{O}(1).
\end{equation}
Численно это реализуется, когда:
\begin{itemize}
\item спектры $\Lambda_\ell$ и $\Lambda_\nu$ образуют близкие плато (масс-иерархии согласованы),
\item но собственные векторы $U_\ell$ и $U_\nu$ \emph{разнонаправлены} из-за твистов/перестановок подрешёток.
\end{itemize}
Связанная диагностическая величина — норма коммутатора:
\begin{equation}
\Xi_{s,t} \equiv \frac{\norm{[H_s,H_t]}_{\mathrm{F}}}{\norm{H_s}_{\mathrm{F}}\norm{H_t}_{\mathrm{F}}}
\quad\Rightarrow\quad
\Xi_{u,d}\ll 1\ \ \text{(кварки)},\qquad \Xi_{\ell,\nu}\sim \mathcal{O}(1)\ \ \text{(нейтрино)}.
\end{equation}
Эмпирически большие $\Xi$ коррелируют с большими смешиваниями.

\section{Связь смешивания с иерархиями масс}
ZFSC предсказывает, что углы смешивания контролируются \emph{и} нормой возмущения геометрии, \emph{и} спектральными зазорами на плато:
\begin{equation}
\theta_{ij}\ \approx\ f\!\left(
\frac{\norm{\delta H}_{\text{эфф}}}{\abs{\lambda_j-\lambda_i}}
\right),
\end{equation}
где $f(x)\simeq x$ при $x\ll 1$ и $f$ насыщается при $x\gtrsim 1$ (что ведёт к большим углам у нейтрино при близких плато и больших твистах).

\section{CP-фаза и инварианты}
Определим джарлскогов инвариант через коммутатор масс-операторов:
\begin{equation}
\mathcal{J}(H_a,H_b) \equiv \frac{1}{2i}\frac{\det\big([H_a,H_b]\big)}{\prod_{i<j}(\lambda_{a,j}-\lambda_{a,i})\prod_{k<l}(\lambda_{b,l}-\lambda_{b,k})}.
\end{equation}
Для пар $(H_u,H_d)$ и $(H_\ell,H_\nu)$ он пропорционален стандартным $J_{\rm CKM},J_{\rm PMNS}$, и не меняется при унитарных перенумерациях баз. Таким образом, \emph{геометрия твистов} фиксирует и величину CP-нарушения.

\section{Численная процедура подбора}
\subsection*{Вход}
\begin{itemize}
\item Размер $N$ и параметры $H(\Delta,r;g_L,g_R;\bm{h})$.
\item Наборы преобразований $T_s$ (границы, перестановки, твисты, деформации).
\end{itemize}
\subsection*{Шаги}
\begin{enumerate}
\item Сгенерировать $H$ и $H_s=T_s[H]$ для $s\in\{u,d,\ell,\nu\}$.
\item Диагонализовать $H_s=U_s\Lambda_s U_s^\dagger$.
\item Выбрать три устойчивых положительных собственных значения в каждом секторе $\Rightarrow$ массы $m_{s,i}=\alpha_s\lambda_{s,i}^{(+)}$.
\item Построить $V_{\rm CKM}=U_u^\dagger U_d$ и $U_{\rm PMNS}=U_\ell^\dagger U_\nu$, извлечь углы, фазы, инварианты.
\item Минимизировать функционал
\[
\chi^2 = \sum_{s,i}\frac{\big(m_{s,i}^{\text{(model)}}-m_{s,i}^{\text{(exp)}}\big)^2}{\sigma_{s,i}^2}
+\sum_{\text{angles}}\frac{(\theta^{\text{(model)}}-\theta^{\text{(exp)}})^2}{\sigma^2}
+\sum_{\text{CP}}\frac{(\delta^{\text{(model)}}-\delta^{\text{(exp)}})^2}{\sigma^2}.
\]
\end{enumerate}

\section{Предсказания ZFSC для феноменологии}
\begin{itemize}
\item \textbf{Малость CKM} объясняется $\Xi_{u,d}\ll 1$ (почти совпадающие геометрии $T_u,T_d$).
\item \textbf{Большие углы PMNS} требуют $\Xi_{\ell,\nu}\sim 1$ (сильный твист/перестановка $T_\nu$ относительно $T_\ell$).
\item \textbf{Корреляции масс и смешивания:} при сжатии плато (уменьшении спектральных зазоров) углы растут даже при фиксированной норме твиста.
\item \textbf{CP-фазы:} знак и величина $\delta$ коррелируют с ориентацией комплексных фаз в $T_s$ и знаком детерминанта коммутатора $[H_a,H_b]$.
\item \textbf{Нулевая/отрицательная мода:} качественно ограничивают допустимые деформации: при переходе через нуль возможны фазовые ``переключения'' структуры смешивания.
\end{itemize}

\section{Минимальная ``геометрия'' для воспроизведения данных}
На практике достаточно:
\begin{enumerate}
\item Взять $T_u=\mathrm{id}$, $T_d=\mathrm{id}+\text{слабый диагональный сдвиг}$ на подрешётке (даёт CKM $\approx \mathbb{I}$).
\item В лептонном секторе задать $T_\ell=\mathrm{id}$, а $T_\nu =$ (перестановка блоков) $\circ$ (фазовый твист на нескольких независимых циклах) $\circ$ (слабая локальная деформация), что приводит к большим углам и допускает ненулевую $\delta_{\rm PMNS}$.
\end{enumerate}

\section{Как это снимает эффект ``подгонки''}
Введя \emph{единую} базовую матрицу $H$ с малыми и счётными геометрическими степенями свободы $T_s$, мы одновременно описываем (i) плато масс в каждом секторе, (ii) структуру смешивания между секторами. Это резко сокращает число независимых параметров по сравнению со свободным вводом Юкав/матриц масс в стандартной феноменологии.

\section{Дорожная карта вычислений}
\begin{enumerate}
\item \textbf{Стабилизация плато:} скан по $(\Delta,r)$ для нахождения устойчивых троек $\lambda_{s,1\ldots 3}^{(+)}$; проверка робастности к шумам/деформациям.
\item \textbf{CKM:} пертурбативный фит $\delta H$ над $H_u$ для малых углов; валидация формулы $K_{ij}=\widetilde{\delta H}_{ij}/(\lambda_j-\lambda_i)$.
\item \textbf{PMNS:} непертурбативный поиск классов $T_\nu$ (минимум независимых циклов твиста), обеспечивающих $(\theta_{12},\theta_{23},\theta_{13})$ и $\delta$.
\item \textbf{Инварианты:} вычисление $\Xi_{s,t}$ и $\mathcal{J}(H_s,H_t)$ как диагнóstик смешивания и CP-нарушения.
\item \textbf{Прогнозы:} корреляции $\{\theta_{ij}\}$ с отношениями масс в пределах плато, предсказания для суммарной массы нейтрино и знака $\delta_{\rm PMNS}$.
\end{enumerate}

\section*{Заключение}
CKM и PMNS в ZFSC следуют из одной матричной базы $H$ и различий геометрии секторных преобразований $T_s$. Малое смешивание кварков и большие углы у нейтрино возникают естественно как следствия малости/большойности норм коммутаторов и относительных твистов собственных баз. Пертурбативные формулы фиксируют связь углов с возмущением $\delta H$ и спектральными зазорами плато, а инварианты коммутаторов обеспечивают базонезависимую диагностику CP-фаз. Это делает ZFSC предсказательной и проверяемой на уровне масс и смешивания одновременно.

\paragraph{Код/данные.} Численные эксперименты можно реализовать в текущем прототипе \texttt{zfsc\_predictor.py}: добавить генерацию $T_s$ (перестановки/твисты), сбор $\Xi_{s,t}$, и модуль для извлечения углов/фаз из $U_u^\dagger U_d$ и $U_\ell^\dagger U_\nu$.

\end{document}
