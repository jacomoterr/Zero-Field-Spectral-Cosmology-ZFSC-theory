\documentclass[12pt,a4paper]{article}
\usepackage[utf8]{inputenc}
\usepackage[russian]{babel}
\usepackage{amsmath,amssymb,amsfonts}
\usepackage{bm}
\usepackage{physics}
\usepackage{siunitx}
\usepackage{booktabs}
\usepackage{hyperref}
\usepackage{geometry}
\usepackage{tikz}
\geometry{margin=2.3cm}
\hypersetup{colorlinks=true,linkcolor=blue,citecolor=blue,urlcolor=blue}
\usepackage[utf8]{inputenc}
\usepackage[russian]{babel}
\usetikzlibrary{arrows.meta}
\usetikzlibrary{positioning}
% В преамбулу:
\usetikzlibrary{arrows.meta, positioning}

\title{Спектральная космология нулевого поля (ZFSC): \\
Вселенная как квантовый компьютер и природа квантовой запутанности}

\author{Евгений Монахов \\
ООО ``VOSCOM ONLINE'' Research Initiative \\
\href{https://orcid.org/0009-0003-1773-5476}{ORCID: 0009-0003-1773-5476}}

\date{Сентябрь 2025}

\begin{document}
\maketitle

\begin{abstract}
В рамках теории Zero-Field Spectral Cosmology (ZFSC) показано, что Вселенную можно интерпретировать как многомерный квантовый компьютер, в котором спектральные свойства дискретных матриц формируют массы частиц и законы взаимодействий.
Рассматривается квантовая запутанность как проявление единой топологии спектральных слоёв, объясняющее феномен ``страшного дальнодействия''.
Обсуждается, у каких частиц запутанность более выражена, а также как высокие моды спектра могут быть кандидатами на тёмную материю.
\end{abstract}

%------------------------------------
\section{Введение}
Современная квантовая механика описывает микромир через вероятностные амплитуды, но многие аспекты --- происхождение масс, иерархия поколений, запутанность частиц --- остаются фундаментальными загадками.
ZFSC интерпретирует Вселенную как систему дискретных спектральных матриц, определённых на нулевом уровне энтропии ($S \to 0$).
В данной работе мы рассматриваем:
\begin{enumerate}
  \item может ли Вселенная быть интерпретирована как многомерный квантовый компьютер,
  \item как запутанность описывается в ZFSC,
  \item какие частицы обладают более выраженной запутанностью,
  \item как высокие моды спектра могут проявляться как частицы тёмной материи.
\end{enumerate}

%------------------------------------
\section{Вселенная как квантовый компьютер}
В ZFSC фундаментальное состояние описывается амплитудным полем:
\[
\Psi = \sum_{i} a_i |i\rangle,
\]
где $\{|i\rangle\}$ --- пространство всех возможных конфигураций, $a_i \in \mathbb{C}$ --- амплитуды.
Это эквивалентно гигантскому квантовому регистру с размерностью $N \to \infty$.

Каждый слой ``луковичной'' структуры матриц $H^{(n)}$ играет роль логического преобразования:
\[
H^{(n)} =
\begin{bmatrix}
H^{(n-1)} & V \\
V^\dagger & H^{(n-1)}
\end{bmatrix}.
\]
Собственные значения $\lambda_k^{(n)}$ соответствуют устойчивым состояниям (массам частиц), а собственные векторы --- ``программам'' взаимодействия.

Таким образом, Вселенная не ``считает'' в привычном смысле, но спектральная эволюция её матричных слоёв формирует наблюдаемую физическую реальность.

%------------------------------------
\section{Квантовая запутанность в ZFSC}
В стандартной КМ запутанность определяется как состояние, которое нельзя разложить на произведение:
\[
|\Psi_{AB}\rangle \neq |\psi_A\rangle \otimes |\phi_B\rangle.
\]

В ZFSC запутанность интерпретируется иначе:
\begin{itemize}
  \item частицы $A$ и $B$ являются не независимыми объектами, а \textbf{проекциями одного и того же собственного вектора матрицы $H^{(n)}$};
  \item ``страшное дальнодействие'' (мгновенная корреляция) возникает, потому что обе проекции принадлежат единому спектральному состоянию, существующему вне времени и пространства;
  \item измерение одной частицы фиксирует весь собственный вектор, автоматически определяя вторую частицу.
\end{itemize}

Математически это выражается так:  
если $\lambda$ --- собственное значение, а $u = (u_A, u_B)$ --- собственный вектор, то запутанные частицы $A,B$ описываются
\[
|\Psi_{AB}\rangle = \sum_k u_{A,k} |k_A\rangle \otimes u_{B,k} |k_B\rangle,
\]
и связь $u_A, u_B$ задаётся топологией матрицы, а не пространственным расстоянием.

%------------------------------------
\section{У каких частиц запутанность сильнее}
По ZFSC запутанность зависит от ``глубины слоя'' и связности матрицы:
\begin{itemize}
  \item \textbf{Нейтрино} --- максимальная запутанность. Они ``лёгкие'' и сильно зависят от структуры PMNS, поэтому их спектральные векторы сильно переплетены.
  \item \textbf{Фотон, глюон} --- высокая запутанность. Безмассовые бозоны сидят на низких модах, где корреляции сильнее.
  \item \textbf{Электрон, мюон, тау} --- умеренная запутанность. Массы фиксированы плато, но всё ещё возможны нелокальные связи.
  \item \textbf{Тяжёлые кварки (t,b)} --- слабая запутанность. Их спектральные моды устойчивы и локализованы, что снижает ``переплетение''.
\end{itemize}

\[
\text{Степень запутанности} \sim \frac{1}{m_{\text{эфф}}} \times C_{\text{связности}},
\]
где $m_{\text{эфф}}$ --- эффективная масса моды, $C_{\text{связности}}$ --- коэффициент связности в матрице.


%------------------------------------
\section{Визуализация запутанности в ZFSC}

\begin{figure}[h!]
\centering
\begin{tikzpicture}[node distance=2cm, thick]

% Матрица (слой)
\node[draw, rectangle, minimum width=4cm, minimum height=2cm, fill=blue!5] (matrix) {Спектральный слой $H^{(n)}$};

% Вектор
\draw[-{Latex[length=3mm]}, thick, red] (matrix) -- ++(0,-2) node[midway, right] {$u_k$};

% Разделение
\node[draw, circle, minimum size=1cm, below left=2cm and 1cm of matrix, fill=green!10] (A) {Частица A};
\node[draw, circle, minimum size=1cm, below right=2cm and 1cm of matrix, fill=green!10] (B) {Частица B};

\draw[-{Latex[length=3mm]}, dashed] (matrix.south west) .. controls +(0,-1) and +(0,1) .. (A.north);
\draw[-{Latex[length=3mm]}, dashed] (matrix.south east) .. controls +(0,-1) and +(0,1) .. (B.north);

% Корреляция
\draw[<->, thick, blue, dotted] (A.east) -- (B.west) node[midway, below] {Мгновенная корреляция};

\end{tikzpicture}
\caption{Запутанные частицы в ZFSC как проекции одного собственного вектора матрицы.}
\end{figure}

%------------------------------------
\section{Иерархия мод и тёмная материя}

\begin{figure}[h!]
\centering
\begin{tikzpicture}[xscale=2, yscale=1.2, thick]

% Ось
\draw[->] (0,0) -- (5,0) node[right] {$\lambda_k$ (собственные значения)};

% Низкие моды
\foreach \x/\label in {0.5/нейтрино, 1.0/лептоны, 1.5/кварки, 2.0/бозоны} {
  \draw[fill=green!60] (\x,0) circle (3pt);
  \node[below] at (\x,-0.1) {\small \label};
}

% Высокие моды
\foreach \x in {3.5, 4.0, 4.5} {
  \draw[fill=purple!70] (\x,0) circle (3pt);
}
\node[below] at (4.0,-0.1) {\small Тёмная материя};

\end{tikzpicture}
\caption{Иерархия мод в ZFSC: низкие значения $\lambda_k$ соответствуют известным частицам, высокие --- кандидатам на тёмную материю.}
\end{figure}

%------------------------------------
\section{Высокие моды и тёмная материя}
ZFSC предсказывает существование высоких собственных значений спектра:
\[
H^{(n)} u_k = \lambda_k u_k, \quad \lambda_k \gg m_{\text{известные частицы}}.
\]
Эти высокие моды не участвуют во взаимодействиях SU(3)$\times$SU(2)$\times$U(1), так как их векторы почти ортогональны нижним слоям.
Они могут быть:
\begin{itemize}
  \item стабильными (не распадаются),
  \item слабо взаимодействующими (почти не замечаются детекторами),
  \item массовыми (несут значительный вклад в плотность энергии).
\end{itemize}
Именно такие состояния являются естественными кандидатами на частицы тёмной материи.

Возможный способ обнаружения:
- искать аномальные колебания в спектрах (резонансы высоких мод);
- наблюдать гравитационные эффекты (искривления, линзирование);
- использовать космологические расчёты плотности с учётом дополнительных $\lambda_k$.

%------------------------------------
\section{Заключение}
ZFSC позволяет интерпретировать Вселенную как многомерный квантовый компьютер, где вычисления не происходят алгоритмически, а заложены в спектральной структуре матриц.
Квантовая запутанность в этой модели объясняется как единство собственных векторов, что снимает проблему ``страшного дальнодействия''.
Запутанность наиболее выражена у нейтрино и безмассовых бозонов, а слабее у тяжёлых кварков.
Высокие моды спектра естественно интерпретируются как кандидаты на тёмную материю.

%------------------------------------
\section{Прогнозы для экспериментов}

ZFSC формулирует ряд проверяемых предсказаний, которые отличают её от стандартной модели и других подходов.

\subsection{Запутанность нейтрино}
Так как нейтрино в модели соответствуют самым ``тонким'' модам спектра, их запутанность должна быть выражена сильнее других частиц.
Предсказания:
\begin{itemize}
  \item корреляции в осцилляциях нейтрино (эффект PMNS) будут демонстрировать нелокальные связи даже при больших расстояниях;
  \item степень запутанности может быть оценена через энтропию фон Неймана для матрицы плотности нейтринных состояний:
  \[
  S_\nu = - \mathrm{Tr}(\rho_\nu \ln \rho_\nu),
  \]
  которая должна быть выше, чем для лептонов или кварков.
  \item специальные эксперименты по регистрации совпадений нейтрино (например, в IceCube или JUNO) могут выявить сверхсильные корреляции.
\end{itemize}

\subsection{Запутанность фотонов и глюонов}
Безмассовые бозоны также должны демонстрировать аномально высокую степень корреляций.
Эксперименты:
\begin{itemize}
  \item лабораторные опыты с запутанными фотонами (Bell-тесты, спутниковые коммуникации);
  \item поиск квантовых нелокальных эффектов в глюонных струях на коллайдерах.
\end{itemize}

\subsection{Высокие моды и тёмная материя}
Высокие собственные значения $\lambda_k$ интерпретируются как кандидаты на тёмную материю.
Возможные методы обнаружения:
\begin{enumerate}
  \item \textbf{Космологические эффекты.} Дополнительные моды вносят вклад в плотность тёмной материи $\Omega_{\mathrm{DM}}$, что отражается в спектре космического микроволнового фона.
  \item \textbf{Гравитационное линзирование.} Высокие моды проявляются через аномалии в распределении массы в галактиках.
  \item \textbf{Резонансы на ускорителях.} При энергиях $E \sim \lambda_k$ могут наблюдаться пики в спектре сечений:
  \[
  \sigma(E) \propto \frac{1}{(E^2 - \lambda_k^2)^2 + \Gamma_k^2},
  \]
  где $\Gamma_k$ --- ширина моды.
  \item \textbf{Подземные детекторы.} В редких случаях высокие моды могут взаимодействовать с ядрами вещества, аналогично поискам WIMP.
\end{enumerate}

\subsection{Сравнительная таблица предсказаний}

\begin{table}[h!]
\centering
\begin{tabular}{|c|c|c|}
\hline
\textbf{Эффект} & \textbf{ZFSC} & \textbf{Стандартная модель} \\
\hline
Запутанность нейтрино & Очень высокая ($S_\nu \gg 0$) & Не описывается явно \\
Запутанность фотонов & Аномально устойчивая & Ограничена экспериментами Белла \\
Тёмная материя & Высокие моды спектра & Вводится феноменологически (WIMP, axion) \\
Гравитон & Нулевая мода $\lambda_0 \approx 0$ & Гипотетический квант поля \\
\hline
\end{tabular}
\caption{Сравнение ключевых предсказаний ZFSC и стандартной модели.}
\end{table}


\end{document}
