\section{Comparison: ZFSC v1.0 vs ZFSC v2.0}

To clarify the conceptual progress, we summarize the main differences between the first formulation of ZFSC and the extended version presented here.

\begin{table}[h!]
\centering
\begin{tabular}{@{}p{4cm}p{5.5cm}p{5.5cm}@{}}
\toprule
 & \textbf{ZFSC v1.0} & \textbf{ZFSC v2.0} \\
\midrule
\textbf{State space} & Only matter modes $|i_{\mathrm{matter}}\rangle$ & Tensor product of matter and quantized geometry $|i_{\mathrm{matter}}\rangle \otimes |j_{\mathrm{geom}}\rangle$ \\
\textbf{Geometry} & Treated as fixed background & Quantized via Fibonacci, golden Laplacian, $q$-Fibonacci, fractal or random matrices \\
\textbf{Parameters} & $(\Delta, r, g_L, g_R, h_1,h_2,h_3)$ & Extended: $(\Delta, r, g_L, g_R, h_1,h_2,h_3, \alpha, \beta, n_*(t))$ \\
\textbf{Entanglement} & Not included & Explicit corrections $\Delta E_s = \alpha I_{AB} + \beta I_{\mathrm{intra}}$ \\
\textbf{Cosmic age} & Not linked to spectrum & Encoded as Fibonacci approximation depth $n_*(t)$ \\
\textbf{Inverse problem} & Forward-only computation of spectra & Inverse eigenvalue problem: reconstruct $H_{\mathrm{geom}}$ from experimental masses and mixing \\
\textbf{Spectral law} & $d\lambda_n/dt = 0$ (stability) & Same, but applied to joint matter+geometry eigenvalues \\
\bottomrule
\end{tabular}
\caption{Conceptual evolution of ZFSC from v1.0 to v2.0.}
\end{table}

\section{Advantages of ZFSC v2.0}
\begin{itemize}
    \item \textbf{Unification:} Matter and geometry are treated on equal footing as quantized states.  
    \item \textbf{Flexibility:} Multiple candidate geometries (Fibonacci, random, fractal) can be tested within the same framework.  
    \item \textbf{Observational anchoring:} Direct reconstruction from experimental data (fermion masses, CKM/PMNS).  
    \item \textbf{Cosmological link:} The age of the Universe is naturally encoded as approximation depth $n_*(t)$ in Fibonacci scaling.  
    \item \textbf{Testability:} The Fibonacci hypothesis becomes falsifiable: if nature does not favor golden-ratio quantization, the weight $c_{\mathrm{Fib}}$ vanishes.  
\end{itemize}

\section{Open Problems}
\begin{enumerate}
    \item \textbf{Numerical implementation:} Efficient algorithms for solving the inverse eigenvalue problem with large-scale matrices remain challenging.  
    \item \textbf{Physical interpretation:} The exact meaning of negative eigenmodes (tachyons) and their role in cosmology needs clarification.  
    \item \textbf{Entanglement parameters:} The constants $\alpha, \beta$ in $\Delta E_s$ require physical derivation or experimental constraint.  
    \item \textbf{Cosmic evolution:} The precise mapping $n_*(t)$ from cosmological time to approximation depth must be calibrated with data (CMB, structure formation).  
    \item \textbf{Universality:} Whether Fibonacci structures are unique, or just one possible realization among many quasiperiodic or fractal candidates.  
    \item \textbf{Experimental signals:} Identifying direct experimental observables beyond fermion masses and mixing (e.g., subtle patterns in bosonic spectra).  
\end{enumerate}
