\documentclass[12pt,a4paper]{article}
\usepackage[utf8]{inputenc}
\usepackage{amsmath}
\usepackage{geometry}
\geometry{margin=2.5cm}

\title{Note on a Timing Invariant for CMS LLP Searches}
\author{Independent researcher}
\date{\today}

\begin{document}
\maketitle

\noindent
A potential linear invariant in long-lived particle (LLP) timing may be expressed as:

\[
\mathrm{Var}\big(t - \alpha\cdot d_0\big) \;\text{is minimized for a universal } \alpha,
\]

where $t$ is the arrival time and $d_0$ the impact parameter.

\textbf{How it can be tested:}  
CMS has collected Run-3 LLP candidates with advanced timing capabilities (ECAL, MIP timing). One can test whether a linear combination of $t$ and $d_0$ shows a minimal variance across LLP classes.

\textbf{Why it may be important:}  
Such a pattern would suggest an unexpected coherence in LLP kinematics and could provide a new discriminant for signal versus background.

\end{document}
