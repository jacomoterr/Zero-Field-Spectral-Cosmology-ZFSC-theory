\documentclass[12pt,a4paper]{article}
\usepackage[utf8]{inputenc}
\usepackage{amsmath}
\usepackage{geometry}
\geometry{margin=2.5cm}

\title{Note on the Absence of an $E^{-4}$ Decoherence Trend in IceCube Data}
\author{Independent researcher}
\date{\today}

\begin{document}
\maketitle

\noindent
A possible check in high-energy neutrino data is the absence of a quartic energy dependence in decoherence:

\[
\Gamma(E) \not\propto E^{-4}, \quad E \in (0.5\text{–}10~\mathrm{TeV}).
\]

\textbf{How it can be tested:}  
Using atmospheric neutrinos at TeV energies, IceCube can extract effective decoherence parameters. A systematic search for $E^{-4}$ scaling should yield null results within uncertainties.

\textbf{Why it may be important:}  
Excluding quartic scaling would constrain classes of exotic decoherence models and strengthen the case for long-range quantum coherence of neutrinos.

\end{document}
