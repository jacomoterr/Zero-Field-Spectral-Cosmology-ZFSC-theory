
\documentclass[a4paper,12pt]{article}
\usepackage{geometry}
\geometry{margin=2.5cm}

% --- Язык и шрифты (компиляция XeLaTeX/LuaLaTeX) ---
\usepackage[russian]{babel}
\usepackage{fontspec}
\usepackage{unicode-math}
\setmainfont{CMU Serif}
\setmathfont{Latin Modern Math}

% --- Математика и рисунки ---
% ---  \usepackage{amsmath,amssymb} ---
\usepackage{tikz}
\usepackage{pgfplots}
\pgfplotsset{compat=1.18}
\everymath{\displaystyle}

\title{Лог-периодическая модуляция видимости в двухщелевом эксперименте с электронами\\[2mm]
\large Краткая фальсифицируемая гипотеза и протокол репликаций}
\author{Евгений Монахов \\ Независимый исследователь, г.Москва \\ evgeny.monakhov@voscom.online}
\date{\today}

\begin{document}
\maketitle

\begin{abstract}
Предлагается короткая, полностью фальсифицируемая гипотеза для стандартной двухщелевой
интерференционной установки с электронами (а также для электронных интерферометров на бипризме).
Суть: при сканировании по энергии электронов $E$ \,или при масштабном рескейлинге геометрии установки
$a,d,L \to s\,a,\,s\,d,\,s\,L$ наблюдаемая видимость интерференционных полос $V$ получает
\emph{слабую лог-периодическую модуляцию} с одной и той же лог-частотой $\omega$:
\begin{equation}
\label{eq:main}
V(E)=V_0\!\left[1+\beta\,\cos\!\Big(\tfrac{\omega}{2}\,\ln\!\frac{E}{E_{\ast}}+\phi\Big)\right],
\qquad
V(s)=V_0\!\left[1+\beta\,\cos\!\big(\omega\,\ln\!\tfrac{s}{s_{\ast}}+\phi\big)\right].
\end{equation}
Здесь $V_0$ --- базовая видимость, $\beta\ll 1$ --- относительная амплитуда, $\omega$ --- лог-частота (\emph{одна и та же}
для обоих сканов), $\phi$ --- фаза, $E_{\ast},s_{\ast}$ --- опорные масштабы. Подтверждение~(\ref{eq:main})
с \emph{одной и той же} $\omega$ для различных масок/геометрий укажет на устойчивый лог-масштабный отпечаток
в свободной электронной интерференции. Отрицательный результат при заявленной точности однозначно опровергает гипотезу.
\end{abstract}

\section{Введение: почему это важно и на что это указывает}
Двухщелевой опыт с электронами --- краеугольный эксперимент квантовой физики: отдельные электроны дают
интерференционную картину, а видимость полос $V$ чувствительна к когерентности, геометрии и среде.
Обычно анализ ведут по зависимостям на \emph{линейной} шкале энергии/геометрии и не ищут
мелкие эффекты, проявляющиеся при анализе по \emph{логарифмической} шкале.

Если у видимости $V$ существует \emph{лог-периодическая} составляющая (косинус от логарифма энергии или масштаба),
то это указывает на присутствие \textbf{узкой дискретной масштабной симметрии} \emph{или}
эквивалентного \textbf{маргинального канала}, который даёт слабую логарифмическую поправку при изменении масштаба.
Это \emph{не} противоречит стандартной интерференции, а добавляет тонкий инвариант, который до сих пор
не был целенаправленно проверен. Наличие такого инварианта, одинакового в энергетическом и геометрическом сканах,
было бы важным фактом для тонких тестов квантовой интерференции и смежных волновых систем.

\paragraph{Критерий ценности.}
Гипотеза легко проверяема на существующих установках; возможен реанализ архивных данных.
Положительный результат откроет новое направление точных масштабных тестов (электроны, фотоны, акустика),
\emph{без} привязки к какой-либо специфической модели. Отрицательный результат задаст строгие пределы на подобные эффекты.

\section{Постановка задачи и ключевая гипотеза}
Рассматривается двухщелевая геометрия с параметрами: ширина щели $a$, межщелевое расстояние $d$,
расстояние до детектора (экрана) $L$. Поперечная координата на детекторе --- $x$.
Для электрона с энергией $E$ длина де Бройля $\lambda\propto E^{-1/2}$.
В параксиальном приближении базовая интенсивность (без тонких эффектов) записывается как
\begin{equation}
I_0(x;E)=I_{\mathrm{ref}}(x;E)\,\Bigl[1+V_0\,\cos\!\Bigl(2\pi\,\frac{d\,x}{\lambda L}\Bigr)\Bigr],
\end{equation}
где $I_{\mathrm{ref}}$ --- медленно меняющаяся огибающая (дифракция одиночной щели, профиль пучка).
\emph{Видимость} полос $V$ определяется стандартно:
\begin{equation}
V=\frac{I_{\max}-I_{\min}}{I_{\max}+I_{\min}}.
\end{equation}

\noindent
\textbf{Гипотеза.} Помимо стандартной структуры, $V$ получает \emph{слабую лог-периодическую модуляцию}
вида~(\ref{eq:main}). Ключевые проверяемые утверждения:
\begin{itemize}
  \item (H1) При скане по энергии $E$ наблюдается гармоника по $\ln E$ с частотой $\omega/2$.
  \item (H2) При масштабировании геометрии $s$ наблюдается гармоника по $\ln s$ \emph{с той же} частотой $\omega$,
  а фаза сдвигается предсказуемо: $\phi\to\phi+\omega\ln s$.
  \item (H3) Для одиночной щели лог-гармоника отсутствует в пределах статистической погрешности.
\end{itemize}

\section{Пояснение параметров формулы~(\ref{eq:main})}
\begin{itemize}
\item \textbf{$V_0$} --- базовая видимость (определяется когерентностью источника, качеством маски, коллимацией,
механическими и электрическими дрейфами). Оценивается локально (по скользящему окну) до извлечения лог-гармоники.
\item \textbf{$\beta$} --- \emph{относительная амплитуда} лог-модуляции. Ожидаемый порядок: $\beta\sim10^{-3}\ldots10^{-2}$
при аккуратной стабилизации. Это соответствует «тонкой ряби» поверх базовой видимости.
\item \textbf{$\omega$} --- \emph{лог-частота} (частота гармоники по аргументу $\ln X$). Ключевое требование гипотезы:
\emph{одна и та же} $\omega$ извлекается как из $V(\ln E)$, так и из $V(\ln s)$ и \emph{не зависит} от конкретных $a,d,L$.
\item \textbf{$\phi$} --- фаза лог-гармоники; при геометрическом рескейле $\phi$ сдвигается как $\phi\to\phi+\omega\ln s$.
\item \textbf{$E_{\ast}, s_{\ast}$} --- опорные масштабы (свободные константы), фиксируют начало отсчёта фазы при переходе к логарифму.
\end{itemize}
\noindent
\textit{Замечание.} Множитель $\tfrac{\omega}{2}$ в энергетическом скане обусловлен зависимостью
$\lambda\propto E^{-1/2}$.

\section{Экспериментальная установка и требования}
\textbf{Вакуум:} $\le 10^{-7}$ mbar. \quad
\textbf{Стабильность энергии:} $\Delta E/E\lesssim10^{-3}$. \quad
\textbf{Антивибрация и экранирование:} подавление механики и электростатики.\\
\textbf{Маски:} несколько наборов $(a,d)$; дополнительно --- электронная бипризма (как независимый контроль).
\textbf{Детектор:} ПЗС/МПД с известной линейностью; стабильная геометрия (или калибруемый масштаб).

\section{Протокол измерений}
\subsection{Энергетический скан $V(E)$}
Фиксировать геометрию и один/несколько пикселей $x_{\ast}$ (например, по максимуму/минимуму с номером $m$).
Собрать $N\sim 300\ldots 500$ точек $V(E)$ на \emph{равномерной сетке по} $\ln E$ в диапазоне $[E_{\min},E_{\max}]$.
Каждую точку интегрировать до статистической ошибки $\sigma_V\lesssim10^{-3}$. Сопровождать записью служебных каналов
(дрейф, ток эмиссии, температура).

\subsection{Геометрический скан $V(s)$}
Масштабировать $a,d,L$ общей факторой $s$ (или эквивалентно менять $L$ и компенсировать масштаб линзой/магнификацией),
при фиксированном $E$. Получить 10--20 точек $V(s)$ на \emph{равномерной сетке по} $\ln s$.
Проверить воспроизведение той же $\omega$ и фазовый закон $\phi\to\phi+\omega\ln s$.

\section{Контроль систематики и фальсификация}
\textbf{Одиночная щель.} Лог-гармоника должна исчезать в пределах погрешности.\\
\textbf{Замена маски/материала.} Извлечённая $\omega$ сохраняется; $\phi$ и $V_0$ могут меняться.\\
\textbf{Поворот маски на $90^{\circ}$.} $\omega$ сохраняется; $\phi$ меняется предсказуемо.\\
\textbf{Фальсификация.} (i) отсутствие узкой гармоники по $\ln E$; (ii) значимая зависимость $\omega$ от $(a,d,L)$;
(iii) несогласие $\omega_E$ и $\omega_s$ в пределах ошибок.

\section{Анализ данных: извлечение $\omega,\beta,\phi$}
Обозначим $x=\ln(E/E_{\ast})$ (или $x=\ln(s/s_{\ast})$). Рекомендуем следующую процедуру:
\begin{enumerate}
  \item \textbf{Предобработка.} Ресемплинг $V$ на равномерную сетку по $x$; мягкая детрендировка (удаление медленной составляющей),
  без подавления узких гармоник.
  \item \textbf{Гармоническая регрессия.} Фит $V(x)\approx C+A\cos(\alpha x)+B\sin(\alpha x)$ на сетке $\alpha\in[\alpha_{\min},\alpha_{\max}]$.
  Выбор $\alpha$ по минимуму остаточной дисперсии. Итог: $\omega=\alpha$, $\phi=\mathrm{atan2}(-B,A)$, $\beta=\sqrt{A^2+B^2}/C$.
  \item \textbf{Оценка ошибок.} Бутстрэп/перестановочные тесты; сравнение с моделью без гармоники (AIC/BIC).
  \item \textbf{Согласие сканов.} Сравнить $\omega_E$ и $\omega_s$, проверить фазовый закон $\phi_s\approx\phi_E+\omega\ln s$.
\end{enumerate}
Альтернатива пункту 2 --- периодограмма Ломба–Скаргла по $x$; результат должен совпадать в пределах ошибок.

\section{Чувствительность: порядки величин и счёт статистики}
Пусть дисперсия единичных оценок видимости $\mathrm{Var}[V]\approx\sigma_V^2$ (доминирует счётная статистика).
Для регрессии на $M$ точках по $x$ стандартная ошибка амплитуды гармоники $\propto\sigma_V/\sqrt{M/2}$.
Чтобы надёжно увидеть $\beta\sim5\times10^{-3}$ на уровне SNR$\sim5$, при $M\sim400$ нужно $\sigma_V\lesssim10^{-3}$,
что достижимо при интеграциях $\mathcal{O}(\text{секунды})$ на точку и стабильном источнике. Точный бюджет \emph{а-постериори}
оценивается бутстрэпом на ваших данных.

\section{Реанализ архивных данных}
Если доступны серии $V(E)$ (или исходные профили интенсивности, позволяющие восстановить $V$), достаточно:
(i) перейти к равномерной сетке по $\ln E$; (ii) применить гармоническую регрессию/периодограмму;
(iii) опубликовать извлечённые $\omega,\beta,\phi$ и их ошибки. Отрицательные результаты важны наряду с положительными.

\section{Что означает положительный/отрицательный результат}
\textbf{Положительный:} существует повторяемая лог-периодическая модуляция видимости с универсальной $\omega$,
совпадающей в энергетическом и геометрическом сканах и инвариантной к замене маски/материала. Это указывает
на присутствие \emph{узкой дискретной масштабной структуры} или маргинального канала в эффективном описании интерференции.
\textbf{Отрицательный:} при заданной чувствительности устанавливаются строгие пределы на возможные лог-модуляции
в стандартной двухщелевой схеме; будущие работы могут улучшать пределы.

\section{Дорожная карта репликаций}
(1) Повторить оба скана на нескольких установках (двухщель, бипризма). \\
(2) Проверить фазовый закон при множественных рескейлах $s$. \\
(3) Исследовать устойчивость $\omega$ к внешним полям (магнитное/электрическое), температуре, материалам маски. \\
(4) Провести аналогичный тест в фотонных и акустических интерферометрах.

\section*{Сводка}
Предложен конкретный, компактный и фальсифицируемый тест: лог-периодическая гармоника по $\ln E$ и $\ln s$
с одной и той же частотой $\omega$ в видимости двухщелевой интерференции электронов. Это можно проверить
на стандартных установках и переанализом архивных данных. Результат --- какой бы он ни был --- имеет самостоятельную ценность.

\section*{Благодарности}
Автор благодарит коллег за обсуждения. Конфликт интересов отсутствует.

\section*{Краткая библиография для ориентира}
\begin{itemize}
  \item Y.~Tonomura et al., \emph{Observation of single-electron interference patterns} (классические эксперименты).
  \item Возможные обзоры по электронной бипризме и интерферометрии (Hasselbach и др.).
\end{itemize}

\end{document}
