\documentclass[a4paper,12pt]{article}
\usepackage{geometry}
\geometry{margin=2.5cm}

\usepackage[russian]{babel}
\usepackage{fontspec}
\usepackage{unicode-math}
\setmainfont{CMU Serif}
\setmathfont{Latin Modern Math}

\usepackage{amsmath}
\everymath{\displaystyle}
\emergencystretch=2em

\title{Энтропийно-циклический предиктор глитчей пульсаров}
\author{Евгений Монахов \\ Независимый исследователь \\ VOSCOM ONLINE}
\date{}

\begin{document}
\maketitle

\section{Введение}
Глитчи пульсаров --- внезапные скачки частоты вращения нейтронных звёзд. 
Несмотря на десятилетия наблюдений, природа глитчей остаётся открытой проблемой. 
Особенно актуален вопрос: можно ли предсказывать время и величину глитчей на основе 
предварительных данных о вращении.

В данной работе предлагается простая квантово-статистическая модель, 
использующая энтропийные и циклические признаки тайминговых рядов для предсказания 
момента и величины глитча. 
Подход не требует сложных физических моделей внутреннего строения звезды и 
опирается только на обработку данных наблюдений.

\section{Энтропийная компонента}
Пусть $\nu(t)$ --- наблюдаемая частота вращения пульсара, заданная на равномерной сетке времени. 
Рассмотрим первые разности:
\[
\Delta_k = \nu(t_{k+1}) - \nu(t_k).
\]
Нормируем их на локальное среднее (на окне длины $W$):
\[
s_k = \frac{\Delta_k}{\langle \Delta \rangle_W}.
\]
Строим гистограмму значений $s_k$ с $B$ бинами и вероятностями $p_b$. 
Энтропия Шеннона:
\[
H = - \sum_{b=1}^B p_b \ln p_b.
\]
Нормируем:
\[
h = \frac{H}{\ln B}, \qquad h \in [0,1].
\]
$h$ характеризует степень хаотичности вращения.

\section{Циклическая компонента}
Для того же окна длиной $M$ точек рассмотрим дискретные гармоники с периодами 
$P \in \{8,16,32,64\}$. Определим взвешенные коэффициенты:
\[
A_P = \frac{1}{M} \left| \sum_{k=1}^M w_k e^{2\pi i k / P} \right|,
\]
где $w_k$ --- нормированные данные (например, $w_k = (\nu(t_k)-\nu(t_1))/(|\nu(t_M)-\nu(t_1)|+\varepsilon)$).
Определим интегральный показатель цикличности:
\[
C = \frac{1}{4}\sum_{P \in \{8,16,32,64\}} \frac{A_P}{A_P+\varepsilon}.
\]

\section{Интегральный фактор предсказания}
Определим безразмерный фактор
\[
\Omega(t) = \exp\left(-\tfrac12 \,[h(t) + C(t)]\right).
\]
Интервал значений: $0 < \Omega \leq 1$. 
Чем меньше $\Omega$, тем больше «готовность системы» к глитчу.

\section{Предсказание вероятности глитча}
Определим интенсивность глитчей как
\[
\lambda(t) = \lambda_0 \exp\!\left[\beta \left(\frac{1}{\Omega(t)} - 1\right)\right],
\]
где $\lambda_0$ --- базовая частота глитчей данного пульсара (можно оценить по статистике), 
$\beta \sim 1$. Тогда вероятность глитча в ближайшем окне $\Delta t$:
\[
P_{\rm glitch}(t,\Delta t) = 1 - \exp\!\left(-\int_t^{t+\Delta t} \lambda(\tau) \, d\tau\right).
\]

\section{Предсказание величины глитча}
Введём характеристику локального «разрыва»:
\[
g(t) = \frac{\operatorname{median}(|\Delta_{k+1}-\Delta_k|)}{\operatorname{median}(|\Delta_k|)+\varepsilon}.
\]
Ожидаемая величина глитча при его возникновении:
\[
\mathbb{E}\!\left[\frac{\Delta \nu}{\nu}\,\big|\,\text{glitch at }t\right] \approx 
K \cdot \frac{1}{\Omega(t)} \cdot \frac{g(t)}{1+g(t)},
\]
где $K$ --- масштабный коэффициент, фиксируемый по первому крупному глитчу данного пульсара 
и далее неизменный.

\section{Алгоритм проверки на данных}
\begin{enumerate}
\item Взять исторические тайминговые ряды $\nu(t)$.
\item Выбрать окно $W=30$--$60$ дней и шаг 1--5 дней.
\item Для каждого окна посчитать $h(t), C(t), \Omega(t)$.
\item Рассчитать $P_{\rm glitch}(t,\Delta t)$ для $\Delta t=30$ дней.
\item Построить «график готовности» пульсара к глитчу.
\item При возникновении глитча сравнить предсказанную и наблюдаемую величины.
\end{enumerate}

\section{Заключение}
Предложенный энтропийно-циклический предиктор глитчей пульсаров основан только на 
обработке тайминговых рядов. 
Он позволяет прогнозировать вероятность и примерную величину глитча на 
основе квантово-статистических признаков вращения. 
Методика доступна для быстрой проверки на архивных данных и может дать 
новые эмпирические корреляции, которые в дальнейшем помогут в построении 
физических моделей внутреннего строения нейтронных звёзд.

\vspace{2em}
\noindent
\textit{Евгений Монахов} \\
Независимый исследователь \\
VOSCOM ONLINE

\end{document}
