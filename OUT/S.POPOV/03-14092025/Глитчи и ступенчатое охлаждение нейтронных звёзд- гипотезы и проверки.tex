\documentclass[a4paper,12pt]{article}
\usepackage{geometry}
\geometry{margin=2.5cm}

% Языки
\usepackage[russian]{babel}

% --- Шрифты ---
\usepackage{fontspec}
\usepackage{unicode-math}
\setmainfont{CMU Serif}
\setmathfont{Latin Modern Math}

% --- Математика ---
\usepackage{amsmath}
\everymath{\displaystyle}
\emergencystretch=2em

% --- Графика ---
\usepackage{tikz}
\usepackage{pgfplots}
\pgfplotsset{compat=1.18}

\title{Глитчи и ступенчатое охлаждение нейтронных звёзд:\\
гипотезы и возможные проверки}
\author{Евгений Монахов \\ Независимый исследователь \\ VOSCOM ONLINE}
\date{}

\begin{document}
\maketitle

\section*{Аннотация}
Предварительный анализ доступных каталогов (кривые охлаждения, данные по глитчам)
показывает признаки дискретных переходов и возможных закономерностей.
Ниже изложены гипотезы и предлагаемые проверки, которые могут уточнить природу
глитчей и ступенчатого охлаждения нейтронных звёзд.

\section{Ступенчатость охлаждения}
Даже в минимально обработанных кривых охлаждения (без сильного сглаживания)
заметны плато и скачки температуры. Такие данные представляют собой 
вычисленные значения эффективной температуры $T_{\mathrm{eff}}(t)$,
извлечённые из рентгеновских наблюдений, но ещё не сглаженные постобработкой.  
Их можно описать кусочно-постоянной функцией:
\[
T(t) \approx T_n, \quad t \in [t_n, t_{n+1}),
\]
где $T_n$ --- значение температуры на $n$-м плато.
Переходы фиксируются скачками
\[
\Delta T_n = T_{n+1} - T_n \neq 0.
\]

\section{Глитчи и их распределение}
Каталожные значения относительных изменений частоты вращения определяются как
\[
\frac{\Delta \nu}{\nu} = \frac{\nu(t^{+}) - \nu(t^{-})}{\nu(t^{-})}.
\]
Распределения $\Delta\nu/\nu$ указывают на группировки вблизи характерных значений.
Гипотеза: существует квантизация порядка $k=8,16$:
\[
\frac{\Delta \nu}{\nu} \approx k \cdot \delta, \quad k \in \{8,16\},
\]
где $\delta$ --- минимальный шаг.

\section{Несовпадение моментов плато и глитчей}
Для разных НЗ плато охлаждения и моменты глитчей не совпадают.
Вероятное объяснение связано с различиями в массе $M$, магнитном поле $B$
и фазовом состоянии ядра. Времена переходов можно описать качественно:
\[
\tau_{\text{перехода}} \sim f(M,B,\rho_{\text{ядра}}).
\]

\section{Физические интерпретации}

\subsection{Эффективная масса}
Расчётные значения массы могут быть занижены из-за неучёта внутренних фазовых переходов.
Эффективная масса:
\[
M_{\text{eff}} = \int \left( \rho(r) + \frac{E_{\text{phase}}(r)}{c^{2}} \right)\, dV,
\]
что даёт $M_{\text{eff}} > M_{\text{model}}$.

\subsection{Релаксация после глитча}
После скачка частоты наблюдается плавное снижение:
\[
\Delta \nu(t) = \Delta \nu_0 \, e^{-t/\tau},
\]
где $\Delta \nu_0$ --- величина скачка, $\tau$ --- характерное время релаксации.
Это соответствует перераспределению углового момента между сверхтекучим ядром и оболочкой.

\section{Предлагаемые проверки}
\begin{enumerate}
  \item Анализ кривых охлаждения в минимально обработанном виде (без сильного сглаживания) для поиска дискретных плато.
  \item Корреляции между скачками температуры и глитчами.
  \item Байесовский анализ распределений $\Delta\nu/\nu$ для проверки гипотезы о кратностях (8,16).
  \item Сравнение разных НЗ как объектов с разными конфигурациями, а не как шумовых реализаций.
  \item Проверка аналогичных кратностей в QPO и данных по гравитационным волнам.
\end{enumerate}

\section*{Заключение}
Предложенные гипотезы указывают на возможную дискретную природу процессов внутри НЗ.
Подтверждение или опровержение их на реальных данных может дать ключ к пониманию
как глитчей, так и ступенчатого охлаждения.

\section{Комментарий о типе данных}
Под «несглаженными» или «минимально обработанными» данными здесь понимаются
опубликованные кривые охлаждения и каталоги глитчей, приведённые в литературе
или архивах миссий (например, Chandra, XMM-Newton, NICER, Jodrell Bank).
Это значения, уже переведённые в физические величины
($T_{\mathrm{eff}}(t)$, $\Delta\nu/\nu$), но не прошедшие дополнительную
сильную аппроксимацию или сглаживание.  

Важно подчеркнуть: речь не идёт о «сырых» телескопных массивах (event lists,
raw spectra, raw counts), доступ к которым ограничен коллаборациями.
\```



\vspace{2em}
\noindent
\textit{Евгений Монахов} \\
Независимый исследователь \\
VOSCOM ONLINE

\end{document}
