\documentclass[a4paper,12pt]{article}
\usepackage{geometry}
\geometry{margin=2.5cm}

% Language
\usepackage[english]{babel}

% Fonts
\usepackage{fontspec}
\usepackage{unicode-math}
\setmainfont{CMU Serif}
\setmathfont{Latin Modern Math}

% Math
\usepackage{amsmath}
\everymath{\displaystyle}
\emergencystretch=2em

\begin{document}

\noindent
To the LIGO Scientific Collaboration \\
to the Virgo Collaboration \\
to the KAGRA Collaboration \\[2ex]

\noindent
\textbf{Subject: Proposal to search for additional spectral peaks in neutron star merger signals} \\[2ex]

Dear Colleagues,

Recent analyses of neutron star mergers (e.g., GW170817 and subsequent NS--NS events) have provided unprecedented insights into the dense-matter equation of state. We would like to draw your attention to a potential feature that may be hidden in the gravitational-wave data.

Our spectral view of dense matter suggests that, beyond the classical inspiral ``chirp'' and the dominant post-merger ringdown, there may exist \emph{additional narrow spectral peaks} associated with internal transitions between quasi-stable states of ultra-dense matter. To keep this suggestion strictly phenomenological and model-agnostic, we outline below a minimal signature that could be searched for without committing to a particular microphysical model.

\paragraph{Phenomenological signature (model-agnostic).}
Let $h_{\rm GR}(t)$ denote the baseline general-relativistic waveform (inspiral + merger + dominant ringdown). We suggest testing the augmented ansatz
\[
h(t) \;=\; h_{\rm GR}(t) \;+\; \sum_{k=1}^{K} A_k\, e^{-t/\tau_k}\,\sin\!\bigl(2\pi f_k\, t + \phi_k \bigr)\, \Theta(t-t_0),
\]
with corresponding frequency-domain excess power modeled by a sum of (damped-mode) Lorentzians,
\[
S_h(f) \;=\; S_{h,{\rm GR}}(f) \;+\; \sum_{k=1}^{K} \frac{H_k}{1 + Q_k^2 \bigl(\tfrac{f}{f_k} - \tfrac{f_k}{f}\bigr)^2}\,,
\]
where $f_k$ are narrow peaks near (but distinct from) the dominant post-merger frequency $f_{\rm RD}$, $\tau_k$ are damping times, $Q_k \!=\! \pi f_k \tau_k$ are quality factors, and $H_k$ are peak heights. The Heaviside step $\Theta$ and onset time $t_0$ allow both late-inspiral (pre-merger micro-oscillations) and early post-merger excitation.

\paragraph{Search ranges (practical guidance).}
\begin{itemize}
  \item \textbf{Frequencies:} $f_k \in \bigl[0.7,\,1.4\bigr]\times f_{\rm RD}$ (typically in the $\sim\!1$--$4$ kHz band for NS--NS).
  \item \textbf{Quality factors:} $Q_k \sim 5$--$50$ (narrow but resolvable with current PSD integration times).
  \item \textbf{Durations:} $\tau_k \sim 5$--$100$ ms (short-lived post-merger ``breathing'' modes), with an allowance for subdominant pre-merger activity within the last $\sim\!100$ ms of inspiral.
  \item \textbf{Amplitudes:} $A_k$ (or $H_k$) $\ll$ dominant mode; persistence across multiple events would be the key discriminator vs. noise.
\end{itemize}

We suggest a targeted spectral re-analysis of existing NS--NS events (GW170817, GW190425, etc.) with:
(i) multi-resolution time-frequency methods (wavelet dictionaries, Prony/ESPRIT); 
(ii) constrained matched filtering against the above damped-sinusoid augmentations; 
(iii) cross-event stacking in normalized frequency units $f/f_{\rm RD}$ to enhance weak, reproducible peaks.

Even marginal evidence of such peaks---if consistent across events and instruments---would provide a new observable connecting gravitational-wave astronomy with the microphysics of supranuclear matter.

We would be grateful if the collaborations could consider this line of analysis. We remain available to clarify the phenomenology and parameter ranges if useful.

With respect and best regards, \\[2ex]

Evgeny Monakhov \\
Independent Researcher \\
VOSCOM ONLINE \\
evgeny.monakhov@voscom.online

\end{document}
